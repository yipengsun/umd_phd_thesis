\chapter{Rest frame approximation}
\label{appx:rfa}

As discussed in the main text,
the rest frame variables, namely \mmSq, \el, \qSq, cannot be computed exactly
in a hadron collider because the $B$ rest frame is unknown\footnote{
    As a reminder, the dominating \bbbar production mechanism in hadron
    colliders is gluon fusion,
    in which case the gluon parton momenta are unknown.
}.
To estimate these variables, an assumption is made:
the \emph{estimated proper velocity}, $\gamma \beta$,
of the $B$ meson along the beam direction ($z$ direction) is the same as the
visible one,
which leads to a relationship between the estimated and visible $B$ momenta
in the $z$ direction:

\begin{equation}
    (p_B)_z = \frac{m_B}{m_\text{vis}}(p_\text{vis})_z
\end{equation}
at LHCb, the $B$ decay vertex is measured to high precision.
Therefore, the $B$ flight direction (orange in \cref{fig:rfa}) and its angle
respect to the $z$ axis (blue $\alpha$ in the same figure) are both
well-measured, which leads to the estimated $B$ momentum:

\begin{equation}
    |p_B| = \frac{m_B}{m_\text{vis}} (p_\text{vis})_z \sqrt{1 + \tan^2\alpha}
\end{equation}
pointing in the $B$ flight direction.


\begin{figure}
    \centering
\resizebox{0.6\textwidth}{!}{
\begin{tikzpicture}[particle/.style={draw, ->, >=stealth, thick}]
    \node (a0) at (0, 0) {};
    \node[right=2.5em of a0] (a1) {$p$ ($z$ axis)};
    \node[left=2.5em of a0] (a2) {$p$};

    \coordinate[above=2em of a1] (b1);
    \coordinate[below=2em of a2] (b2);
    \coordinate[below=0.9em of b1] (b3);

    \node[left=2em of b1, gray] (d1) {$\overline\nu_\tau$};
    \coordinate[above right=1.5em and 1.5em of b1] (d2);
    \coordinate[below right=0.5em and 2.5em of b1] (d3);

    \node[above left=0.5em and 1em of d2, gray] (e1) {$\overline\nu_l$};
    \node[above right=0.7em and 0.4em of d2, gray] (e2) {$\nu_\tau$};
    \node[above right=0.1em and 2.3em of d2, red] (e3) {$l^-$};

    \coordinate[above right=0.2em and 1.2em of d3] (f1);
    \coordinate[below right=1em and 0.4em of d3] (f3);

    \draw [particle] (a1) -- (a0);
    \draw [particle] (a2) -- (a0);

    \draw [particle, orange] (a0) -- (b1) node[midway, left, xshift=-5pt] {$B$ flight dir};
    \draw [particle, red] (a0) -- (b3) node[midway, right, xshift=8pt] {visible};

    \draw [particle, gray] (b1) -- (d1);
    \draw [particle, red, dashed] (b1) -- (d2) node[midway, left] {$\tau^-$};
    \draw [particle, red, dashed] (b1) -- (d3) node[midway, above, xshift=6pt] {$D^0$};

    \draw[particle, gray] (d2) -- (e1);
    \draw[particle, gray] (d2) -- (e2);
    \draw[particle, red] (d2) -- (e3);

    \draw[particle, red] (d3) -- (f1);
    \draw[particle, red] (d3) -- (f3);

    \draw pic["\tiny$\textcolor{blue}{\alpha}$",
              draw=blue,thick,-,angle eccentricity=1.3,angle radius=18pt,
              fill=blue,fill opacity=.5,text opacity=1]
        {angle=a1--a0--b1};
\end{tikzpicture}
}
    \caption{Schematic diagram for rest frame approximation (RFA).}
    \label{fig:rfa}
\end{figure}
