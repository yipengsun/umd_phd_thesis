\chapter{Form factor variation}
\label{appx:ff-var}

Suppose a form factor $\mathcal{F}$ can be estimated with:

\begin{equation}
    \mathcal{F}(z) \approx f_0 + f_1 z + f_2 z^2
\end{equation}

where $z$ is some expansion parameter, and $\bm{f} = (f_0, f_1, f_2)^T$ are
expansion coefficients.

The log-likelihood of the polynomial can be written as:

\begin{equation}
    \log L(\bm{f}) = - \frac{1}{2}
    \underbrace{
        \sum_{i}^N
        \frac{(\mathcal{F^\text{mea}_\text{$i$}} -
            \bm{z_i}^T \bm{f})^2}{\sigma_i^2}
    }_{\equiv \chi^2}
\end{equation}

where $\bm{z_i} = (1, z_i, z_i^2)^T$,
$\mathcal{F^\text{mea}_\text{$i$}}$ indicates each measured value of
$\mathcal{F}$,
and $\sigma_i$ the uncertainty associated with each measurement, assuming each
measurement is an independent Gaussian random variable.

To obtain best fit values of $\bm{f}$, we maximize $\chi^2$ by requiring
$\nabla_{\bm{f}} \chi^2 = \bm{0}$.
Noting $\chi^2$ can be rewritten more concisely in a tensorial form:

\begin{equation}
    \chi^2 = (\bm{\mathcal{F^\text{mea}}} - Z \bm{f})^T V^{-1}
        (\bm{\mathcal{F^\text{mea}}} - Z \bm{f})
\end{equation}

where $Z$ is a $N \times 3$ matrix with
$Z_i \equiv \bm{z_i} = (1, z_i, z^2_i)^T$,
and $V$ a $N \times N$ matrix with $V_{ij} = \delta_{ij} \sigma^2_j$.

Computing $\nabla_{\bm{f}} \chi^2$:

\begin{align}
    \nabla_{\bm{f}} \chi^2 &=
            -Z^T V^{-1} (\bm{\mathcal{F^\text{mea}}} - Z \bm{f})
            - (\bm{\mathcal{F^\text{mea}}} - Z \bm{f})^T V^{-1} Z
        \\
        %%%%
        &=
            -Z^T V^{-1} (\bm{\mathcal{F^\text{mea}}} - Z \bm{f})
            -Z^T \underbrace{(V^{-1})^T}_{ = V^{-1}}
                (\bm{\mathcal{F^\text{mea}}} - Z \bm{f})
        \\
        %%%%
        & = -2 Z^T V^{-1} (\bm{\mathcal{F^\text{mea}}} - Z \bm{f})
\end{align}

Therefore, the best fit ${\bm{\hat{f}}}$ are linear functions of the
measurements $\bm{\mathcal{F^\text{mea}}}$:

\begin{align}
    & -2 Z^T V^{-1} (\bm{\mathcal{F^\text{mea}}} - Z \bm{\hat{f}}) = 0 \\
    &\Longrightarrow Z^T V^{-1} Z \bm{\hat{f}}
        = Z^T V^{-1} \bm{\mathcal{F^\text{mea}}} \\
    &\Longrightarrow \bm{\hat{f}}
        = \underbrace{
            (Z^T V^{-1} Z)^{-1} Z^T V^{-1}
        }_{\equiv B} \bm{\mathcal{F^\text{mea}}}
\end{align}

Using error propagation to find uncertainties of $\bm{\hat{f}}$, denote as U:

\begin{equation}
    U = B V B^T = (Z^T V^{-1} Z)^{-1}
\end{equation}

In the case where each $\mathcal{F^\text{mea}_\text{$i$}}$ is independent and
identically distributed with a variance of $\sigma^2$,
$V^{-1} = \frac{1}{\sigma^2} \text{Id}_N$,
and $U$ (a $3 \times 3$ matrix) becomes:

\begin{equation}
    U = (Z^T Z)^{-1} \sigma^2
\end{equation}

However, $\bm{\hat{f}}$ are often correlated, making linear expansion in terms
of them hard.
Still, we can always find a linear transformation $A$ s.t.
$\bm{\tilde{f}} \equiv A \bm{\hat{f}}$ has an identity covariance matrix.

\begin{enumerate}
    \item Find the eigenvectors $\bm{r^i_j}$ satisfying
        $(\bm{r^i})^T \bm{r^j} = \delta_{ij}$,
        and eigenvalues $\lambda_i$ of of $U$.

        Note that $\lambda_i$ are all positive because $U$ is positive-definite.

    \item For $M_{ij} = \bm{r^i_j}$, $M \bm{\hat{f}}$ has a covariance
        matrix element
        $V_{ij} = M_{ik} U_{kl} M^T_{lj} = \bm{r^i_k} U_{kl} \bm{r^j_l}
            = \lambda_j \delta_{ij}$.

    \item Let $C_{ij} = \delta_{ij} \frac{1}{\sqrt{\lambda_j}}$, without summing
        over repeated indices.
        Then $A = C M$.

    \item Note that
        $\mathcal{\hat{F}}(z) = \bm{z}^T \bm{\hat{f}} = \bm{z}^T A^{-1} \bm{\tilde{f}}$.
        Its linear variations in terms of $\bm{\tilde{f}}$ are
        $\bm{z}^T A^{-1} \bm{\delta \tilde{f}}$.
\end{enumerate}

In our \Hammer-based form factor variation, the covariance $U$ are taken from
corresponding papers directly, and the expansion coefficients are defined in
terms of $\bm{\hat{f}}$, which are often correlated.
Still, the variations must be uncorrelated due to \HistFactory\ requirements.

Thus, a typical variation workflow for generating $\pm 1 \sigma$ form factor
variations is:

\begin{enumerate}
    \item Find $A^{-1}$ as defined in the previous paragraph.

    \item Provide form factor variation information to \Hammer.
        We have 2 equivalent procedures:

        \begin{itemize}
            \item We can supply the $A^{-1}$ to \Hammer\ form factor class
                directly.

                For example, for \lstinline{BGLVar} form factor of $B \rightarrow D$
                decays,
                the matrix is named as \lstinline{apa0mat}.

                In this case, the form factor variations are in an
                \emph{orthonormal} error eigenbasis, so the form of variations
                are very simple. For example, a $+1\sigma$ variation in the
                first component can be supplied as\footnote{
                    Here \lstinline{delta_ap0} is a misnomer.
                }:
                \begin{lstlisting}
                    ham.setFFEigenvectors{"BtoD", "BGLVar", {"delta_ap0", 1}};
                \end{lstlisting}

            \item Alternatively, contract $A^{-1} \delta\bm{\hat{f}}$ manually,
                and provide the variation in the original basis the form factors
                are parameterized.

                This works well for \lstinline{BGLVar}, because by default the
                covariance is set to identity.

                In this case, one does not manually input the 2-dimensional
                covariance matrix manually in C++.
                The trade-off is: the variations become more complex.
                For example, the same $+1\sigma$ variation in the
                error eigenbasis now becomes:

                \begin{lstlisting}
                    ham.setFFEigenvectors("BtoD", "BGLVar",
                      {{"delta_ap0", -1.7339e-05}, {"delta_ap1", 0.0001643},
                       {"delta_ap2", 0.000192972}, {"delta_a01", -0.100070},
                       {"delta_a02", 0.002478870}, {"delta_a00", 0.0064648}}
                    );
                \end{lstlisting}

                with precisions reduced for simplicity\footnote{
                    Now \lstinline{delta_ap0} etc. are consistent with the common
                    form factor parameterization in the literature.
                    Also note that the \lstinline{delta_a00} is fixed by the
                    unitarity requirement.
                }.
        \end{itemize}

        In both cases, the matrix $\bm{z}^T A^{-1}$
        ($\bm{z}^T \text{Id}$ in the second case) is computed only once,
        which can be contracted with any linear variation at a later time.
\end{enumerate}
