\chapter{Formal description of selected methods used in this analysis}

\section{Shift negative efficiencies} \label{appx:formal:shift-neg-eff}

Sometimes efficiencies obtained from \sPlot is negative, which is unphysical.
Here we present a procedure to shift negative efficiencies back to positive,
using the mean $\mu$ and standard deviation $\sigma$ from \sPlot.

For a given efficiency, assuming Gaussian distributed,
we can compute the total probability between $[0, 1]$,
the region where the efficiency have a physical meaning,
then pick $\mu_\text{shifted}$ such that
$P([0, \mu_\text{shifted}]) = P([\mu_\text{shifted}, 1])$ as the new efficiency.

For a Gaussian distribution with mean $\mu$ and standard deviation $\sigma$, its
cumulative distribution function (c.d.f) is given by:

% formula from PDG 2020, 39.4.3, p. 236
\begin{equation}
    P([-\infty, x]) =
        \frac{1}{2} \left( 1 + \text{erf}\left( \frac{x - \mu}{\sigma \sqrt{2}} \right) \right)
\end{equation}
where \erf\ is the error function.

Therefore, the total probability between $[0, 1]$ is:

\begin{align}
    P([0, 1]) &=
        \frac{1}{2} \left( 1 + \erf\left( \frac{1 - \mu}{\sigma \sqrt{2}} \right) \right) -
        \frac{1}{2} \left( 1 + \erf\left( \frac{0 - \mu}{\sigma \sqrt{2}} \right) \right) \\
              & = \frac{1}{2} \left(
                \erf\left( \frac{1 - \mu}{\sigma \sqrt{2}} \right) -
                \erf\left( \frac{0 - \mu}{\sigma \sqrt{2}} \right)
              \right)
\end{align}

And the c.d.f evaluated at $\mu_\text{shifted}$ is:
\begin{align}
    P([-\infty, \mu_\text{shifted}])
        &= P([-\infty, 0]) + \frac{1}{2} P([0, 1]) \\
        &= \frac{1}{2} \left[
            1 + \frac{1}{2} \left(
                \erf\left( \frac{0 - \mu}{\sigma \sqrt{2}} \right) +
                \erf\left( \frac{1 - \mu}{\sigma \sqrt{2}} \right)
            \right)
        \right]
\end{align}

Define $k \equiv \frac{1}{2} \left(
    \erf\left( \frac{0 - \mu}{\sigma \sqrt{2}} \right) +
    \erf\left( \frac{1 - \mu}{\sigma \sqrt{2}} \right)
\right)$,
using \erfinv, we find the expression for $\mu_\text{shifted}$:

\begin{equation}
    \mu_\text{shifted} = \sigma \sqrt{2} \erfinv{k} + \mu
\end{equation}


\section{Apply fit yields as a weight}

Often times, it is desirable to look at distributions of variables other than
fit variables.
One can construct such distributions with nominal fit template building
procedure by replacing fit variables with variables of interest, but this
approach is not very flexible:
For example, if a different binning scheme is required, the distributions need
to be rebuilt from scratch.

It is possible to construct an event-by-event weight $u$ such that building
distributions of any variable (of arbitrary binning scheme, possibly unbinned)
with $u$ gives correct fitted yield and shape.
In the following paragraphs, a method of constructing $u$ is provided.

The weight $u$ is constructed from fit templates and fitted yields.
A typical fit template consists of a nominal template and several variational
ones. Denote:

\begin{itemize}
    \item A fit template can be represented by its yield $N$.\footnote{
            When a binned template has only 1 bin, it is exactly represented by
            $N$.
        }
    \item A nominal fit template with yield\footnote{
        By \emph{yield}, we mean \texttt{histogram.Interal()}.}
        $N$ is constructed with weight $w$ and
        scaling factor\footnote{
            The scaling factor $s$ is unabsorbed from $w$ for convenience.
        } $s \equiv 1$.
        Denote as: $(N; w, 1)$.
    \item Each variation is represented by a parameter $\alpha_i$, with
        two templates specifying variations at $\alpha_i = \pm 1$.

        Denote $\pm$ variation templates as
        $(N_{\alpha_i}^\pm; w_{\alpha_i}^\pm, s_{\alpha_i}^\pm)$.

        When $\alpha_i$ is not at $\pm 1$, an interpolation function $f$ is used
        to approximate template at this given $\alpha_i$.
        Denote the (yield of the) interpolated template as
        $f(\alpha_i; N, N_{\alpha_i}^+, N_{\alpha_i}^-)$.
\end{itemize}

Thus, the final variated template of yield $N_f$ with $m$ variations can be
written as:

\begin{equation}
    N_f = N + \sum_i^m f(\alpha_i; N, N_{\alpha_i}^+, N_{\alpha_i}^-)
    \label{eqn:formal:final-template-yield}
\end{equation}

Suppose the binning is fine enough such that \emph{each event is in its own bin},
then for each bin $N = w \cdot s = w$.
Thus \cref{eqn:formal:final-template-yield} can be rewritten as:

\begin{equation}
    w_f = w + \sum_i^m
    f(\alpha_i;
      w,
      w_{\alpha_i}^+ \cdot s_{\alpha_i}^+,
      w_{\alpha_i}^- \cdot s_{\alpha_i}^-
     )
\end{equation}

Noting $N_f$ and fitted yield $N_\text{fitted}$ are related by a constant
scaling factor $k \equiv \frac{N_\text{fitted}}{N_f}$, the expression for $u$ is
obtained:

\begin{equation}
    u = k \cdot \left( w + \sum_i^m
    f(\alpha_i;
      w,
      w_{\alpha_i}^+ \cdot s_{\alpha_i}^+,
      w_{\alpha_i}^- \cdot s_{\alpha_i}^-
     )\right)
\end{equation}

\paragraph{Remark} Often (but not always!) variation templates are required to
have the same yield as the nominal template (this is to represent a
\emph{shape-only} variation), thus a scaling factor
$s_{\alpha_i}^\pm \equiv \frac{N}{N_{\alpha_i,\text{raw}}^\pm}$
is applied on top of $w_{\alpha_i}^\pm$.
