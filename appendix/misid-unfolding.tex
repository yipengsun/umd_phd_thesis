\chapter{Unfolding procedure of fake muon control sample}
\label{appx:unfold-tech}

The notation that we use in the main text and throughout this appendix is as follows:

\begin{itemize}
\item Hats as that in $\hat{t}$ refer to species tagged as a track of species $t$ based
  on the cuts listed in \cref{tab:selection-for-tagged-species}.
\item Apostrophes as that in $t'$ are used simply to indicate that the track species may or may not
  be the same that in $t$.
\item Subscripts such as $t_\text{fake}$ indicate which selection the event satisfies.
\item Yields denoted with the letter $n$ refer to the number of events in the \pidcalib (or MC) samples, while
  yields with tilde ($\tilde{n}$) refer the number of events in our fake \muon samples.
\end{itemize}

The various samples considered in this procedure are described below and shown
schematically in \cref{fig:relation-unfolding-sets}:

\begin{itemize}
\item $t_\text{PIDCalib}$: The \pidcalib sample (or MC in the case of ghost) with minimal cuts
  that is considered to be pure in the track species $t$.
\item $t_\text{acc}$: The subset of $t_\text{PIDCalib}$ that satisfies all \muon
cuts other than PID (i.e., track in \muon acceptance).
\item $t_\text{fake}$: The subset of $t_{acc}$ that \emph{fails} \muon PID and thus contributes exclusively to the fake \muon sample.
\item $\hat{\mu}_\text{sig}$: The subset of $t_{acc}$ that \emph{passes} \muon PID
  and thus contributes to the real \muon samples.
\item $\hat{t}'_\text{fake}$: The subset of $t_\text{fake}$ that passes the $\hat{t}'$ tagging requirements
  from \cref{tab:selection-for-tagged-species}.
\end{itemize}

\begin{figure}[ht]
    \centering
    \resizebox{0.9\columnwidth}{!}{
        \begin{tikzpicture}
    \pgfdeclarelayer{nodelayer}
    \pgfdeclarelayer{edgelayer}
    \pgfsetlayers{nodelayer,edgelayer}

	\begin{pgfonlayer}{nodelayer}
		\node (0) at (-7.75, -4) {};
		\node (1) at (8.25, -4) {};
		\node (2) at (-7.75, 4) {};
		\node (3) at (8.25, 4) {};
		\node (6) at (-6.5, -3.5) {$t_\text{PIDCalib}$};
		\node (7) at (-6, 3) {};
		\node (8) at (-6, -3) {};
		\node (9) at (6, -3) {};
		\node (10) at (6, 3) {};
		\node (11) at (-5.3, -2.6) {$t_\text{acc}$};
		\node (14) at (0.75, -2) {};
		\node (15) at (0.75, 2) {};
		\node (17) at (-1, -1.25) {$t_\text{fake}$};
		\node (18) at (1.75, -1.25) {$\hat{\mu}_\text{sig}$};
		\node (19) at (-3, 1) {};
		\node (20) at (-3, -0.5) {};
		\node (21) at (-1, -0.5) {};
		\node (22) at (-1, 1) {};
		\node (23) at (-2, 0.25) {};
		\node (24) at (-2, 0.25) {$\hat{t}'_\text{fake}$};
	\end{pgfonlayer}
	\begin{pgfonlayer}{edgelayer}
		\draw (1.center)
			 to (0.center)
			 to (2.center)
			 to (3.center)
			 to cycle;
		\draw (10.center)
			 to (9.center)
			 to (8.center)
			 to (7.center)
			 to cycle;
		\draw (15.center)
			 to [in=180, out=180, looseness=4.50] (14.center)
			 to [in=0, out=0, looseness=2.50] cycle
			 to (14.center);
		\draw (19.center)
			 to (22.center)
			 to (21.center)
			 to (20.center)
			 to cycle;
	\end{pgfonlayer}
\end{tikzpicture}

    }
    \caption{Relations between $t$-specie-enriched samples used in unfolding.}
    \label{fig:relation-unfolding-sets}
\end{figure}


We now describe step 2 and 3 in the main text with more detail.

\paragraph{Step 2} The probability of a true $t$ to be classified as $\hat{t}'$ can
be computed as:

\begin{equation}
    \misEff[t_\text{fake}]{\hat{t}'_\text{fake}} = \frac{n_{\hat{t}'_\text{fake}}}{n_{t_\text{fake}}}
\end{equation}

where $n$ denote yields in the \emph{fake} samples as shown
in \cref{fig:relation-unfolding-sets}.
Note that all efficiencies are binned in $p_B$, $\eta_B$, and nTracks with a
consistent binning scheme.


\paragraph{Step 3} In a particular bin, the measured
yields $\tilde{n}_{\hat{t}'}$, the true yields $\tilde{n}_{t}$, and the response matrix $M$
have the following relation:

\begin{equation}
    \begin{pmatrix*}[l]
        \tilde{n}_{\hat{\pi}} \\
        \tilde{n}_{\hat{K}}   \\
        \tilde{n}_{\hat{p}}   \\
        \tilde{n}_{\hat{e}}   \\
        \tilde{n}_{\hat{g}}   \\
    \end{pmatrix*}
    =
    \begin{pmatrix*}[l]
        \misEff[\pi]{\hat{\pi}} & \misEff[K]{\hat{\pi}} & \misEff[p]{\hat{\pi}} & \misEff[e]{\hat{\pi}} & \misEff[g]{\hat{\pi}} \\
        \misEff[\pi]{\hat{K}}   & \misEff[K]{\hat{K}}   & \misEff[p]{\hat{K}}   & \misEff[e]{\hat{K}}   & \misEff[g]{\hat{K}}   \\
        \misEff[\pi]{\hat{p}}   & \misEff[K]{\hat{p}}   & \misEff[p]{\hat{p}}   & \misEff[e]{\hat{p}}   & \misEff[g]{\hat{p}}   \\
        \misEff[\pi]{\hat{e}}   & \misEff[K]{\hat{e}}   & \misEff[p]{\hat{e}}   & \misEff[e]{\hat{e}}   & \misEff[g]{\hat{e}}   \\
        \misEff[\pi]{\hat{g}}   & \misEff[K]{\hat{g}}   & \misEff[p]{\hat{g}}   & \misEff[e]{\hat{g}}   & \misEff[g]{\hat{g}}   \\
    \end{pmatrix*}
    \begin{pmatrix*}[l]
        \tilde{n}_{{\pi}} \\
        \tilde{n}_{{K}}   \\
        \tilde{n}_{{p}}   \\
        \tilde{n}_{{e}}   \\
        \tilde{n}_{{g}}   \\
    \end{pmatrix*}
\end{equation}
where $\tilde{n}$ denote yields in the fake \muon sample,
and the $t/\hat{t}'$ subscripts in $\tilde{n}$ only refer to
the true/tagged species, without any relation on \pidcalib samples.
The \emph{fake} subscripts in the efficiencies are dropped for readability.

The Bayesian (iterative) unfolding method provided by \RooUnfold\ is used to
obtain the true yields.
Naive matrix inversion cannot be used because it is sensitive to statistical
fluctuations.  % TODO: Should I cite Cowan here?

To find $\misEff[\hat{t}_\text{fake}']{t_\text{fake}}$, recall its definition:

\begin{equation}
    \misEff[\hat{t}_\text{fake}']{t_\text{fake}} \equiv
        \frac{
            \text{Number of $\hat{t}'$ from $t$}
        }{
            \text{Total number of $\hat{t}'$ based on unfolding}
        }
\end{equation}

With the unfolded true yields, we can compute:

\begin{itemize}
    \item Number of $\hat{t}'$ from $t$:
        $\tilde{n}_t \cdot \misEff[t_\text{fake}]{\hat{t}_\text{fake}'}$
    \item Total number of $\hat{t}'$ based on unfolding:
        $\sum_t \tilde{n}_t \cdot \misEff[t_\text{fake}]{\hat{t}_\text{fake}'}$
\end{itemize}

Therefore:

\begin{equation}
    \misEff[\hat{t}_\text{fake}']{t_\text{fake}} =
        \frac{
            \tilde{n}_t \cdot \misEff[t_\text{fake}]{\hat{t}_\text{fake}'}
        }{
            \sum_t \tilde{n}_t \cdot \misEff[t_\text{fake}]{\hat{t}_\text{fake}'}
        }
\end{equation}
