%Abstract Page

\begin{center}
\large{{ABSTRACT}}
\vspace{3em}
\end{center}
%%%%
\hspace{-.15in}
%%%%
\begin{tabular}{ll}
Title of Dissertation:    & {\large MEASUREMENT OF \RDX IN SEMILEPTONIC} \\
                          & {\large \B DECAYS AND UPGRADE OF} \\
                          & {\large THE LHCB UPSTREAM TRACKER} \\
\\
                          & {\large Yipeng Sun} \\
                          & {\large Doctor of Philosophy, 2023} \\
\\
Dissertation Directed by: & {\large  Professor Manuel Franco Sevilla} \\
                          & {\large  Department of Physics} \\
\end{tabular}

\vspace{3em}
\doublespacing \normalsize

The LHCb experiment at the Large Hadron Collider provides an unique opportunity
to study flavor physics, for example lepton flavor universality (LFU),
with high luminosity.
%%%%
One topic of the thesis is the upgrade of the LHCb upstream tracker which
greatly increases the readout bandwidth of the tracker,
paving the way for future precision measurements with even higher luminosity.
%%%%
Another topic of the thesis is a preliminary measurement of the
\RDX, define as the ratio
of branching fractions $\BFDTau / \BFDMu$, with LHCb 2016 data.
\RDX is a proxy to test LFU,
which is a property of the standard model (SM) requiring
the three generations of leptons ($e, \mu, \tau$) couple to gauge bosons of
the eletroweak interactions with the same strength.
It is an important probe for testing the validity of SM and possibly providing
hints to new physics beyond SM.
