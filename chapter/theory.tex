% Potentially useful sources:
%   https://physics.stackexchange.com/questions/707380/is-lepton-flavour-universality-an-accidental-symmetry-of-the-standatd-model
%   https://physics.stackexchange.com/questions/672626/what-is-an-accidental-symmetry
%   https://physics.stackexchange.com/questions/55350/two-ways-to-form-su2-singlets
%   https://physics.stackexchange.com/questions/607444/chiral-symmetry
%   https://physics.stackexchange.com/questions/429180/how-does-bhabha-scattering-imply-the-existence-of-quark-lepton-substructure
%   https://physics.stackexchange.com/questions/96362/what-does-a-rm-su2-isospin doublet-really-mean
%   https://physics.stackexchange.com/questions/385159/quark-gluon-color-relationship-in-pure-qcd
%   https://physics.stackexchange.com/questions/327008/how-do-the-infinite-dimensional-representations-of-the-poincar%C3%A9-group-work
%   Rep. of Lorentz group: http://sharif.edu/~sadooghi/QFT-I-96-97-2/LorentzPoincareMaciejko.pdf
%   https://scholar.harvard.edu/files/noahmiller/files/representation-theory-quantum.pdf
%
% Very useful:
%   https://physics.stackexchange.com/questions/266963/what-are-the-differences-if-any-between-the-dysons-series-definition-and-the
%   https://pages.cs.wisc.edu/~guild/symmetrycompsproject.pdf
%   https://physics.stackexchange.com/questions/578059/why-the-little-group-of-a-massless-spin 1-particle-is-iso2-rather-than-so2
%   https://math.stackexchange.com/questions/1334965/direct-sum-vs-direct-product-vs-tensor-product
%   https://physics.stackexchange.com/questions/619178/standard-model-notation-on-doublets
%   https://physics.stackexchange.com/questions/190416/what-role-does-spontaneous-symmetry-breaking-play-in-the-higgs-mechanism
%   https://physics.stackexchange.com/questions/316094/helicity-of-antiparticles
%
%   From Jack(!): https://physics.stackexchange.com/questions/92157/off-shell-corrections-to-massive-vector-boson-propagator-in-polarization-form
%
% Useful papers not cited:
%   On gauge symmetry: https://arxiv.org/pdf/1901.10420.pdf
%   2-component spinor notation: https://arxiv.org/pdf/0812.1594.pdf
%   V-A current: https://warwick.ac.uk/fac/sci/physics/staff/academic/boyd/stuff/neutrinolectures/weak.pdf
%   Review on V-A theory and chirality: https://www.hep.phy.cam.ac.uk/~thomson/lectures/partIIIparticles/Handout9_2009.pdf
%   Weak current: https://www.sciencedirect.com/science/article/pii/0550321389903064
%   Helicity amplitude: http://www-library.desy.de/preparch/desy/proc/proc13-03/Koerner.pdf


\chapter{Theory of semileptonic $B$ decays}
\label{ref:theory}

\section{Construction of the electroweak theory in the SM}

Through a close collaboration between experimentalists and theorists,
particle physicists developed and validated a \emph{quantum field theory}
to describe \emph{interactions} of \emph{elementary particles},
known as the standard model of particle physics (SM),
the most precise theory of the sub-atomic world to date.
A construction of the SM is outlined\footnote{
    ``I am an experimentalist!'' says the author waving his hands.
} first with the help of
\cite{Robinson_2011,Schwichtenberg_2018},
followed by brief discussions on quantum eletrodynamics (QED),
the electroweak theory, and the lepton flavor universality in the SM.


\subsection{Outline of the construction}

It is experimentally established that the speed of light in vacuum is constant
in any reference frame.
Then, all transformations preserving the speed of light\footnote{
    Each of such transformations, denote as $L$, must leave the Minkowski metric
    $\eta$ intact: $L^T \eta L = \eta$.
} can be identified:
rotations, Lorentz boosts, parity ($\vec{x} \rightarrow -\vec{x}$),
time reversal ($t \rightarrow -t$), and spacetime translations.
All but the translation transformations form a group called the
``Lorentz group''.
Focusing on the restricted Lorentz group $SO^+(1, 3)$,
the group that preserves the direction of space and time\footnote{
    Loosely speaking, Lorentz group contains 4 ``classes'': $\{I, T, P, TP\}$.
    $SO^+(1,3)$ is the one that contains the identity element.
    The other 3 can be obtained by applying parity ($P$), time reversal ($T$),
    or both ($TP$) to $SO^+(1,3)$.
}:
%%%%
it has 6 generators $J_i, K_i$ with $i \in \{1,2,3\}$,
which obey the following commutation relations:
\begin{align}
    [J_i, J_j] & = i \epsilon_{ijk} J_k \\
    [J_i, K_j] & = i \epsilon_{ijk} K_k \\
    [K_i, K_j] & = -i \epsilon_{ijk} J_k
\end{align}
such that each Lorentz transformation $L$ of boost $\vec{\phi}$ and rotation
$\vec{\theta}$ can be generated with the exponential map:
$L = e^{i \vec{J} \cdot \vec{\theta} + i \vec{K} \cdot \vec{\phi}}$.
% SO+(1,3) = SU(2) x SU(2), direct product
%%%%
A new set of generators can be defined
$N^\pm = \frac{1}{2}(J_i \pm i K_i)$ such that each set of $\{N^\pm_i\}$
are generators of the $SU(2)$ group, and $N^+_i, N^-_j$ commute.
Hence $SO^+(1, 3)$ contains two copies of $SU(2)$.

As $L \in SO^+(1, 3)$ is a Lorentz transformation,
it can \emph{act} on \emph{some objects}.
If we map each $L$ to a $n \times n$ invertible matrix $M$ of complex numbers,
the objects $L$ can act on are represented by $n$-dimensional complex vectors.
A map from $SO^+(1, 3)$ to the (vector) space of $M$ is called a
``representation'' of $SO^+(1, 3)$;
there are many representations for a single group.
As an example, each generator of $SU(2)$,
denote as $\{N_1, N_2, N_3\}$,
can be mapped to a Pauli matrix $N_i \mapsto \sigma_i$,
forming a 2-dimensional representation of $SU(2)$.
More generally,
for a $n$-dimensional representation of $SU(2)$,
let $N_3$ be represented by a diagonal matrix and its largest eigenvalue $j$.
Due to commutation relation $[N_i, N_j] = i \epsilon_{ijk} N_k$,
only $N_3$ can be diagonal.
It can be shown that the spectrum of the $N_3$ eigenvalues is discrete:
$\{ -j, -j+1, \dots, j-1, j\}$ with $2j + 1$ elements.
Since $N_3$ is a $n$-dimensional invertible matrix, $2j + 1 = n$.
Hence, such a representation can be labelled with $j = \frac{n - 1}{2}$
which is a half-integer.

Since $SO^+(1, 3)$ contains two copies of $SU(2)$,
connections between the representations of these two groups can be made
and the representations of $SO^+(1, 3)$ can be labelled as $(j, j')$.
%%%%
That is, Lorentz transformations can act on many different
\emph{classes} of objects,
with each class corresponds to a $(j, j')$ representation.
Some of such representations are \emph{irreducible}\footnote{
    Roughly speaking, there is no transformation that can simultaneously
    block-diagonalize all matrices representing the group elements.
},
which correspond to specific types of \emph{elementary particles}
of spin $j + j'$:
the $(0, 0)$ representation corresponds to spin 0 scalar particles;
the $(\frac{1}{2}, 0)$ and $(0, \frac{1}{2})$ representations both corresponds
to spin $\frac{1}{2}$ particles,
with the former left-handed and the latter right-handed;
the $(\frac{1}{2}, \frac{1}{2})$
representation corresponds to spin 1 particles.
There are many more irreducible representations,
but in the SM only the spin 0, $\frac{1}{2}$, and 1 particles are present.

Now the SM can be constructed as follows:
first, consider elementary particles with spin 0, $\frac{1}{2}$, and 1,
which are represented by linear operators in a space.
A Lorentz-invariant Lagrangian density\footnote{
    Will be referred to as ``Lagrangian'' in later text for brevity.
} can be built from
linear, quadratic, and the lowest possible derivative terms of these
operators.
A Lagrangian built in this manner contains no interaction (i.e. a free
Lagrangian).
Then, require the Lagrangian to be invariant\footnote{
    Local gauge invariances represent mathematical redundancy in the description
    of the system,
    and must be preserved to have a consistent theory.
} under certain local gauge transformations
such as that from $U(1), SU(2)$, and $SU(3)$.
%%%%
For example, a global gauge transformation
$\psi \rightarrow e^{i\alpha} \psi$ where $\alpha \in \mathbb{R}$
can be promoted into a local one by
changing $\alpha$ from a number to a function of the spacetime coordinate
$\alpha(x^\mu)$.
All such local gauge transformations form a $U(1)$ group.
%%%%
These requirements can be satisfied by adding/removing certain terms to the
Lagrangian;
these terms make interactions between elementary particles happen.
At this point, there is nothing quantum about the theory.
The theory is ``quantized'' by forcing canonical commutation/anti-commutation
relation for integer spin and half-integer spin particles,
respectively\footnote{
    Note that at this point, the terms
    \emph{elementary particles, interactive, and quantum field theory}
    are defined.
}.
Finally, the scattering amplitude from an initial state to a final state,
which is of utmost importance experimentally,
can be computed in a perturbative
manner\footnote{
    To be more precise, the scattering amplitude $\braket{f}{i}$
    between an initial state $\ket{i}$
    and a final state $\ket{f}$
    can be calculated in the following
    manner \cite{Weigand}:
    \begin{enumerate}[noitemsep,nosep]
        \item The amplitude $\braket{f(y_1, \dots, y_r)}{i(x_1, \dots, x_n)}$
            is related to the interactive vacuum
            $\ket{\Omega}$ by the LSZ reduction formula:
            $\braket{f}{i} \propto \bra{\Omega} T \phi(y_1) \dots \phi(y_r) \phi(x_1) \dots \phi(x_n) \ket{\Omega}$,
            where $\phi(x)$, an operator in the Heisenberg picture,
            creates and destroys particles at spacetime point $x$ in the
            interactive vacuum $\ket{\Omega}$,
            and $T$ stands for time-ordering.

        \item Denote the free and the interaction part of the Hamiltonian
            as $H_0, H_\text{int}$, accordingly.

        \item At a given time $t$, a Heisenberg operator and a interactive
            operator are related by
            $\phi(t_0, \vec{x}) = U^\dagger (t, t_0) \Phi_I(t, \vec{x}) U(t, t_0)$,
            where $U(t, t_0) = T e^{-i \int_{t_0}^t H_I(t') d t'}$, and
            $H_I(t) = e^{iH_0(t-t_0)} H_\text{int} e^{-iH_0(t-t_0)}$.

        \item The interactive vacuum $\ket{\Omega}$ and the free vacuum
            $\ket{0}$ are related by
            $\ket{\Omega} \propto \lim_{\tau \rightarrow \infty(1 - i\epsilon)} U(t_0, -\tau)\ket{0}$.

        \item The scattering amplitude can now be written as
            $\braket{f}{i} \propto \bra{0} T \prod_i \Phi_I(x_i) e^{-i \int_{-\infty}^{\infty} H_I(t) dt} \ket{0}$,
            where $\Phi_I$ is the $\phi$ operator in the interaction picture
            and has a \emph{free} mode expansion that can create and destroy
            particles in the \emph{free} vacuum $\ket{0}$.

        \item At this point,
            the exponential involving $H_I$ are expanded in a Taylor series:
            $e^{-i \int_{-\infty}^{\infty} H_I(t) dt} = \sum_n \frac{(-i)^n}{n!} T\left\{
                \int_{-\infty}^{\infty} \cdots \int_{-\infty}^{\infty}
                H_{I}^1 \cdots H_{I}^n d t_1 \cdots d t_n
            \right\}$.
            Each term in the series are turned into a series of
            normal-ordered operators with contractions via the Wick's theorem.

        \item The contractions are Feynman propagators with the
            corresponding field operators which are specified by the commutation
            relations between the operators.

        \item The first (non-trivial non-vanishing) order terms are identified
            as tree-level Feynman diagrams.
    \end{enumerate}
    Higher order terms may involve infinities and require renormalization.
}.


\subsection{Quantum eletrodynamics (QED)}
\label{qed}

Following the recipe provided above, first identify all\footnote{
    This refers to linear, quadratic, and lowest possible derivative terms.
} possible Lorentz scalar
terms in a free Lagrangian for spin $\frac{1}{2}$ particles:
the second order terms
$S_1 = (\chi_L)^\dagger \xi_R, S_2 = (\xi_R)^\dagger\chi_L$,
and the first order partial derivative terms (identified as the kinetic terms)
$S_3 = (\chi_L)^\dagger \partial_\mu \bar{\sigma}^\mu \chi_L,
S_4 = (\xi_R)^\dagger \partial_\mu {\sigma}^\mu \xi_R$,
where each $\chi_L, \xi_R$ is a 2-component Weyl spinor.
This shows that both left- and right-handed particles are needed for Lorentz
invariance,
which motivates a 4-component Dirac spinor
$\psi = \bigl(\begin{smallmatrix} \chi_L \\ \xi_R \end{smallmatrix}\bigr)$.
%%%%
To combine $\psi$ into the form of $S_1, S_2$,
the matrix
$\gamma_0 = \bigl(\begin{smallmatrix} 0 & I_{2 \times 2} \\ I_{2 \times 2} & 0 \end{smallmatrix}\bigr)$
can be used such that $\psi^\dagger \gamma_0 \psi$ is a Lorentz scalar.
%%%%
On the other hand,
the forms of $S_3, S_4$ suggest the use of the matrices of the form
$\bigl(\begin{smallmatrix} \bar{\sigma}^\mu & 0 \\ 0 & \sigma^\mu \end{smallmatrix}\bigr)$,
which are generated by $\gamma_0 \gamma^\mu$,
where $\gamma^\mu = \bigl(\begin{smallmatrix} 0 & {\sigma}_\mu \\ \bar{\sigma}_\mu & 0 \end{smallmatrix}\bigr)$,
noting that $\gamma^0$ can be defined in the same way since
$\sigma^0 = \bar{\sigma}^0 = I_{2 \times 2}$.
Therefore, a Lagrangian
$\mathcal{L} = a \psi^\dagger \gamma_0 \psi + b \psi^\dagger \gamma_0 \gamma^\mu \partial_\mu \psi$
can be written, where $a, b \in \mathbb{C}$.
Defining a shortcut $\bar{\psi} \equiv \psi^\dagger \gamma_0$,
and requiring that the Lagrangian yields the familiar Dirac equation
$(i \gamma_\mu \partial^\mu - m)\psi = 0$ to set $a = -m$ and $ b = i$,
we arrive at the Dirac Lagrangian which describes free spin $\frac{1}{2}$
particles of mass $m$:

\begin{equation}
    \mathcal{L}_\text{Dirac} = \bar{\psi} (i\gamma^\mu\partial_\mu - m) \psi
\end{equation}

The Lagrangian is invariant under a global gauge transformation
$\psi \rightarrow e^{i\alpha} \psi$ where $\alpha \in \mathbb{R}$,
but \emph{not} a local $U(1)$ one where $\alpha$ is replaced
with $\alpha(x^\mu)$,
due to the derivative term $\partial_\mu$.
To enforce such invariance,
A spin 1 field operator $A_\mu$,
which satisfies the gauge transformation
$A_\mu \rightarrow A_\mu + \frac{1}{g} \partial_\mu \alpha(x^\mu)$,
is introduced by replacing the standard derivative $\partial_\mu$ with the
covariant derivative, defined as:

\begin{equation}
    \fsl{D}_\mu \equiv \partial_\mu - i g A_\mu
\end{equation}
at this point, the Lagrangian reads
$\mathcal{L} = \mathcal{L}_\text{Dirac} + g A_\mu\bar{\psi}\gamma^\mu\psi$.
However,
the equation of motion of $A_\mu$
suggests the interaction term should vanish: $g A_\mu\bar{\psi}\gamma^\mu\psi = 0$,
because $0 = \partial_{A_\mu} \mathcal{L} = g \bar{\psi}\gamma^\mu\psi$.
To retain the interaction term,
an additional kinetic term for spin 1 particles is needed,
which is found by doing an exercise of finding all Lorentz-invariant
derivative terms, as in the spin $\frac{1}{2}$ case.
Such term reads $\frac{1}{4}F_{\mu\nu}F^{\mu\nu}$,
where $F^{\mu\nu} = \partial^\mu A^\nu - \partial^\nu A^\mu$.
Putting all pieces together, we obtain the Lagrangian for QED:

\begin{equation}
    \mathcal{L}_\text{QED} = \bar{\psi} (i\gamma^\mu\fsl{D}_\mu - m) \psi - \frac{1}{4}F_{\mu\nu}F^{\mu\nu}
\end{equation}


\subsection{The electroweak theory}
\label{ew-th}

As demonstrated in \cref{qed}, the local gauge invariance principle provides a
method of introducing interactions to free theories.
Motivated by the experimental observation that in beta decay, electrons and
electron neutrinos are produced in pairs,
we can add two copies of spin $\frac{1}{2}$ particles,
labelled as $\psi_e, \psi_{\neu_e}$,
to the Lagrangian\footnote{
    The Lagrangian reads
    $\mathcal{L} = \bar{\Psi} (i \gamma^\mu \fsl{D}_\mu I_{2 \times 2} - M)\Psi -
    \frac{1}{4} F_{\mu\nu}F^{\mu\nu}$,
    where $M = \bigl(\begin{smallmatrix} m_e & 0 \\ 0 & m_{\neu_e} \end{smallmatrix}\bigr)$.
    Note that the identity matrix is often omitted.
} and write both in a doublet
of the form
$\Psi = \bigl(\begin{smallmatrix} \psi_e \\ \psi_{\neu_e} \end{smallmatrix}\bigr)$,
such that one particle can be rotated into the other via a $SU(2)$ rotation
$e^{i \alpha_i \frac{\sigma^i}{2}}$.

It is then natural to postulate a theory of $SU(2) \times U(1)$ local gauge
symmetry.
%%%%
Following a similar route in QED,
noting that $SU(2)$ has 3 non-commutative generators and $U(1)$ has 1 generator,
three spin 1 particles $W^1, W^2, W^3$ are added to the theory for $SU(2)$ in
addition to one spin 1 particle $B$ for $U(1)$.
%%%%
After taking the fact non-commutative nature of the $SU(2)$ generators into
account,
a Lagrangian with correct kinetic terms can be constructed,
in which the covariant derivative is defined as:

\begin{equation}
    \fsl{D}_\mu = \partial_\mu - i[g_2 W_\mu^a T^a_{SU(2)} + g_1 B_\mu Y_{U(1)}]
\end{equation}
where $T^a_{SU(2)} = \frac{1}{2} \sigma^a$ are the generators of $SU(2)$,
and
$Y_{U(1)} = C \bigl(\begin{smallmatrix} 1 & 0 \\ 0 & 1 \end{smallmatrix}\bigr)$.
%%%%
However, naive mass terms $\bar{\Psi} M \Psi$ are forbidden because they
spoil $SU(2)$ gauge symmetry:
an $SU(2)$ transformation $e^{i \alpha_i \frac{\sigma^i}{2}}$ transforms
such terms into
$\bar{\Psi} e^{-i \alpha_i \frac{\sigma^i}{2}} M e^{i \alpha_i \frac{\sigma^i}{2}} \Psi$
where $e^{-i \alpha_i \frac{\sigma^i}{2}} M e^{i \alpha_i \frac{\sigma^i}{2}} \neq M$
because $[\sigma^i, M] \neq 0$ in general as $M \neq m I_{2 \times 2}$.
%%%%
So the theory constructed above requires \emph{all} particles to be massless,
which is inconsistent with the experimental results of massive force carrying
bosons in the weak interaction\footnote{
    If $W^1, W^2, W^3$ are all massless, the weak interaction would behave like
    electromagnetism, which it does not.
}, nor a massive electron.

To introduce \emph{gauge invariant} mass-like terms,
a complex scalar field is introduced with a potential
$V(\phi^\dagger, \phi) = \frac{1}{4}\lambda \left(\phi^\dagger \phi - \frac{1}{2}v^2\right)^2$,
which leads to a non-zero vacuum expectation value (VEV):
$\bra{0}\phi\ket{0} = v$ for $\phi$ degenerate on a complex circle.
Write the complex scalar fields as a $SU(2)$ doublet
$\Phi = \bigl(\begin{smallmatrix} \phi^+ \\ \phi^0 \end{smallmatrix}\bigr)$,
then pick a particular vacuum and expand around it such that
$\Phi = \frac{1}{\sqrt{2}} \bigl(\begin{smallmatrix} v + h \\ 0 \end{smallmatrix}\bigr)$
where $h(x^\mu)$ is a real scalar field known as the Higgs field.
%%%%
Doing so breaks the degeneracy (symmetry) of the vacuum,
an effect called spontaneous symmetry breaking.
%%%%
The couplings between $\Phi$ and the spin 1 gauge fields are embedded in the
kinetic term $\fsl{D}_\mu \bar\Phi \fsl{D}^\mu \Phi$,
and a non-zero VEV leads to terms of the form
$\frac{1}{2}\kappa W_\mu W^{\mu}$
where $W$ is a linear combination of $W^1, W^2$,
and $\kappa$ is a coupling constant which can be interpreted as the mass
of the $W$.
Indeed,
after a field redefinition, we find three massive spin 1 fields and a massless
one:
\begin{align}
    W^+_\mu &= \frac{1}{\sqrt{2}}(W^1_\mu - i W^2_\mu) \\
    W^-_\mu &= \frac{1}{\sqrt{2}}(W^1_\mu + i W^2_\mu) \\
    Z_\mu &= \cos\theta_w W^3_\mu - \sin\theta_w B_\mu \\
    A_\mu &= \sin\theta_w W^3_\mu + \cos\theta_w B_\mu
\end{align}
where $\theta_w = \tan^{-1}\left(\frac{g_1}{g_2}\right)$,
and $m_{W^\pm} = \frac{g_2 v}{2} = m_W$, $m_Z = \frac{m_W}{\cos\theta_w}$,
$m_A = 0$.
The procedure above is referred to as the Higgs mechanism.

Still, the spin $\frac{1}{2}$ particles are massless.
Furthermore, the weak interaction is experimentally shown to maximally violate
parity,
that is, only the left-handed particles interact weakly.
Formally speaking, only the left-handed particles form $SU(2)$ doublets
which transform as
$\Psi_L \rightarrow e^{i \alpha_i \frac{\sigma^i}{2}} \Psi_L$;
the right-handed particles are $SU(2)$ singlets which do not transform
$\psi_R \rightarrow \psi_R$.
As a side note, previously we stated that $SU(2)$ gauge symmetry forbids
terms like $\bar{\Psi} M \Psi$ because $M \neq m I_{2 \times 2}$ in
general,
(e.g. electron and electron neutrinos have \emph{very} different masses).
Parity violation provides a different view point on the same issue:
naive mass terms contain \emph{only} combinations of left-handed doublets
and right-handed singlets, such as $\bar{\Psi}_L \psi_R$.
Inspecting the $SU(2)$ transformation property of such terms reveals that
they are not $SU(2)$ invariant thus forbidden:
$\bar{\Psi}_L \psi_R \rightarrow \bar{\Psi}_L e^{-i \alpha_i \frac{\sigma^i}{2}} \psi_R \neq \bar{\Psi}_L \psi_R$.

To introduce mass to spin $\frac{1}{2}$ particles, Yukawa terms
$\lambda_\text{Yuk} \bar{\Psi}_L \Phi \psi_R$,
which can be readily checked\footnote{
    $U(1)$:
    $\bar{\Psi}_L \Phi \psi_R \rightarrow
    (\bar{\Psi}_L e^{-i \alpha}) \Phi (e^{i \alpha} \psi_R)
    = \bar{\Psi}_L \Phi \psi_R$;
    $SU(2)$:
    $\bar{\Psi}_L \Phi \psi_R \rightarrow
    (\bar{\Psi}_L e^{-i \alpha_i \frac{\sigma^i}{2}}) (e^{i \alpha_i \frac{\sigma^i}{2}} \Phi) \psi_R
    = \bar{\Psi}_L \Phi \psi_R$.
} to be Lorentz, $U(1)$, and $SU(2)$ invariant, are added to the Lagrangian.
Through the familiar Higgs mechanism, which has a non-zero VEV for $\Phi$,
fermion mass terms such as the electron mass term
$\frac{\lambda_e v}{\sqrt{2}}({\psi^\dagger_{L,e} \psi_{R,e} + \psi^\dagger_{R,e}\psi_{L,e}})$
are generated.

The full SM,
which includes the strong interactions on top of the electroweak interactions,
can be constructed in the same way outlined above with an additional $SU(3)$
gauge group such that the full symmetry group is
$SU(3) \times SU(2)_L \times U(1)$.
Its construction will not be discussed in this thesis.


\subsection{Lepton flavor universality in the SM}
\label{ref:theory:lfu}

With the electroweak theory developed in \cref{ew-th},
and accepting the experimental observation that leptons do not interact strongly
(i.e. they are $SU(3)$ singlets),
it is now possible to discuss lepton flavor universality (LFU):
the SM permits the addition of arbitrary generations of spin $\frac{1}{2}$
$SU(2)$ lepton doublets and singlets,
and each generation couples to all electroweak gauge fields,
namely $\W^\pm_\mu, Z_\mu$, and $A_\mu$,
with the same strengths through the same covariant derivative term
$\fsl{D}_\mu$.
The LFU is manifested by the very construction of the SM.
The SM does contain Yukawa terms in which the different flavors of leptons
couple to the Higgs doublet $\Phi$ with different strengths,
leading to different masses of the leptons.

The three flavor generations of leptons are purely determined by experiments,
as the SM places no constraint on the number of flavor generations.


\section{Semileptonic $B \rightarrow D \ellm\neulb$ decays in the SM}

Semileptonic $B \rightarrow D \ellm\neulb$ decays,
where $B$ stands for a \Bzb or \Bm meson,
$D$ for a generic charm meson, and \ellm for a $e^-$, \mun or \taum,
play an central role in this analysis as the signal, normalization,
and most of the background decays fall under this category.
A theoretical model describing these processes enables the calculation of the
decay matrix elements.
This, in turn, facilitates generation of the Monte-Carlo simulation samples that
are critical in several stages of the analysis.
This section describes the formalism of such a modeling.

The parton level process involved in the $B \rightarrow D \ellm\neulb$ decays is
$b \rightarrow c \ellm\neulb$,
as already shown in \cref{fig:decay-diagrams},
which is mediated by the weak charged current in the SM.
% NOTE: cite Schwartz QFT textbook maybe
Its matrix element is given by \cite{Tanaka_1995}:
\begin{equation}
    \mathcal{M}^{\lambda_l}_{\lambda_D}(q^2, \theta_l) =
    \frac{G_F}{\sqrt{2}} \frac{m^2_W}{m^2_W - q^2}
    \sum_{\lambda_W} \eta_{\lambda_W}
    L^{\lambda_l}_{\lambda_W}(q^2, \theta_l)
    H^{\lambda_D}_{\lambda_W}(q^2)
    \label{eqn:b-d-master-formula}
\end{equation}
in which the total amplitude is obtained from summing over the helicity states
($\pm,0,s$)
of the off-shell virtual $W$,
polarized in the $B$ rest frame as:
\begin{align}
    & \epsilon_\mu(\pm) = \frac{1}{\sqrt{2}}(0, \pm1, -i, 0)
    & \epsilon_\mu(0) &= \frac{1}{\sqrt{q^2}}(|\vec{p}^*_D|, 0, 0, -q^0)
    & \epsilon_\mu(s) &= \frac{q^\mu}{\sqrt{q^2}}
\end{align}
where $q^0$ is the energy of the virtual $W$ boson in the $B$ rest frame,
and $q^2 = (q_0, -\vec{p}^*_D)^2$ is the invariant mass of the $W$.
The metric factor $\eta_{\lambda_W}$ is given by:
\begin{equation}
    \eta_{\pm,0,s} = \{1, 1, \frac{q^2 - m_W^2}{m_W^2}\} \approx \{1, 1, -1\}
\end{equation}
where $s$ denotes the scalar state of the virtual $W$.
As shown in \cref{eqn:b-d-master-formula},
the leptonic and hadronic currents factorize because leptons
are color singlets which do not interact strongly.
The currents are defined as:
\begin{align}
    L^{\lambda_l}_{\lambda_W}(q^2, \theta_l)
    & =
    \epsilon_\mu(\lambda_W)
    \bra{l(p_l, \lambda_l) \neulb(p_\neulb)} \bar{l} \gamma^\mu (1 - \gamma_5) \neul
    \ket{0} \\
    H^{\lambda_D}_{\lambda_W}(q^2)
    & =
    \epsilon^*_\mu(\lambda_W)
    \bra{D(p_D, \lambda_D)} \bar{c} \gamma^\mu (1 - \gamma_5) b
    \ket{B(p_B)}
\end{align}
where $p_i, \lambda_i$ refer to the four-momenta and helicities of the particle
$i$;
$\theta_l$ refers to the angle between the lepton and the $W$ flight direction
in the $\ellm\neulb$ rest frame,
as shown in \cref{fig:b-d-decay-schematic}.

\begin{figure}[!htb]
    \centering
    \includegraphics[width=0.65\textwidth]{./figs-theory/b_d_angular_vars.pdf}
    \caption{
        Angular variables of the $\Bzb \rightarrow D^{*+} \ellm\neulb$ decays.
        For $\Bm \rightarrow \Dz\ellm\neulb$ decays,
        only the $\theta_l$ variable exists.
    }
    \label{fig:b-d-decay-schematic}
\end{figure}

\paragraph{Leptonic currents}
The leptonic currents are straightforward to evaluate,
as the weak interaction permits a perturbative calculation and the currents can
be sufficiently approximated by tree-level decays.
Evaluated in the $B$ rest frame, each lepton current of the polarization combintation
$(\lambda_W, \lambda_l)$ is a function of the lepton mass,
the invariant mass squared $q^2$ of the virtual $W$,
and $\theta_l$ as defined in \cref{fig:b-d-decay-schematic}.
The results are listed in eqs. (2.8) and (2.9) in
\cite{HAGIWARA1989569}, and are copied over with minor changes on notations:
\begin{align}
    & L^-_\pm(q^2, \theta_l) = -2 \sqrt{q^2} v d_\pm
    & L^-_0(q^2, \theta_l) &= -2 \sqrt{q^2} v d_0
    & L^-_s(q^2, \theta_l) &= 0
    \label{eqn:b-d-l-current-1} \\
    & L^+_\pm(q^2, \theta_l) = \mp \sqrt{2} m_l v d_0
    & L^+_0(q^2, \theta_l) &= \sqrt{2} m_l v(d_+ - d_-)
    & L^+_s(q^2, \theta_l) &= -2 m_l v
    \label{eqn:b-d-l-current-2}
\end{align}
where $v = \sqrt{1 - \frac{m^2_l}{q^2}}$,
$d_\pm = \frac{1 + \cos\theta_l}{\sqrt{2}}$, and $d_0 = \sin\theta_l$.

\paragraph{Hadronic currents}
The calculation of the hadronic currents, however,
involves non-perturbative quantum chromodynamics (QCD) effects as both the
initial and final states are mesons which are quark-anti-quark pairs bounded by
the strong interactions of QCD.
Hence, these currents are not computable in an analytical way and each of them
can be re-expressed in terms of form factors (FFs),
which are a function of \qSq, instead.
These FFs are parameterized and estimated with relatively large theoretical
uncertainties with four types of methods:
use of functional properties of the matrix elements\footnote{
    Such as analyticity, unitarity, and dispersion relations.
},
heavy quark effective theory (HQET),
various quark models\footnote{
    For example QCD sum rule (QCDSR) and light cone sum rule (LCSR).
},
and lattice QCD
\cite{Bernlochner_2022}.
As mentioned in the introduction,
for \RDX ratio measurements,
the hadronic uncertainties are mostly factored out and the main importance
of the FFs lies in the precision of the lepton universality ratio predictions
and kinematic decay distributions.
% NOTE: Check the slides for CLN vs. BGL
%   https://indico.cern.ch/event/656737/contributions/2676125/attachments/1525023/2384258/Siegen.pdf
The estimations of FFs
and the differential decay rate (which will be provided shortly after)
are often performed at a particular $q^2$ value, such as when the $D$ meson is
produced at rest in the $B$ rest frame\footnote{
    Referred to as ``zero recoil'' due to the fact that the $D$ meson is
    \emph{not recoiling} in the $B$ rest frame.
},
and are extrapolated to the full $q^2$ range by performing fits to experimental
inputs.

While not analytically calculable,
the hadronic currents are constrained by the conservation of angular momentum,
which implies that the $D$ and the virtual $W$ should have the same helicity.
For $B \rightarrow D^0$ decays,
whose only possible helicity state is 0,
the only non-zero hadronic amplitudes are $H^0_0, H^0_s$.
For $B \rightarrow D^*$ decays,
the amplitudes $H^+_+$ and $H^-_-$ are allowed as well.

\paragraph{Differential decay rate}
The differential decay rate is found by first inserting
\cref{eqn:b-d-l-current-1,eqn:b-d-l-current-2}
into \cref{eqn:b-d-master-formula} then integrating over the phase space and
summing over the lepton helicities:
\begin{align}
    \frac{d\Gamma}{d q^2 d\cos\theta_l} =&
    \frac{G^2_F |V_{cb}|^2 | \vec{p}^*_D|^2 q^2}{256 \pi^3 m^2_B}
    \left(1 - \frac{m^2_l}{q^2}\right)^2 \times
    \nonumber \\
    &\bigg[
        (1 - \cos\theta_l)^2 |H_+|^2 + (1 + \cos\theta_l)^2 |H_-|^2 +
        2 \sin^2\theta_l |H_0|^2 +
    \nonumber \\
    & \frac{m^2_l}{q^2} \left(
        (\sin^2\theta_l(|H_+|^2 + |H_-|^2) + 2|H_s + H_0 \cos\theta_l|^2)
    \right)
    \bigg]
    \label{eqn:ff-ps}
\end{align}
where the \qSq dependence and the helicities of the $D$ meson in the hadronic
currents $H_{\{\pm,0,s\}}$ are omitted.

\paragraph{\RDX calculation}
The parameterization of the hadroic currents $H_i$ requires form factors,
which will be described later.
Once the $H_i$ are resolved,
plugging these into the differential decay rate gives
an explicit expression for $\frac{d\Gamma}{d q^2}$
which can be used to compute \RDX:

\begin{equation}
    \RDX = \frac{
        \int_{m^2_\tau}^{(m_B - m_{D^{(*)}})^2} d q^2 \frac{d\Gamma}{d q^2}
    }{
        \int_{m^2_\mu}^{(m_B - m_{D^{(*)}})^2} d q^2 \frac{d\Gamma}{d q^2}
    }
\end{equation}

In the next sections the form factor parameterizations relevant to this analysis
for \Dz, \Dstar, and \Dstst mesons are discussed.


\subsection{Form factors in $B \rightarrow \Dz\ellm\neulb$ decays}
\label{ref:theory:ff-d0}

As discussed before,
a form factor parameterization of the hadronic currents is needed in order to
evaluate the differential decay rate.
In this analysis,
simulates samples for $B \rightarrow \Dz \ellm\neulb$
and $B \rightarrow \Dstar\ellm\neulb$
are generated with the CLN
parameterization and later reweighted offline to BGL
(more details regarding the reweighting process will be provided in
\cref{ref:mc-cor:ff}).
The CLN and BGL parameterizations will be discussed next.

\paragraph{CLN}
For $B \rightarrow \Dz\ellm\neulb$ decays the axial vector part of the hadronic
current is vanishing due to conservation of angular momentum and parity
\cite{Bernlochner_2022}.
Each hadronic current is therefore characterized by \emph{two} FFs:
\begin{align}
    \bra{D} \bar{c} \gamma^\mu(1 - \gamma^5) b \ket{B} &=
    \bra{D} \bar{c} \gamma^\mu b \ket{B}
    \nonumber \\
    &=
    \sqrt{m_B m_D} \left[
        h_+(w)(v_B + v_D)^\mu + h_-(w)(v_B - v_D)^\mu
    \right]
    \label{eqn:b-d-cln-param}
\end{align}
where $v_i = \frac{p_i}{m_i}$,
$w(\qSq) \equiv v_B \cdot v_D = \frac{m^2_B + m^2_D - \qSq}{2 m_B m_D}$.
%%%%
In the CLN parameterization \cite{Caprini_1998},
the FFs $h_+, h_-$ are re-expressed as $V_1(w), S_1(w)$:
\begin{align}
    V_1(w) &= h_+(w) - \frac{m_B - m_D}{m_B + m_D} h_-(w) \\
    S_1(w) &= h_+(w) - \frac{m_B + m_D}{m_B - m_D} \cdot \frac{w-1}{w+1} h_-(w)
\end{align}
and an expression for $V_1(w)$ is obtained by first analytically continuing $w$
outside the physical region in the complex plane, then a conformal mapping
\begin{equation}
    w \rightarrow z: z =
    \frac{\sqrt{w+1} - \sqrt{2}}{\sqrt{w+1} + \sqrt{2}}
\end{equation}
is performed such that the physical region is mapped inside the unit circle of
the complex $z$-plane.
With dispersion techniques based on first principles and additional
consideration from heavy quark symmetries to further reduce number of degrees of
freedom,
$V_1(z)$ is expressed in powers of $z$, known as ``$z$-expansion'', in the
following form:
\begin{equation}
    V_1(w) = V_1(1) \cdot \left[
        1 - 8 \rho^2 z(w) + (51 \rho^2 - 10) z(w)^2 -
        (252 \rho^2 - 84) z(w)^3
    \right]
\end{equation}
where $V_1(0), \rho^2$ are FF parameters,
noting that $V_1(0)$ corresponds to the hadronic current at zero recoil.
The factor $V_1(0)$,
either fitted from external data with HQET constraints or computed with lattice
QCD,
represents an overall normalization which is \emph{not cancelled} when
performing CLN to BGL reweighting but \emph{is cancelled} in the subsequent
\RD ratio.

The leptonic current corresponding to $S_1$ is helicity suppressed and,
as a result,
it is proportional to the lepton mass.
Therefore this hadronic amplitude
is insensitive to input data
coming from semileptonic $B \rightarrow \Dz \ell^- \neuLb$ decays
to light leptons $\ell^- = e^-, \mun$
and is constrained based on HQET:
\begin{equation}
    S_1(w) = \left\{
        1 + \Delta(-0.019 + 0.041(w - 1) - 0.015(w - 1)^2)
    \right\} V_1(w)
\end{equation}
where $\Delta$ is a parameter with its valued derived from HQET constraints.

The remaining parameter $\rho^2$
is obtained from a fit to experimentally measured \qSq distributions by
saturating the dispersive bounds at $1\sigma$
uncertainty \cite{Bernlochner_2022}.
The values for the parameters listed above are copied from the corresponding MC
generation code,
and are listed in \cref{tab:ff-cln-b-d}.

\begin{table}[!htb]
    \centering
    \caption{
        $B \rightarrow \Dz \ellm\neulb$ CLN FF parameterization.
    }
    \label{tab:ff-cln-b-d}
    \begin{tabular}{c|c|c}
        \toprule
        \textbf{FF parameter} & \textbf{Value} & \textbf{\Hammer name} \\
        \midrule
        $V_1(0)$ & 1.035 & \smalltt{G1}     \\
        $\rho^2$ & 1.131 & \smalltt{RhoSq}  \\
        $\Delta$ & 0.38  & \smalltt{Delta}  \\
        \bottomrule
    \end{tabular}
\end{table}

\paragraph{BGL}
Similar to CLN, the BGL parameterization
\cite{Boyd_1995,Boyd:1997kz}
is based on dispersion relations, analyticity, and crossing symmetry but
with reduced constraints,
providing more flexibility at the cost of a large number of parameters.
The BGL parameterization is more suited to a data-driven approach of the
determination of the FFs.

BGL adopts the following parameterization for
$B \rightarrow \Dz\ellm\neulb$ decays:
\begin{equation}
    \bra{D} \bar{c} \gamma^\mu b \ket{B} =
    f_+ (p_B + p_D)^\mu + (f_0 - f_+) \frac{q^\mu}{q^2} (m^2_B - m^2_D)
\end{equation}
Both $f_+, f_0$ are expressed as a series of $z$:
\begin{equation}
    f_{+,0}(z(w)) = \frac{1}{P_{+,0}(z(w))\phi_{+,0}(z(w))}
    \sum^\infty_{n = 0} a_{+,0}^n z(w)^n
\end{equation}
where $P_{+,0}(w)$ are known as \emph{Blaschke factors} with their expressions
listed in eq. (2.12) in \cite{Bigi_2016},
and $\phi_{+,0}$, \emph{outer functions}, are defined in eqs. (2.13) and (2.14)
of the same reference.
The $a_{+,0}$ parameters are taken from Table 4 in \cite{Bigi_2016} at $N = 2$,
which are obtained from a fit to a combination of
an unquenched lattice calculation at \emph{non-zero recoil} and experimental
data.
The $N = 2$ fit result is chosen because it is the only case that the best
fitted values lie within the unitary bounds and has a non-trivial correlation
matrix as reported in
Table 5 of \cite{Bigi_2016}.
The correspondence between the parameter notations used in this paper,
in Ref.~\cite{Bigi_2016}, and in \Hammer is displayed in
\cref{tab:ff-bgl-b-d}.

\begin{table}[!htb]
    \centering
    \caption{
        $B \rightarrow \Dz \ellm\neulb$ BGL FF parameterization
        notation correspondence.
    }
    \label{tab:ff-bgl-b-d}
    \begin{tabular}{c|c|c}
        \toprule
        \textbf{FF parameter} & \textbf{Ref.~\cite{Bigi_2016}} & \textbf{\Hammer name} \\
        \midrule
        $a_+^n$     & $a_n$     & \smalltt{ap}, a vector     \\
        $a_0^n$     & $b_n$\parnote{
            The $b_0$ coefficient is fixed by the other coefficients:
            $b_0 = 4.99 a_0 + 0.32 a_1 + 0.021 a_2 - 0.065 b_q - 0.004 b_2$
            due to the constraint that $f_+ = f_0$ at $\qSq = 0$.
        }
                                & \smalltt{a0}, a vector     \\
        \bottomrule
    \end{tabular}
    \begin{flushleft}
        \parnotes
    \end{flushleft}
\end{table}


\subsection{Form factors in $B \rightarrow \Dstar\ellm\neulb$ decays}
\label{ref:theory:ff-dst}

\paragraph{CLN}
For $B \rightarrow \Dstar\ellm\neulb$ decays,
the hadronic currents are parameterized by \emph{four} FFs,
$V$ and $A_{0,1,2}$:
\begin{align}
    \bra{\Dstar} \bar{c} \gamma^\mu b \ket{B}
    =\hphantom{ } &
        2i V (m_B + m_\Dstar) \varepsilon^{\mu\nu\alpha\beta}
        \epsilon^*_\nu (p_\Dstar)_\alpha (p_B)_\beta \\
    \bra{\Dstar} \bar{c} \gamma^\mu \gamma^5 b \ket{B}
    =\hphantom{ } &
        A_1 (m_B + m_\Dstar) \epsilon^{*\mu} -
        \nonumber \\
    =\hphantom{ } &
        A_2 \frac{\epsilon^* \cdot \vec{p_B} (p_B + p_\Dstar)^\mu}{m_B + m_\Dstar} +
        \nonumber \\
    =\hphantom{ } &
        2 m_\Dstar q^\mu (A_0 - A_3) \frac{\epsilon^* \cdot \vec{p_B}}{q^2}
\end{align}
in which $2 m_\Dstar A_3 = A_1(m_B + m_\Dstar) - A_2(m_B - m_\Dstar)$.
In the CLN parameterization,
these FFs are re-expressed as:
\begin{align}
    & A_1(w) = \frac{w+1}{2} r_\Dstar h_{A_1}(w)
    & A_0(w) = \frac{R_0(w)}{r_\Dstar} h_{A_1}(w)
    \label{eqn:b-dst-ff-param-1} \\
    & A_2(w) = \frac{R_2(w)}{r_\Dstar} h_{A_1}(w)
    & V(w) = \frac{R_1(w)}{r_\Dstar} h_{A_1}(w)
    \label{eqn:b-dst-ff-param-2}
\end{align}
where $r_\Dstar = 2 \frac{\sqrt{m_B m_\Dstar}}{m_B + m_\Dstar}$.
With dispersive bounds, these are expressed as:
\begin{align}
    h_{A_1}(w) &= h_{A_1}(1) \left[
        1 - 8 \rho^2 z(w) + (53 \rho^2 - 15) z(w)^2 - (231 \rho^2 - 91) z(w)^3
    \right] \label{eqn:cln-dst-bad-correlation} \\
        R_1(w) &= R_1(1) - 0.12(w-1) + 0.05(w-1)^2
        \label{eqn:cln-dst-bad-coeff-1} \\
    R_2(w) &= R_2(1) + 0.11(w-1) - 0.06(w-1)^2
        \label{eqn:cln-dst-bad-coeff-2} \\
    R_0(w) &= R_0(1) - 0.11(w-1) + 0.01(w-1)^2
        \label{eqn:cln-dst-bad-coeff-3}
\end{align}
The parameters used in $B \rightarrow \Dstar\ellm\neulb$ MC generation are
listed in \cref{tab:ff-cln-b-dst}.

\begin{table}[!htb]
    \centering
    \caption{
        $B \rightarrow \Dstar \ellm\neulb$ CLN FF parameterization.
    }
    \label{tab:ff-cln-b-dst}
    \begin{tabular}{c|c|c}
        \toprule
        \textbf{FF parameter} & \textbf{Value} & \textbf{\Hammer name} \\
        \midrule
        $h_{A_1}(1)$ & 0.908 & \smalltt{F1}     \\
        $\rho^2$     & 1.122 & \smalltt{RhoSq}  \\
        $R_1(1)$     & 1.270 & \smalltt{R1}  \\
        $R_2(1)$     & 0.852 & \smalltt{R2}  \\
        $R_0(1)$     & 1.15  & \smalltt{R0}  \\
        \bottomrule
    \end{tabular}
\end{table}

In the CLN parameterization,
%%%%
the slope and curvature,
that is, the coefficients of the $z$ and $z^2$ terms
in \cref{eqn:cln-dst-bad-correlation},
of the Isgur-Wise functions\footnote{
    The form factor description in the heavy quark limit.
}
share the same parameter $\rho$,
making the two too closely cross-constrained by the model.
%%%%
Additionally,
the fit of the normalization factors of the form factor ratios at zero recoil
$R_i(w = 1)$
does not take the fitted slopes $R'_i(w = 1)$ into account,
but uses the fixed values,
such as $-0.12$ in \cref{eqn:cln-dst-bad-coeff-1},
obtained from heavy quark expansion considering the contributions from
$1/m_Q$ instead.
Such an approach is not self-consistent
\cite{LHCb-ANA-2020-056}.
Therefore, it is preferable to use the BGL parameterization.


\paragraph{BGL}
The BGL parameterization scheme for $B \rightarrow \Dstar\ellm\neulb$ decays is
\cite{Bazavov_2021}:
\begin{align}
    \bra{\Dstar} \bar{c} \gamma^\mu b \ket{B}
    =\hphantom{ } &
        i\sqrt{m_B m_\Dstar} h_V \varepsilon^{\mu\nu\alpha\beta}
        \epsilon^*_\nu (v_\Dstar)_\alpha (v_B)_\beta \\
    \bra{\Dstar} \bar{c} \gamma^\mu \gamma^5 b \ket{B}
    =\hphantom{ } &
        \sqrt{m_B m_\Dstar} \Big[
            h_{A_1}(w+1) \epsilon^{*\mu} -
    \nonumber \\
    &\hphantom{\sqrt{m_B m_\Dstar} \Big[}
            h_{A_2} (\epsilon^* \cdot \vec{v_B}) (v_B)^\mu -
            h_{A_3} (\epsilon^* \cdot \vec{v_B}) (v_\Dstar)^\mu
        \Big]
\end{align}
with the FFs $\{h_V,h_{A_1,A_2,A_3}\}$ re-expressed as
$\{g,f,\mathcal{F}_1,\mathcal{F}_2\}$:
\begin{align}
    g(w) =& \frac{h_V(w)}{m_B \sqrt{r}} \\
    f(w) =& m_B \sqrt{r}(1 + w) h_{A_1}(w) \\
    \mathcal{F}_1 =& m^2_B \sqrt{r}(1+w)\left[
        (w - r) h_{A_1}(w) - (w - 1)(r h_{A_2}(w) + h_{A_3}(w))
    \right] \\
    \mathcal{F}_2 =& \frac{1}{\sqrt{r}} \left[
        (1+w)h_{A_1}(w) + (rw - 1) h_{A_2}(w) + (r-w)h_{A_3}(w)
    \right]
\end{align}
where $r = \frac{m_\Dstar}{m_B}$.
These FFs are proportional to $H_- - H_+, H_+ + H_-, H_0, H_s$ accordingly.
With $z$-expansion, the FFs are expanded into the form:
\begin{align}
    g(w) =& \frac{1}{P_{1^-}(z(w))\phi_g(z(w))}
        \sum^N_{n = 0} a_n z(w)^n \\
    f(w) =& \frac{1}{P_{1^+}(z(w))\phi_f(z(w))}
        \sum^N_{n = 0} b_n z(w)^n \\
    \mathcal{F}_1(w) =& \frac{1}{P_{1^+}(z(w))\phi_{\mathcal{F}_1}(z(w))}
        \sum^N_{n = 0} c_n z(w)^n \\
    \mathcal{F}_2(w) =& \frac{1}{P_{0^-}(z(w))\phi_{\mathcal{F}_2}(z(w))}
        \sum^N_{n = 0} d_n z(w)^n
\end{align}
where $P_i(z(w))$ are given in eq. (61) in \cite{Bazavov_2021}, and
$\phi_i(z(w))$ in eqs. (67--70) in the same reference.
The coefficients $\{a_n, b_n, c_n, d_n\}$ are given by a fit to unquenched
lattice calculation at non-zero recoil and data;
their values are listed in the right-most column of Table XII in the
\textbf{v1} version of \cite{Bazavov_2021}.
The parameter naming correspondence is displayed in
\cref{tab:ff-bgl-b-dst}.

\begin{table}[!htb]
    \centering
    \caption{
        $B \rightarrow \Dstar \ellm\neulb$ BGL FF parameterization
        notation correspondence.
    }
    \label{tab:ff-bgl-b-dst}
    \begin{tabular}{c|c|c}
        \toprule
        \textbf{FF parameter} & \textbf{Ref.~\cite{Bazavov_2021}} & \textbf{\Hammer name} \\
        \midrule
        $a_n$     & $a_n$     & \smalltt{avec}, a vector     \\
        $b_n$     & $b_n$     & \smalltt{bvec}, a vector     \\
        $c_n$     & $c_n$     & \smalltt{cvec}, a vector\parnote{
            The $c_n$ coefficients starts from $c_1$ instead of $c_0$.
        } \\
        $d_n$     & $d_n$     & \smalltt{dvec}, a vector\parnote{
            The $c_3$ and $d_2$ coefficients are unused in the reweighting
            based on a claim that they are not implemented in \Hammer.
            Latest version of \Hammer may have added support to these
            parameters.
        } \\
        \bottomrule
    \end{tabular}
    \begin{flushleft}
        \parnotes
    \end{flushleft}
\end{table}


\subsection{Form factors in $B \rightarrow D^{**}\ellm\neulb$ decays}
\label{ref:theory:ff-dstst}

\paragraph{ISGW2}
The ISGW2 parameterization \cite{Scora_1995} is based on a
non-relativistic constituent quark model
\cite{Bernlochner_2022}
and is treated as fully predictive,
that is, without any undetermined parameter.
While historically important in providing a description of the FFs,
it is no longer considered to provide state-of-art FFs
due to its inability to describe the experimental data.

\paragraph{BLR}
The BLR parameterization \cite{Bernlochner_2018} provides a
FF parameterization for the four lightest ($1P$) excited \Dstst
mesons
($D^*_0$, $D'_1$, $D_1$, $D^*_2$),
grouped into two heavy quark isospin doublets
$D^{1/2+}: (D^*_0, D'_1)$ and $D^{3/2+}: (D_1, D^*_2)$.
This parameterization utilizes a HQET expansion,
considers all possible $b \rightarrow c\ellm\neulb$ four-Fermi interactions,
and includes
$\mathcal{O}(\Lambda_\text{QCD} / m_{c,b})$ and $\mathcal{O}(\alpha_s)$
corrections
which results in FFs that can be improved by data-driven fits.
The parameterizations are listed separately in
\cref{tab:ff-blr-1-2,tab:ff-blr-3-2} for the two isospin doublets.

\begin{table}[!htb]
    \centering
    \caption{
        FF parameterization for $D^{1/2+}$ doublet $(D^*_0, D'_1)$.
    }
    \label{tab:ff-blr-1-2}
    \begin{tabular}{c|c|c}
        \toprule
        \textbf{FF parameter} & \textbf{Ref.~\cite{Bernlochner_2018}} & \textbf{\Hammer name} \\
        \midrule
        $\zeta(1)$       & 0.7     & \smalltt{zt1}     \\
        $\zeta'$         & 0.2     & \smalltt{ztp}     \\
        $\hat{\zeta}_1$  & 0.6     & \smalltt{zeta1}   \\
        \bottomrule
    \end{tabular}
\end{table}

\begin{table}[!htb]
    \centering
    \caption{
        FF parameterization for $D^{3/2+}$ doublet $(D_1, D^*_2)$.
    }
    \label{tab:ff-blr-3-2}
    \begin{tabular}{c|c|c}
        \toprule
        \textbf{FF parameter} & \textbf{Ref.~\cite{Bernlochner_2018}} & \textbf{\Hammer name} \\
        \midrule
        $\tau(1)$       & 0.7     & \smalltt{t1}     \\
        $\tau'$         & -1.6    & \smalltt{tp}     \\
        $\hat{\tau}_1$  & -0.5    & \smalltt{tau1}   \\
        $\hat{\tau}_2$  & 2.9     & \smalltt{tau2}   \\
        \bottomrule
    \end{tabular}
\end{table}
