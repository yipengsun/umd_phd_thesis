% Potentially useful sources:
%   https://physics.stackexchange.com/questions/707380/is-lepton-flavour-universality-an-accidental-symmetry-of-the-standatd-model
%   https://physics.stackexchange.com/questions/672626/what-is-an-accidental-symmetry
%   https://physics.stackexchange.com/questions/55350/two-ways-to-form-su2-singlets
%   https://physics.stackexchange.com/questions/607444/chiral-symmetry
%   https://physics.stackexchange.com/questions/429180/how-does-bhabha-scattering-imply-the-existence-of-quark-lepton-substructure
%   https://physics.stackexchange.com/questions/96362/what-does-a-rm-su2-isospin-doublet-really-mean
%   https://physics.stackexchange.com/questions/385159/quark-gluon-color-relationship-in-pure-qcd
%   https://physics.stackexchange.com/questions/327008/how-do-the-infinite-dimensional-representations-of-the-poincar%C3%A9-group-work

% Schwartz
%   p. 585 (hypercharge)


\chapter{Theory of semileptonic $B$ decays}
\label{ref:theory}

\section{Lepton flavor universality in the SM}
\label{ref:theory:lfu}

Through a close collaboration between experimentalists and theorists,
particle physicists were able to come up with a \emph{quantum field theory}
to describe \emph{interactions} of \emph{elementary particles},
known as the standard model of particle physics (SM).
In this section a construction of the SM is outlined following
\cite{Robinson_2011}
and its implications on lepton flavor universality are discussed.

To construct the SM,
in general one starts from a free Lagrangian density,
then demands the Lagrangian is invariant under local gauge transformations of
certain Lie algebras to arrive at a Lagrangian with interactions,
adding appropriate terms when necessary.
Then it is (sometimes) possible to compute the scattering amplitude from an
initial state to a final state in a perturbative manner\footnote{
    To be more precise, the scattering amplitude $\braket{f}{i}$
    between an initial state $\ket{i}$
    and a final state $\ket{f}$
    can be calculated in the following
    manner \cite{Weigand}:
    \begin{enumerate}[noitemsep,nosep]
        \item The amplitude $\braket{f(y_1, \dots, y_r)}{i(x_1, \dots, x_n)}$
            is related to the interactive vacuum
            $\ket{\Omega}$ by the LSZ reduction formula:
            $\braket{f}{i} \propto \bra{\Omega} T \phi(y_1) \dots \phi(y_r) \phi(x_1) \dots \phi(x_n) \ket{\Omega}$,
            where $\phi(x)$, an operator in the Heisenberg picture,
            creates a particle at spacetime point $x$ in the interactive
            vacuum $\ket{\Omega}$,
            and $T$ stands for time-ordering.

        \item Denote the free and the interaction part of the Hamiltonian
            as $H_0, H_\text{int}$, accordingly.

        \item At a given time $t$, a Heisenberg operator and a interactive
            operator are related by
            $\phi(t_0, \vec{x}) = U^\dagger (t, t_0) \Phi_I(t, \vec{x}) U(t, t_0)$,
            where $U(t, t_0) = T e^{-i \int_{t_0}^t H_I(t') d t'}$, and
            $H_I(t) = e^{iH_0(t-t_0)} H_\text{int} e^{-iH_0(t-t_0)}$.

        \item The interactive vacuum $\ket{\Omega}$ and the free vacuum
            $\ket{0}$ are related by
            $\ket{\Omega} \propto \lim_{\tau \rightarrow \infty(1 - i\epsilon)} U(t_0, -\tau)\ket{0}$.

        \item The scattering amplitude can now be written as
            $\braket{f}{i} \propto \bra{0} T \prod_i \Phi_I(x_i) e^{-i \int_{-\infty}^{\infty} H_I(t) dt} \ket{0}$,
            where $\Phi_I$ is the $\phi$ operator in the interaction picture
            and has a \emph{free} mode expansion that can create and destroy
            particles in the \emph{free} vacuum $\ket{0}$.

        \item At this point,
            the exponential involving $H_I$ are expanded in a Taylor series:
            $e^{-i \int_{-\infty}^{\infty} H_I(t) dt} = \sum_n \frac{(-i)^n}{n!} T\left\{
                \int_{-\infty}^{\infty} \cdots \int_{-\infty}^{\infty}
                H_{I}^1 \cdots H_{I}^n d t_1 \cdots d t_n
            \right\}$.
            Each term in the series are turned into a series of
            normal-ordered operators with contractions via the Wick's theorem.

        \item The contractions are Feynman propagators with the
            corresponding field operators which are specified by the commutation
            relations between the operators.

        \item The first (non-vanishing) order terms are identified as tree-level
            Feynman diagrams.
    \end{enumerate}
    Higher order terms may involve infinities and require renormalization.
}.

For example, consider the construction of the Quantum Eletrodynamics (QED)
first:
the Dirac equation, a relativistic wave equation for spin $1/2$ particles\footnote{
    This is because $\gamma^\mu$ are $4 \times 4$ matrices constructable from
    $2 \times 2$ Pauli matrices.
    Using the Weyl representation, the 4-component spinor field $\psi$ can
    be viewed as a 2-component left-handed spin $1/2$ particle and a 2-component
    right-handed spin $1/2$ particle.
}, implies\footnote{
    Note that $\psi^\dagger \psi$ is not a Lorentz scalar because:
    Lorentz transformations $S$ for spinors are generated by
    the $\gamma^\mu$ by the following relation:
    $S^{\mu\nu} = \frac{i}{4}[\gamma^\mu, \gamma^\nu]$,
    and $S = \exp(\frac{i}{2}\omega)$
} the following Lagrangian:




\section{Form factors in $B \rightarrow D^{(*)}\ell\neulb$ decays}
\label{ref:theory:ff-d0-dst}


\section{Computation of \RD and \RDst}
\label{ref:theory:rdx}


\section{Form factors in $B \rightarrow D^{**}\ell\neulb$ decays}
\label{ref:theory:ff-dstst}
