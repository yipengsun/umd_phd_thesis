% Potentially useful sources:
%   https://physics.stackexchange.com/questions/707380/is-lepton-flavour-universality-an-accidental-symmetry-of-the-standatd-model
%   https://physics.stackexchange.com/questions/672626/what-is-an-accidental-symmetry
%   https://physics.stackexchange.com/questions/55350/two-ways-to-form-su2-singlets
%   https://physics.stackexchange.com/questions/607444/chiral-symmetry
%   https://physics.stackexchange.com/questions/429180/how-does-bhabha-scattering-imply-the-existence-of-quark-lepton-substructure
%   https://physics.stackexchange.com/questions/96362/what-does-a-rm-su2-isospin-doublet-really-mean
%   https://physics.stackexchange.com/questions/385159/quark-gluon-color-relationship-in-pure-qcd
%   https://physics.stackexchange.com/questions/327008/how-do-the-infinite-dimensional-representations-of-the-poincar%C3%A9-group-work
%
% Very useful:
%   https://physics.stackexchange.com/questions/266963/what-are-the-differences-if-any-between-the-dysons-series-definition-and-the
%   https://pages.cs.wisc.edu/~guild/symmetrycompsproject.pdf
%   https://physics.stackexchange.com/questions/578059/why-the-little-group-of-a-massless-spin-1-particle-is-iso2-rather-than-so2
%   https://math.stackexchange.com/questions/1334965/direct-sum-vs-direct-product-vs-tensor-product
%
% Useful papers not cited:
%   On gauge symmetry: https://arxiv.org/pdf/1901.10420.pdf


\chapter{Theory of semileptonic $B$ decays}
\label{ref:theory}

\section{Lepton flavor universality in the SM}
\label{ref:theory:lfu}

Through a close collaboration between experimentalists and theorists,
particle physicists were able to come up with a \emph{quantum field theory}
to describe \emph{interactions} of \emph{elementary particles},
known as the standard model of particle physics (SM).
In this section a construction of the SM is outlined following
\cite{Robinson_2011,Schwichtenberg_2018}
and its implications on lepton flavor universality are discussed\footnote{
    ``I am an experimentalist!'' says the author waving his hands.
}.

It is experimentally established that the speed of light in vacuum is constant
in any reference frame.
Then, all transformations preserving the speed of light\footnote{
    Each of such transformations, denote as $A$, must leave the Minkowski metric
    $\eta$ intact: $A^T \eta A = \eta$.
} can be identified:
rotations, Lorentz boosts, parity ($\vec{x} \rightarrow -\vec{x}$),
time reversal ($t \rightarrow -t$), and translations.
All but the translation transformations form a group called
``Lorentz group''.
Focusing on the restricted Lorentz group $SO^+(1, 3)$,
which preserves direction of space and time\footnote{
    Loosely speaking, Lorentz group contains 4 ``classes'': $\{I, T, P, TP\}$.
    $SO^+(1,3)$ is the one that contains the identity element.
    The other 3 can be obtained by applying parity ($P$), time reversal ($T$),
    or both ($TP$) to $SO^+(1,3)$.
},
and accepting that it is advantageous to extend $SO^+(1,3)$ to work with vectors
with complex components,
it is shown that the extended group, $SO^+(1,3)_\mathbb{C}$,
contains two copies of $SU(2)_\mathbb{C}$.
% SO+(1,3) = SU(2) x SU(2), direct product

One may ask the following question, then:
``What kinds of object does the group $SO^+(1,3)_\mathbb{C}$ act on?''
To answer this question, the group elements can be represented by $n \times n$
matrices
(with the understanding that any group element $\Lambda$ can be generated from 3
rotation generators $\vec{J}$ and 3 boost generators $\vec{K}$
with the exponential map:
$\Lambda = e^{i \vec{\theta} \cdot \vec{J} + i \vec{\phi} \cdot \vec{K}}$,
and the commutation relations between any two generators can be worked out in
any representation),
and the objects the group acts on by vectors of $n$ complex numbers.
Then the properties of a particular ``representation'' can be studied with
the familiar tool of linear algebra.
As a side note,
$SU(2)_\mathbb{C}$ can be represented by $2j + 1$ dimensional matrices,
where $j$ is a half-integer;
the $j = \frac{1}{2}$ case has 3 Pauli matrices as the group generators.
Since $SO^+(1,3)_\mathbb{C}$ contains two copies of $SU(2)_\mathbb{C}$,
its representations can be labelled as $(j, j')$.
It turns out the answer to the question at the beginning of the paragraph is:
each \emph{irreducible representation}\footnote{
    Roughly speaking, there is no transformation that can simultaneously
    block-diagonalize all matrices representing the group elements.
}
corresponds to a specific type of \emph{elementary particle}.
The $(0, 0)$ representation corresponds to spin-0 scalar particles;
the $(\frac{1}{2}, 0)$ and $(0, \frac{1}{2})$ representations both corresponds to
spin-$\frac{1}{2}$ particles,
with the former left-handed and the latter right-handed;
the $(\frac{1}{2}, \frac{1}{2}) = (\frac{1}{2}, 0) \otimes (0, \frac{1}{2})$
representation corresponds to spin-1 particles.
There are many more irreducible representations,
but in SM only the spin 0, $\frac{1}{2}$, and 1 particles are present.


in general one starts from a free Lagrangian density,
then demands the Lagrangian is invariant under local gauge transformations of
certain Lie algebras to arrive at a Lagrangian with interactions,
adding appropriate terms when necessary.
Then it is (sometimes) possible to compute the scattering amplitude from an
initial state to a final state in a perturbative manner\footnote{
    To be more precise, the scattering amplitude $\braket{f}{i}$
    between an initial state $\ket{i}$
    and a final state $\ket{f}$
    can be calculated in the following
    manner \cite{Weigand}:
    \begin{enumerate}[noitemsep,nosep]
        \item The amplitude $\braket{f(y_1, \dots, y_r)}{i(x_1, \dots, x_n)}$
            is related to the interactive vacuum
            $\ket{\Omega}$ by the LSZ reduction formula:
            $\braket{f}{i} \propto \bra{\Omega} T \phi(y_1) \dots \phi(y_r) \phi(x_1) \dots \phi(x_n) \ket{\Omega}$,
            where $\phi(x)$, an operator in the Heisenberg picture,
            creates a particle at spacetime point $x$ in the interactive
            vacuum $\ket{\Omega}$,
            and $T$ stands for time-ordering.

        \item Denote the free and the interaction part of the Hamiltonian
            as $H_0, H_\text{int}$, accordingly.

        \item At a given time $t$, a Heisenberg operator and a interactive
            operator are related by
            $\phi(t_0, \vec{x}) = U^\dagger (t, t_0) \Phi_I(t, \vec{x}) U(t, t_0)$,
            where $U(t, t_0) = T e^{-i \int_{t_0}^t H_I(t') d t'}$, and
            $H_I(t) = e^{iH_0(t-t_0)} H_\text{int} e^{-iH_0(t-t_0)}$.

        \item The interactive vacuum $\ket{\Omega}$ and the free vacuum
            $\ket{0}$ are related by
            $\ket{\Omega} \propto \lim_{\tau \rightarrow \infty(1 - i\epsilon)} U(t_0, -\tau)\ket{0}$.

        \item The scattering amplitude can now be written as
            $\braket{f}{i} \propto \bra{0} T \prod_i \Phi_I(x_i) e^{-i \int_{-\infty}^{\infty} H_I(t) dt} \ket{0}$,
            where $\Phi_I$ is the $\phi$ operator in the interaction picture
            and has a \emph{free} mode expansion that can create and destroy
            particles in the \emph{free} vacuum $\ket{0}$.

        \item At this point,
            the exponential involving $H_I$ are expanded in a Taylor series:
            $e^{-i \int_{-\infty}^{\infty} H_I(t) dt} = \sum_n \frac{(-i)^n}{n!} T\left\{
                \int_{-\infty}^{\infty} \cdots \int_{-\infty}^{\infty}
                H_{I}^1 \cdots H_{I}^n d t_1 \cdots d t_n
            \right\}$.
            Each term in the series are turned into a series of
            normal-ordered operators with contractions via the Wick's theorem.

        \item The contractions are Feynman propagators with the
            corresponding field operators which are specified by the commutation
            relations between the operators.

        \item The first (non-vanishing) order terms are identified as tree-level
            Feynman diagrams.
    \end{enumerate}
    Higher order terms may involve infinities and require renormalization.
}.

For example, consider the construction of the Quantum Eletrodynamics (QED)
first:
the Dirac equation, a relativistic wave equation for spin $1/2$ particles\footnote{
    This is because $\gamma^\mu$ are $4 \times 4$ matrices constructable from
    $2 \times 2$ Pauli matrices.
    Using the Weyl representation, the 4-component spinor field $\psi$ can
    be viewed as a 2-component left-handed spin $1/2$ particle and a 2-component
    right-handed spin $1/2$ particle.
}, implies\footnote{
    Note that $\psi^\dagger \psi$ is not a Lorentz scalar because:
    Lorentz transformations $S$ for spinors are generated by
    the $\gamma^\mu$ by the following relation:
    $S^{\mu\nu} = \frac{i}{4}[\gamma^\mu, \gamma^\nu]$,
    and $S = \exp(\frac{i}{2}\omega)$
} the following Lagrangian:




\section{Form factors in $B \rightarrow D^{(*)}\ell\neulb$ decays}
\label{ref:theory:ff-d0-dst}


\section{Computation of \RD and \RDst}
\label{ref:theory:rdx}


\section{Form factors in $B \rightarrow D^{**}\ell\neulb$ decays}
\label{ref:theory:ff-dstst}
