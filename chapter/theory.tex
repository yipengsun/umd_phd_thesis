% Potentially useful sources:
%   https://physics.stackexchange.com/questions/707380/is-lepton-flavour-universality-an-accidental-symmetry-of-the-standatd-model
%   https://physics.stackexchange.com/questions/672626/what-is-an-accidental-symmetry
%   https://physics.stackexchange.com/questions/55350/two-ways-to-form-su2-singlets
%   https://physics.stackexchange.com/questions/607444/chiral-symmetry
%   https://physics.stackexchange.com/questions/429180/how-does-bhabha-scattering-imply-the-existence-of-quark-lepton-substructure
%   https://physics.stackexchange.com/questions/96362/what-does-a-rm-su2-isospin doublet-really-mean
%   https://physics.stackexchange.com/questions/385159/quark-gluon-color-relationship-in-pure-qcd
%   https://physics.stackexchange.com/questions/327008/how-do-the-infinite-dimensional-representations-of-the-poincar%C3%A9-group-work
%   Rep. of Lorentz group: http://sharif.edu/~sadooghi/QFT-I-96-97-2/LorentzPoincareMaciejko.pdf
%
% Very useful:
%   https://physics.stackexchange.com/questions/266963/what-are-the-differences-if-any-between-the-dysons-series-definition-and-the
%   https://pages.cs.wisc.edu/~guild/symmetrycompsproject.pdf
%   https://physics.stackexchange.com/questions/578059/why-the-little-group-of-a-massless-spin 1-particle-is-iso2-rather-than-so2
%   https://math.stackexchange.com/questions/1334965/direct-sum-vs-direct-product-vs-tensor-product
%   https://physics.stackexchange.com/questions/619178/standard-model-notation-on-doublets
%   https://physics.stackexchange.com/questions/190416/what-role-does-spontaneous-symmetry-breaking-play-in-the-higgs-mechanism
%
% Useful papers not cited:
%   On gauge symmetry: https://arxiv.org/pdf/1901.10420.pdf
%   2-component spinor notation: https://arxiv.org/pdf/0812.1594.pdf


\chapter{Theory of semileptonic $B$ decays}
\label{ref:theory}

\section{Construction of the electroweak theory in the SM}

Through a close collaboration between experimentalists and theorists,
particle physicists were able to come up with a \emph{quantum field theory}
to describe \emph{interactions} of \emph{elementary particles},
known as the standard model of particle physics (SM).
A construction of the SM is outlined\footnote{
    ``I am an experimentalist!'' says the author waving his hands.
} first with the help of
\cite{Robinson_2011,Schwichtenberg_2018},
followed by brief discussions on quantum eletrodynamics (QED),
the electroweak theory, and the lepton flavor universality in the SM.
The full SM,
which includes the strong interaction on top of the electroweak interactions,
can be constructed in the same way but will not be discussed because the strong
interaction is not the focus of this thesis.


\subsection{Outline of the construction}

It is experimentally established that the speed of light in vacuum is constant
in any reference frame.
Then, all transformations preserving the speed of light\footnote{
    Each of such transformations, denote as $A$, must leave the Minkowski metric
    $\eta$ intact: $A^T \eta A = \eta$.
} can be identified:
rotations, Lorentz boosts, parity ($\vec{x} \rightarrow -\vec{x}$),
time reversal ($t \rightarrow -t$), and translations.
All but the translation transformations form a group called
``Lorentz group''.
Focusing on the restricted Lorentz group $SO^+(1, 3)$,
which preserves direction of space and time\footnote{
    Loosely speaking, Lorentz group contains 4 ``classes'': $\{I, T, P, TP\}$.
    $SO^+(1,3)$ is the one that contains the identity element.
    The other 3 can be obtained by applying parity ($P$), time reversal ($T$),
    or both ($TP$) to $SO^+(1,3)$.
},
and accepting that it is advantageous to extend $SO^+(1,3)$ to work with vectors
with complex components,
it is shown that the extended group, $SO^+(1,3)_\mathbb{C}$,
contains two copies of $SU(2)_\mathbb{C}$.
% SO+(1,3) = SU(2) x SU(2), direct product

One may ask the following question, then:
``What kinds of object does the group $SO^+(1,3)_\mathbb{C}$ act on?''
To answer this question, the group elements can be represented\footnote{
    With the understanding that any group element $\Lambda$ can be generated
    from 3 rotation generators $\vec{J}$ and 3 boost generators $\vec{K}$
    with the exponential map:
    $\Lambda = e^{i \vec{\theta} \cdot \vec{J} + i \vec{\phi} \cdot \vec{K}}$,
    and the commutation relations between the generators,
    namely $[J_i, J_j], [K_i, K_j], [J_i, K_j]$,
    can be worked out in any representation.
} by $n \times n$ matrices,
and the objects the group acts on by vectors of $n$ complex numbers.
Then the properties of a particular ``representation'' can be studied with
the familiar tool of linear algebra.
As a side note,
$SU(2)_\mathbb{C}$ can be represented by $2j + 1$ dimensional matrices,
where $j$ is a half-integer;
the $j = \frac{1}{2}$ case has 3 Pauli matrices as the group generators.
Since $SO^+(1,3)_\mathbb{C}$ contains two copies of $SU(2)_\mathbb{C}$,
its representations can be labelled as $(j, j')$.
It turns out the answer to the question at the beginning of the paragraph is:
each \emph{irreducible representation}\footnote{
    Roughly speaking, there is no transformation that can simultaneously
    block-diagonalize all matrices representing the group elements.
}
corresponds to a specific type of \emph{elementary particle}.
The $(0, 0)$ representation corresponds to spin 0 scalar particles;
the $(\frac{1}{2}, 0)$ and $(0, \frac{1}{2})$ representations both corresponds to
spin $\frac{1}{2}$ particles,
with the former left-handed and the latter right-handed;
the $(\frac{1}{2}, \frac{1}{2})$
representation corresponds to spin 1 particles.
There are many more irreducible representations,
but in the SM only the spin 0, $\frac{1}{2}$, and 1 particles are present.

Now the SM can be constructed as follows:
First, consider elementary particles with spin 0, $\frac{1}{2}$, and 1,
which are represented by linear operators in a space.
A Lorentz-invariant Lagrangian density\footnote{
    Will be referred as ``Lagrangian'' in later text for brevity.
} can be built from
linear, quadratic, and the lowest possible derivative terms of these
operators.
A Lagrangian built in this manner contains no interaction (i.e. a free
Lagrangian).
Then, require the Lagrangian to be invariant\footnote{
    Speaking with a limited knowledge: local gauge invariances represent
    mathematical redundancy in the description of the system,
    and must be preserved to have a consistent theory.
} under certain local gauge transformations
such as that from $U(1), SU(2)$, and $SU(3)$.
These requirements can be satisfied by adding/removing certain terms to the
Lagrangian;
these terms make interactions between elementary particles happen.
At this point, there is nothing quantum about the theory.
The theory is ``quantized'' by forcing canonical commutation/anti-commutation
relation for integer spin and half-integer spin particles,
respectively\footnote{
    Note that at this point, the terms
    \emph{elementary particles, interactive, and quantum field theory}
    are defined.
}.
Finally, the scattering amplitude from an initial state to a final state,
which is of utmost importance experimentally,
can be computed in a perturbative
manner\footnote{
    To be more precise, the scattering amplitude $\braket{f}{i}$
    between an initial state $\ket{i}$
    and a final state $\ket{f}$
    can be calculated in the following
    manner \cite{Weigand}:
    \begin{enumerate}[noitemsep,nosep]
        \item The amplitude $\braket{f(y_1, \dots, y_r)}{i(x_1, \dots, x_n)}$
            is related to the interactive vacuum
            $\ket{\Omega}$ by the LSZ reduction formula:
            $\braket{f}{i} \propto \bra{\Omega} T \phi(y_1) \dots \phi(y_r) \phi(x_1) \dots \phi(x_n) \ket{\Omega}$,
            where $\phi(x)$, an operator in the Heisenberg picture,
            creates and destroys particles at spacetime point $x$ in the
            interactive vacuum $\ket{\Omega}$,
            and $T$ stands for time-ordering.

        \item Denote the free and the interaction part of the Hamiltonian
            as $H_0, H_\text{int}$, accordingly.

        \item At a given time $t$, a Heisenberg operator and a interactive
            operator are related by
            $\phi(t_0, \vec{x}) = U^\dagger (t, t_0) \Phi_I(t, \vec{x}) U(t, t_0)$,
            where $U(t, t_0) = T e^{-i \int_{t_0}^t H_I(t') d t'}$, and
            $H_I(t) = e^{iH_0(t-t_0)} H_\text{int} e^{-iH_0(t-t_0)}$.

        \item The interactive vacuum $\ket{\Omega}$ and the free vacuum
            $\ket{0}$ are related by
            $\ket{\Omega} \propto \lim_{\tau \rightarrow \infty(1 - i\epsilon)} U(t_0, -\tau)\ket{0}$.

        \item The scattering amplitude can now be written as
            $\braket{f}{i} \propto \bra{0} T \prod_i \Phi_I(x_i) e^{-i \int_{-\infty}^{\infty} H_I(t) dt} \ket{0}$,
            where $\Phi_I$ is the $\phi$ operator in the interaction picture
            and has a \emph{free} mode expansion that can create and destroy
            particles in the \emph{free} vacuum $\ket{0}$.

        \item At this point,
            the exponential involving $H_I$ are expanded in a Taylor series:
            $e^{-i \int_{-\infty}^{\infty} H_I(t) dt} = \sum_n \frac{(-i)^n}{n!} T\left\{
                \int_{-\infty}^{\infty} \cdots \int_{-\infty}^{\infty}
                H_{I}^1 \cdots H_{I}^n d t_1 \cdots d t_n
            \right\}$.
            Each term in the series are turned into a series of
            normal-ordered operators with contractions via the Wick's theorem.

        \item The contractions are Feynman propagators with the
            corresponding field operators which are specified by the commutation
            relations between the operators.

        \item The first (non-trivial non-vanishing) order terms are identified
            as tree-level Feynman diagrams.
    \end{enumerate}
    Higher order terms may involve infinities and require renormalization.
}.


\subsection{Quantum eletrodynamics (QED)}
\label{qed}

Following the recipe provided above, first identify all\footnote{
    This refers to linear, quadratic, and lowest possible derivative terms.
} possible Lorentz scalar
terms in a free Lagrangian for spin $\frac{1}{2}$ particles:
the second order terms
$S_1 = (\chi_L)^\dagger \xi_R, S_2 = (\xi_R)^\dagger\chi_L$;
the first order partial derivative terms
$S_3 = (\chi_L)^\dagger \partial_\mu \bar{\sigma}^\mu \chi_L,
S_4 = (\xi_R)^\dagger \partial_\mu {\sigma}^\mu \xi_R$.
Each $\chi_L, \xi_R$ is a 2-component Weyl spinor.
This shows that both left- and right-handed particles are needed for Lorentz
invariance,
which motivates a 4-component Dirac spinor
$\psi = \bigl(\begin{smallmatrix} \chi_L \\ \xi_R \end{smallmatrix}\bigr)$.
%%%%
To combine $\psi$ into the form of $S_1, S_2$,
the matrix
$\gamma_0 = \bigl(\begin{smallmatrix} 0 & I_{2 \times 2} \\ I_{2 \times 2} & 0 \end{smallmatrix}\bigr)$
can be used such that $\psi^\dagger \gamma_0 \psi$ is a Lorentz scalar.
%%%%
On the other hand,
the forms of $S_3, S_4$ suggest the use of the matrices of the form
$\bigl(\begin{smallmatrix} \bar{\sigma}^\mu & 0 \\ 0 & \sigma^\mu \end{smallmatrix}\bigr)$,
which are generated by $\gamma_0 \gamma^\mu$,
where $\gamma^\mu = \bigl(\begin{smallmatrix} 0 & {\sigma}_\mu \\ \bar{\sigma}_\mu & 0 \end{smallmatrix}\bigr)$,
noting that $\gamma^0$ can be defined in the same way since
$\sigma^0 = \bar{\sigma}^0 = I_{2 \times 2}$.
Therefore, a Lagrangian
$\mathcal{L} = a \psi^\dagger \gamma_0 \psi + b \psi^\dagger \gamma_0 \gamma^\mu \partial_\mu \psi$
can be written, where $a, b \in \mathbb{C}$.
Defining a shortcut $\bar{\psi} \equiv \psi^\dagger \gamma_0$,
and requiring that the Lagrangian yields the familiar Dirac equation
$(i \gamma_\mu \partial^\mu - m)\psi = 0$ to set $a = -m$ and $ b = i$,
one arrives at the Dirac Lagrangian which describes free spin $\frac{1}{2}$
particles of mass $m$:

\begin{equation}
    \mathcal{L}_\text{Dirac} = \bar{\psi} (i\gamma^\mu\partial_\mu - m) \psi
\end{equation}

The Lagrangian is invariant under a global transformation
$\psi \rightarrow e^{i\alpha} \psi$ where $\alpha \in \mathbb{R}$.
Promoting $\alpha$ from a number to a function $\alpha(x^\mu)$
makes the transformation $\psi \rightarrow e^{i\alpha(x^\mu)} \psi$ local.
All such transformations form a $U(1)$ group.
However, it is checked that $\mathcal{L}_\text{Dirac}$ is not invariant
under such local transformations due to the derivative term $\partial_\mu$.
To enforce such invariance,
A vector field operator $A_\mu$,
which satisfies the gauge transformation
$A_\mu \rightarrow A_\mu + \frac{1}{g} \partial_\mu \alpha(x^\mu)$,
is introduced by replacing the standard derivative $\partial_\mu$ with the
covariant derivative, defined as:

\begin{equation}
    \fsl{D}_\mu \equiv \partial_\mu - i g A_\mu
\end{equation}
at this point, the Lagrangian reads
$\mathcal{L} = \mathcal{L}_\text{Dirac} + g A_\mu\bar{\psi}\gamma^\mu\psi$.
However, it is ``unstable''\footnote{
    This is not a standard terminology.
} as the equation of motion of $A_\mu$
suggests the interaction term $g A_\mu\bar{\psi}\gamma^\mu\psi = 0$,
because $0 = \partial_{A_\mu} \mathcal{L} = g \bar{\psi}\gamma^\mu\psi$.
Additional kinetic terms for spin 1 particles are needed to ``stabilize'' the
Lagrangian,
which is found by doing a similar exercise as in the spin $\frac{1}{2}$ case.
Such terms read $\frac{1}{4}F_{\mu\nu}F^{\mu\nu}$,
where $F^{\mu\nu} = \partial^\mu A^\nu - \partial^\nu A^\mu$.
Putting all pieces together, one obtains the Lagrangian for QED:

\begin{equation}
    \mathcal{L}_\text{QED} = \bar{\psi} (i\gamma^\mu\fsl{D}_\mu - m) \psi - \frac{1}{4}F_{\mu\nu}F^{\mu\nu}
\end{equation}


\subsection{The electroweak theory}
\label{ew-th}

As demonstrated in \cref{qed}, local gauge invariance principle provides a
method of introducing interactions to free theories.
Motivated by the experimental observation that in beta decay, electrons and
electron neutrinos are produced in pairs,
one can add two copies of spin $\frac{1}{2}$ particles,
labelled as $\psi_e, \psi_{\neu_e}$,
to the Lagrangian\footnote{
    The Lagrangian reads
    $\mathcal{L} = \bar{\Psi} (i \gamma^\mu \fsl{D}_\mu I_{2 \times 2} - M)\Psi -
    \frac{1}{4} F_{\mu\nu}F^{\mu\nu}$,
    where $M = \bigl(\begin{smallmatrix} m_e & 0 \\ 0 & m_{\neu_e} \end{smallmatrix}\bigr)$.
    Note that the identity matrix is often omitted.
} and write both in a doublet
of the form
$\Psi = \bigl(\begin{smallmatrix} \psi_e \\ \psi_{\neu_e} \end{smallmatrix}\bigr)$,
such that one particle can be rotated into the other via a $SU(2)$ rotation
$e^{i \alpha_i \frac{\sigma^i}{2}}$.

It is then natural to postulate a theory of $SU(2) \times U(1)$ local gauge
symmetry.
Following a similar route as in QED to ensure gauge invariance,
three spin 1 particles $W^1, W^2, W^3$ are added to the theory in addition to
$B$ which is the only generator of $U(1)$,
because the $SU(2)$ group has three generators.
After taking the fact that the three generators of $SU(2)$ do not commute into
account, a Lagrangian with correct kinetic terms can be constructed, in which
the covariant derivative is defined as:

\begin{equation}
    \fsl{D}_\mu = \partial_\mu - i[g_2 W_\mu^a T^a_{SU(2)} + g_1 B_\mu Y_{U(1)}]
\end{equation}
where $T^a_{SU(2)} = \frac{1}{2} \sigma^a$ are the generators of $SU(2)$,
and
$Y_{U(1)} = C \bigl(\begin{smallmatrix} 1 & 0 \\ 0 & 1 \end{smallmatrix}\bigr)$.
However, naive mass terms $\bar{\Psi} M \Psi$ are forbidden because they
spoil $SU(2)$ gauge symmetry\footnote{
    A $SU(2)$ transformation $e^{i \alpha_i \frac{\sigma^i}{2}}$ transforms such
    terms into
    $\bar{\Psi} e^{-i \alpha_i \frac{\sigma^i}{2}} M e^{i \alpha_i \frac{\sigma^i}{2}} \Psi$
    where $e^{-i \alpha_i \frac{\sigma^i}{2}} M e^{i \alpha_i \frac{\sigma^i}{2}} \neq M$
    because $[\sigma^i, M] \neq 0$ in general as $M \neq m I_{2 \times 2}$.
}.
So the theory constructed above requires \emph{all} particles to be massless,
which is inconsistent with the experimental results of massive force carrying
bosons in the weak interaction\footnote{
    If $W^1, W^2, W^3$ are all massless, the weak interaction would behave like
    electromagnetism, which it does not.
}, nor a massive electron.

To introduce \emph{gauge invariant} mass-like terms,
a complex scalar field is introduced with a potential
$V(\phi^\dagger, \phi) = \frac{1}{4}\lambda \left(\phi^\dagger \phi - \frac{1}{2}v^2\right)^2$,
which leads to a non-zero vacuum expectation value (VEV):
$\bra{0}\phi\ket{0} = v$ for $\phi$ degenerate on a complex circle.
Write the complex scalar fields as a $SU(2)$ doublet
$\Phi = \bigl(\begin{smallmatrix} \phi^\dagger \\ \phi^{\hphantom{\dagger}} \end{smallmatrix}\bigr)$,
then pick a particular vacuum and expand around it such that
$\Phi = \frac{1}{\sqrt{2}} \bigl(\begin{smallmatrix} v + h \\ 0 \end{smallmatrix}\bigr)$
where $h(x^\mu)$ is a real scalar field known as the Higgs field.
Doing so breaks the degeneracy (symmetry) of the vacuum and is called a
spontaneous symmetry breaking.
The couplings between $\Phi$ and the spin 1 gauge fields are embedded in the
kinetic term $\fsl{D}_\mu \bar\Phi \fsl{D}^\mu \Phi$,
and a non-zero VEV leads to terms of the form $\frac{1}{2}\kappa Z_\mu Z^\mu$ which
can be interpreted as mass terms.
After a field redefinition, one finds three massive spin 1 fields and a massless
one:

\begin{align}
    W^+_\mu &= \frac{1}{\sqrt{2}}(W^1_\mu - i W^2_\mu) \\
    W^-_\mu &= \frac{1}{\sqrt{2}}(W^1_\mu + i W^2_\mu) \\
    Z_\mu &= \cos\theta_w W^3_\mu - \sin\theta_w B_\mu \\
    A_\mu &= \sin\theta_w W^3_\mu + \cos\theta_w B_\mu
\end{align}
where $\theta_w = \tan^{-1}\left(\frac{g_1}{g_2}\right)$,
and $m_{W^\pm} = \frac{g_2 v}{2}$, $m_Z = \frac{m_{W^+}}{\cos\theta_w}$,
$m_A = 0$.
The procedure above is referred as the Higgs mechanism.

Still, the spin $\frac{1}{2}$ particles are massless.
Furthermore, the weak interaction is experimentally shown to maximally violate
parity,
that is, only the left-handed particles interact weakly.
Formally speaking, only the left-handed particles form $SU(2)$ doublets
which transform as
$\Psi_L \rightarrow e^{i \alpha_i \frac{\sigma^i}{2}} \Psi_L$;
the right-handed particles are $SU(2)$ singlets which do not transform
$\psi_R \rightarrow \psi_R$.
As a side note, previously it is argued that $SU(2)$ gauge symmetry forbids
terms like $\bar{\Psi} M \Psi$ because $M \neq m I_{2 \times 2}$ in
general,
(e.g. electron and electron neutrinos have \emph{very} different masses).
Parity violation provides a different view point on the same issue:
naive mass terms contain \emph{only} combinations of left-handed doublets
and right-handed singlets, such as $\bar{\Psi}_L \psi_R$.
Inspecting the $SU(2)$ transformation property of such terms reveals that
they are not $SU(2)$ invariant thus forbidden:
$\bar{\Psi}_L \psi_R \rightarrow \bar{\Psi}_L e^{-i \alpha_i \frac{\sigma^i}{2}} \psi_R \neq \bar{\Psi}_L \psi_R$.

To introduce mass to spin $\frac{1}{2}$ particles, Yukawa terms
$\lambda_\text{Yuk} \bar{\Psi}_L \Phi \psi_R$,
which can be readily checked\footnote{
    $U(1)$:
    $\bar{\Psi}_L \Phi \psi_R \rightarrow
    (\bar{\Psi}_L e^{-i \alpha}) \Phi (e^{i \alpha} \psi_R)
    = \bar{\Psi}_L \Phi \psi_R$;
    $SU(2)$:
    $\bar{\Psi}_L \Phi \psi_R \rightarrow
    (\bar{\Psi}_L e^{-i \alpha_i \frac{\sigma^i}{2}}) (e^{i \alpha_i \frac{\sigma^i}{2}} \Phi) \psi_R
    = \bar{\Psi}_L \Phi \psi_R$.
} to be Lorentz, $U(1)$, and $SU(2)$ invariant, are added to the Lagrangian.
Through the familiar Higgs mechanism, which has a non-zero VEV for $\Phi$,
terms such as the electron mass term
$\frac{\lambda_e v}{\sqrt{2}}({\psi^\dagger_{L,e} \psi_{R,e} + \psi^\dagger_{R,e}\psi_{L,e}})$
are generated.


\subsection{Lepton flavor universality in the SM}
\label{ref:theory:lfu}

With the electroweak theory developed in \cref{ew-th},
and accepting the experimental observation that leptons do not interact strongly
(i.e. they are $SU(3)$ singlets),
it is now possible to discuss lepton flavor universality:
the SM permits addition of arbitrary generations of spin $\frac{1}{2}$ color
singlets (leptons), and each generation couples to all but the Higgs field $h$
with the same strengths through the same covariant derivative term.
The LFU is manifested by the very construction of the SM.
They do, however, couple to the Higgs differently in the Yukawa sector which
are realized by their different masses.

The three flavor generations of leptons are purely determined by experiments,
as the SM places no constraint on number of flavor generations.


\section{Semileptonic $B$ decays in the SM}

The SM provides predictions on the differential decay rates of the
semileptonic $B$ decays such as $B \rightarrow D^{(*)}\ell\neulb$.
This analysis has an extensive use of MC simulations
which require inputs of such decay rates in order to model the corresponding
decay processes.
It is therefore very important to compute the decay rates accurately.


\section{Form factors in $B \rightarrow D^{(*)}\ell\neulb$ decays}
\label{ref:theory:ff-d0-dst}




\section{Computation of \RD and \RDst}
\label{ref:theory:rdx}


\section{Form factors in $B \rightarrow D^{**}\ell\neulb$ decays}
\label{ref:theory:ff-dstst}
