% Potentially useful sources:
%   https://physics.stackexchange.com/questions/707380/is-lepton-flavour-universality-an-accidental-symmetry-of-the-standatd-model
%   https://physics.stackexchange.com/questions/672626/what-is-an-accidental-symmetry
%   https://physics.stackexchange.com/questions/55350/two-ways-to-form-su2-singlets
%   https://physics.stackexchange.com/questions/607444/chiral-symmetry
%   https://physics.stackexchange.com/questions/429180/how-does-bhabha-scattering-imply-the-existence-of-quark-lepton-substructure
%   https://physics.stackexchange.com/questions/96362/what-does-a-rm-su2-isospin-doublet-really-mean
%   https://physics.stackexchange.com/questions/385159/quark-gluon-color-relationship-in-pure-qcd

% Schwartz
%   p. 585 (hypercharge)


\chapter{Theory of semileptonic $B$ decays}
\label{ref:theory}

Through a close collaboration between experimentalists and theorists,
particle physicists were able to come up with a \emph{quantum field theory}
to describe \emph{interactions} of \emph{elementary particles}
(all slanted terms will be defined more precisely later).
Schematically, such a theory can be constructed as following:

\begin{enumerate}
    \item Assume special relativity holds so that the Lagrangian density
        $\mathcal{L}$
        must be a Lorentz scalar (Lorentz invariant) and the
        energy-momentum follows a relativistic relation:

        \begin{equation}
            E^2 = m^2 + \vec{p} \cdot \vec{p}
            \label{eqn:e-p-dispersion}
        \end{equation}

    \item Assume the time evolution of any quantum state $| \rangle$ is governed
        by the Schrödinger's equation:

        \begin{equation}
            i \frac{\partial}{\partial t} | \rangle = H | \rangle
            \label{eqn:schodinger}
        \end{equation}

    \item The relativistic energy-momentum relation \cref{eqn:e-p-dispersion}
        implies the Klein-Gordon equation:

        \begin{equation}
            (\partial_\mu \partial^\mu + m^2) \phi = 0
            \label{eqn:k-g}
        \end{equation}
        which solves the squared version of \cref{eqn:schodinger},
        and its solutions contain operators to create and destroy \emph{scalar}
        particles on $| \rangle$.

        It is then possible to construct a Lagrangian density\footnote{
            Starting from here, to be more concise, ``Lagrangian'' is used
            in place of ``Lagrangian density''.
            Same for ``Hamiltonian'' in place of ``Hamiltonian density''.
        } $\mathcal{L}_0$
        that leads to \cref{eqn:k-g} by the Euler-Lagrange equation:
        \begin{equation}
            \mathcal{L}_0 = \frac{1}{2} \partial_\mu \partial^\mu \phi -
                \frac{1}{2} m \phi^2
        \end{equation}
        where the subscript 0 indicates that the Lagrangian density is for a
        spin-0 scalar particle.

    \item The Dirac equation solves the relativistic version of
        \cref{eqn:schodinger},
        noting that $\gamma^\mu$ are $4 \times 4$ matrices constructed from
        $2 \times 2$ Pauli matrices\footnote{
            Clifford algebra.
        },
        representable with Chiral spinors,
        this suggests that there are both left-handed and right-handed particles,
        each with a spin of $\frac{1}{2}$.

        The corresponding Lagrangian density $\mathcal{L}_{1/2}$ is:

    \item A free quantum theory Lagrangian can be constructed from ,
        it is possible to use complex scalars in .
        A free Hamiltonian density can be constructed by from the free
        Lagrangian density via the Legendre transformation

    \item The free theory has the wrong phenomenology because it doesn't permit
        interactions between different types of particles which is required by
        experimental results.
        Suppose additional interaction terms are added,
        the transition amplitude can be found by going into the interaction
        picture, where the states evolve by the interaction Hamiltonian
        and the operator fields by the free Hamiltonian.
        It is possible to express the time evolution of a state with a Dyson
        series which contains a time-ordering operator.
        Each term can be expanded into a set of normal-ordered integrals with
        the Wick's theorem.
        The normal ordering makes the bra-ket computable because it contains
        propagators which are scalar-valued functions.

    \item Interactions can be put into the free Lagrangian by requiring the
        Lagrangian remains invariant under a local gauge transformation:

        Note that the naive mass terms need to be dropped because they violate
        $U(1)$ gauge symmetry.

        The ``mass'' of each particle, which can be a gauge boson or a fermion,
        is put back by the Higgs mechanism,
        which is introduced by a gauge fixing due to a non-0 vacuum expectation
        value of the Higgs field (a scalar doublet).
\end{enumerate}

The focus is on the eletroweak part which will be briefly discussed to show LFU,
before moving on to a description of form factor which is crucial to compute
transition amplitude.
then the expressions for \RDX are given.
Finally, a brief discussion on the form factors characterizing the $1P$ state
$D$ meson transition is provided.


\section{Lepton universality in the SM}
\label{ref:theory:lfu}


\section{Form factors in $B \rightarrow D^{(*)}\ell\neulb$ decays}
\label{ref:theory:ff-d0-dst}


\section{Computation of \RD and \RDst}
\label{ref:theory:rdx}


\section{Form factors in $B \rightarrow D^{**}\ell\neulb$ decays}
\label{ref:theory:ff-dstst}
