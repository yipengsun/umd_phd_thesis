% Potentially useful sources:
%   https://physics.stackexchange.com/questions/707380/is-lepton-flavour-universality-an-accidental-symmetry-of-the-standatd-model
%   https://physics.stackexchange.com/questions/672626/what-is-an-accidental-symmetry
%   https://physics.stackexchange.com/questions/55350/two-ways-to-form-su2-singlets
%   https://physics.stackexchange.com/questions/607444/chiral-symmetry
%   https://physics.stackexchange.com/questions/429180/how-does-bhabha-scattering-imply-the-existence-of-quark-lepton-substructure
%   https://physics.stackexchange.com/questions/96362/what-does-a-rm-su2-isospin-doublet-really-mean
%   https://physics.stackexchange.com/questions/385159/quark-gluon-color-relationship-in-pure-qcd

% Schwartz
%   p. 585 (hypercharge)


\chapter{Theory of semileptonic $B$ decays}
\label{ref:theory}

Through a close collaboration between experimentalists and theorists,
particle physicists were able to come up with a \emph{quantum field theory}
to describe \emph{interactions} of \emph{elementary particles}
(all slanted terms will be defined more precisely later).
Schematically, such a theory can be constructed as following:

\begin{enumerate}
    \item Assume special relativity holds so that the Lagrangian density
        $\mathcal{L}$
        must be a Lorentz scalar (Lorentz invariant) and the
        energy-momentum follows a relativistic relation:

        \begin{equation}
            E^2 = m^2 + \vec{p} \cdot \vec{p}
            \label{eqn:e-p-dispersion}
        \end{equation}

    \item Assume the time evolution of a quantum state is governed by the
        Schrödinger's equation:

        \begin{equation}
            i \frac{\partial}{\partial t} | \rangle = H | \rangle
            \label{eqn:schodinger}
        \end{equation}
        where $| \rangle$ is an arbitrary quantum state.

    \item It can be shown that the relativistic energy-momentum relation
        \cref{eqn:e-p-dispersion} leads to the Klein-Gordon equation:

        \begin{equation}
            (\partial_\mu \partial^\mu + m) \phi = 0
        \end{equation}
        which solves the squared version of \cref{eqn:schodinger},
        and its solutions contain operators to create and destroy \emph{scalar}
        particles on states $| \rangle$.

        It is then possible to construct a Lagrangian density $\mathcal{L}_0$:
        \begin{equation}
            \mathcal{L}_0 = \frac{1}{2} \partial_\mu \partial^\mu \phi -
                \frac{1}{2} m \phi^2
        \end{equation}
        where the subscript 0 indicates that the Lagrangian density is for a
        spin-0 scalar particle.
\end{enumerate}


\section{Lepton universality in the SM}
\label{ref:theory:lfu}


\section{Form factors in $B \rightarrow D^{(*)}\ell\neulb$ decays}
\label{ref:theory:ff-d0-dst}


\section{Computation of \RD and \RDst}
\label{ref:theory:rdx}


\section{Form factors in $B \rightarrow D^{**}\ell\neulb$ decays}
\label{ref:theory:ff-dstst}
