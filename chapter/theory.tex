% Potentially useful sources:
%   https://physics.stackexchange.com/questions/707380/is-lepton-flavour-universality-an-accidental-symmetry-of-the-standatd-model
%   https://physics.stackexchange.com/questions/672626/what-is-an-accidental-symmetry
%   https://physics.stackexchange.com/questions/55350/two-ways-to-form-su2-singlets
%   https://physics.stackexchange.com/questions/607444/chiral-symmetry
%   https://physics.stackexchange.com/questions/429180/how-does-bhabha-scattering-imply-the-existence-of-quark-lepton-substructure
%   https://physics.stackexchange.com/questions/96362/what-does-a-rm-su2-isospin-doublet-really-mean
%   https://physics.stackexchange.com/questions/385159/quark-gluon-color-relationship-in-pure-qcd

% Schwartz
%   p. 585 (hypercharge)


\chapter{Theory of semileptonic $B$ decays}
\label{ref:theory}

\section{Lepton flavor universality in the SM}
\label{ref:theory:lfu}

Through a close collaboration between experimentalists and theorists,
particle physicists were able to come up with a \emph{quantum field theory}
to describe \emph{interactions} of \emph{elementary particles},
known as the standard model of particle physics (SM).
In this section a construction of the SM is outlined following
\cite{Robinson_2011}
and its implications on lepton flavor universality are discussed.

To construct the SM,
in general one starts with a free Lagrangian density\footnote{
    Will be referred as just ``Lagrangian'' in the following text for brevity,
    same for ``Hamiltonian''.
},
then demands the Lagrangian is invariant under local gauge transformation of
certain Lie groups to arrive at an Lagrangian with interactions,
adding appropriate terms when necessary.
Then it is (sometimes) possible to compute the transition amplitude from an
initial state to a final state in a perturbative manner\footnote{
    To be more precise, the calculation can be carried out in the following
    manner:
    \begin{enumerate}[noitemsep,nosep]
        \item Obtain a Hamiltonian $\mathcal{H}$ from the Lagrangian
            $\mathcal{L}$ with the Legendre transformation.
        \item Identify the free part and the interaction part of the Hamiltonian,
            denote as $\mathcal{H}_F, \mathcal{H}_I$, respectively.
        \item Assume the time evolution of a state $\ket{i}$ is governed by the
            Schrödinger equation $i \partial_t \ket{i} = H_I \ket{i}$
            in the interaction picture.
        \item Express the scattering process as $\bra{f} S \ket{i}$,
            the scattering operator $S$ can be identified as
            $S = \exp({i \int_{-\infty}^\infty \mathcal{H}_I d^4 x})$.
        \item The $S$ operator can be expand in a Taylor-like series:

            $S = \sum_n \frac{(-i)^n}{n!} T\left\{
                \int_{-\infty}^{\infty} \cdots \int_{-\infty}^{\infty}
                \mathcal{H}_{I}^1 \cdots \mathcal{H}_{I}^n d^4 x_1 \cdots d^4 x_n
            \right\}$,
            where $T$ stands for time-ordering.
        \item Each term in the series can be expressed by the Wick's theorem
            as a series of normal-ordered terms with contractions.
            The contractions can be identified as the Feynman propagators with
            the canonical commutation relations.
        \item The first non-vanish order terms are identified as tree-level
            Feynman diagrams and can be computed.
    \end{enumerate}
    Higher-order terms may involve infinities and require renormalization,
    with is not the focus of this text.
}.

by going into the
interaction picture\footnote{
    In the interaction picture the time evolution for operators are governed by
    the free Hamiltonian, whereas the states by the Interaction Hamiltonian.
}, identify the interaction Hamiltonian, and use the $S$-matrix formalism to
expand the $S$-matrix into a series of

\begin{enumerate}
    \item Assume special relativity holds so that the Lagrangian density
        $\mathcal{L}$
        must be a Lorentz scalar (Lorentz invariant) and the
        energy-momentum follows a relativistic relation:

        \begin{equation}
            E^2 = m^2 + \vec{p} \cdot \vec{p}
            \label{eqn:e-p-dispersion}
        \end{equation}

        Additionally,
        interpret the Schrödinger's equation as a description to the time
        evolution of a spin-0 scalar field $\phi$:

        \begin{equation}
            i \partial t \phi = H \phi
            \label{eqn:schodinger}
        \end{equation}

    \item Replacing $E^2$ and $\vec{p} \cdot \vec{p}$
        with the corresponding operators in
        \cref{eqn:e-p-dispersion}
        then inserting into the squared version of
        \cref{eqn:schodinger},
        we have the Klein-Gordon equation\footnote{
            The metric $\eta_{\mu\nu}$ follows particle physicists' convention
            $(+ - - -)$.
        }:

        \begin{equation}
            (\partial_\mu \partial^\mu + m^2) \phi = 0
            \label{eqn:k-g}
        \end{equation}

        For a spin $1/2$ particle,
        we can also construct a linear equation for a $n$-component spinor
        $\psi$, the Dirac equation:

        \begin{equation}
            (i \gamma^\mu \partial_\mu - m I) \psi = 0
            \label{eqn:dirac}
        \end{equation}
        where $\gamma^\mu$ operates on the spinor space, and $I$ refers to the
        identity operator in the same spinor space which will be omitted in
        later text for brevity.

        Requiring the squared version of the operator
        $(i \gamma^\mu \partial_\mu - m)$ to recover \cref{eqn:k-g} when
        operating\footnote{
            Note that $i \gamma^\mu \partial_\mu \psi = m \psi$,
            so there will be no cross term.
        } on $\psi$:
        \begin{equation}
            (i \gamma^\mu \partial_\mu - m)
            (i \gamma^\nu \partial_\nu - m) \psi =
            (\gamma^\mu\gamma^\nu \partial_\nu \partial_\mu + m^2) \psi = 0
        \end{equation}
        we find a relation between $\gamma^\mu, \gamma^\nu$:
        \begin{equation}
            \left\{ \gamma^\mu, \gamma^\nu \right\} = 2 \eta^{\mu\nu}
        \end{equation}
        which are $4 \times 4$ matrices constructable from $2 \times 2$ Pauli
        matrices.


        It is then possible to construct a Lagrangian density\footnote{
            Starting from here, to be more concise, ``Lagrangian'' is used
            in place of ``Lagrangian density''.
            Same for ``Hamiltonian'' in place of ``Hamiltonian density''.
        } $\mathcal{L}_\text{free}$
        that leads to the correct dynamical equations
        (i.e. Klein-Gordon and Dirac)
        by the Euler-Lagrange equation\footnote{
            That is:
            $\frac{\partial \mathcal{L}}{\partial \phi} - \partial_\mu (\frac{\partial \mathcal{L}}{\partial (\partial_\mu \phi)}) = 0$.
        }:

        \begin{equation}
            \mathcal{L}_\text{free} =
                \frac{1}{2} \partial_\mu \phi \partial^\mu \phi -
                \frac{1}{2} m \phi^2
        \end{equation}

    \item The Dirac equation solves the relativistic version of
        \cref{eqn:schodinger},
        noting that $\gamma^\mu$ are $4 \times 4$ matrices constructed from
        $2 \times 2$ Pauli matrices\footnote{
            Clifford algebra.
        },
        representable with Chiral spinors,
        this suggests that there are both left-handed and right-handed particles,
        each with a spin of $\frac{1}{2}$.

        The corresponding Lagrangian density $\mathcal{L}_{1/2}$ is:

    \item A free quantum theory Lagrangian can be constructed from ,
        it is possible to use complex scalars in .
        A free Hamiltonian density can be constructed by from the free
        Lagrangian density via the Legendre transformation

    \item The free theory has the wrong phenomenology because it doesn't permit
        interactions between different types of particles which is required by
        experimental results.
        Suppose additional interaction terms are added,
        the transition amplitude can be found by going into the interaction
        picture, where the states evolve by the interaction Hamiltonian
        and the operator fields by the free Hamiltonian.
        It is possible to express the time evolution of a state with a Dyson
        series which contains a time-ordering operator.
        Each term can be expanded into a set of normal-ordered integrals with
        the Wick's theorem.
        The normal ordering makes the bra-ket computable because it contains
        propagators which are scalar-valued functions.

    \item Interactions can be put into the free Lagrangian by requiring the
        Lagrangian remains invariant under a local gauge transformation:

        Note that the naive mass terms need to be dropped because they violate
        $U(1)$ gauge symmetry.

        The ``mass'' of each particle, which can be a gauge boson or a fermion,
        is put back by the Higgs mechanism,
        which is introduced by a gauge fixing due to a non-0 vacuum expectation
        value of the Higgs field (a scalar doublet).
\end{enumerate}

The focus is on the eletroweak part which will be briefly discussed to show LFU,
before moving on to a description of form factor which is crucial to compute
transition amplitude.
then the expressions for \RDX are given.
Finally, a brief discussion on the form factors characterizing the $1P$ state
$D$ meson transition is provided.




\section{Form factors in $B \rightarrow D^{(*)}\ell\neulb$ decays}
\label{ref:theory:ff-d0-dst}


\section{Computation of \RD and \RDst}
\label{ref:theory:rdx}


\section{Form factors in $B \rightarrow D^{**}\ell\neulb$ decays}
\label{ref:theory:ff-dstst}
