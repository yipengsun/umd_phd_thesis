\chapter{Correction and validation of the MC simulation}
\label{ref:mc-cor}

This chapter describes corrections to MC,
mainly to correct the modelling of form factor and detector responses.
The MC correction procedure is the following:
%%%%
First, the outdated form factor modellings are updated and known
problematic responses, such as tracking efficiency and \B production kinematics,
are corrected.
This is termed as \emph{initial reweighting}, and is described in
\cref{sec:data-mc:init-rwt}.
%%%%
Then, a \emph{preliminary} fit is performed to build
\emph{post-fit cocktail},
with the building procedure outlined in \cref{sec:data-mc:postfit-cocktail}.
Finally, agreements between data and cocktail on additional kinematic
and geometric variables of the \B decay daughters are assessed,
and a multi-stage reweighting process is
performed to correct disagreements if necessary.
Outlined in \cref{sec:data-mc:final-rwt}, such procedure is called \emph{final
reweighting}.


\section{Initial reweighting}
\label{sec:data-mc:init-rwt}

\subsection{Form factor model}

The form factor model for a $B \rightarrow D \ell\neulb$ \emph{defines} the
truth-level \qSq distribution (with angular variables integrated out) which
in turn affect the estimated \qSq, a fit variable.
Therefore, it is important to use the state-of-art form factor models.
Naively this is hard, because the form model model is defined at MC generator
level, any change in form factor may require a regeneration of all MC samples.
In reality, however, for a given phase space point\footnote{
    The phase space is parameterized by \qSq and possibly angular variables,
    as defined in \cref{ref:theory:ff}.
}  \PSpt, the probability density that an event is generated \emph{at}
\PSpt is proportional to the differential decay rate evaluated at
\PSpt.
%%%%
For an event (or equivalently a given \PSpt point)
generated with an old form factor model,
its contribution (weight) in the new model is given by the ratio of differential
decay rates
\begin{equation}
    r = \left.
            \frac{d\Gamma_\text{new} / d\PSpt}{d\Gamma_\text{old} / d\PSpt}
        \right|_\text{eval at phase space point}
\end{equation}
So a change in form factor model is equivalently to a reweight, with weight given
by $r$.

\Hammer, the current state-of-art form factor reweighting program, is used
in this analysis to reweight
$B \rightarrow (\Dz| \Dstar| D^{**}) \lepton\neub$ MC samples.
The details are given below.

\paragraph{\boldmath{\Dz}} The \Dz MC samples are generated with an up-to-date
CLN form factor model.
Still, it is reweighted to a BGL model whose parameters are taken
from Table 4 in \cite{Bigi_2016} at $N = 2$, as these
parameters are obtained from a fit to lattice and experimental data.
The $N = 2$ is chosen because it is the only case where the covariance matrix
is reported.
\Cref{fig:ff-d0} shows the shift of true \qSq distributions due to form factor
reweighting for \Dz samples.
The changes are minimal because both models are up-to-date.

% Generated in /rdx-run2-analysis with:
%   make plot-all-ff-rwt
\begin{figure}[htb]
    \centering
    \includegraphics[width=0.45\textwidth]{
        ./figs-mc-correction/reweighting-form-factors/norm/D0Mu.pdf
    }
    \includegraphics[width=0.45\textwidth]{
        ./figs-mc-correction/reweighting-form-factors/sig/D0Tau.pdf
    }
    \caption{
        Form factor reweight for $B \rightarrow \Dz \lepton\neub$ MC samples.
    }
    \label{fig:ff-d0}
\end{figure}

\paragraph{\boldmath{\Dstar}} The \Dstar MC samples are reweighted from a CLN
to a BGL model with parameters taken from the right-most column of Table XII
in the \textbf{v1} version of \cite{Bazavov_2021},
the first work to calculate \Dstar form factors at non-zero recoil
based on lattice-QCD with constraints from external measurements.

The $c_3$ and $d_2$ parameters are not implemented in \Hammer (yet).
Given the large uncertainties on these sub-dominate parameters,
they are set to 0 with the corresponding entries in the covariance matrix
removed.

\begin{figure}[htb]
    \centering
    \includegraphics[width=0.45\textwidth]{
        ./figs-mc-correction/reweighting-form-factors/norm/DstMu.pdf
    }
    \includegraphics[width=0.45\textwidth]{
        ./figs-mc-correction/reweighting-form-factors/norm/Dst0Mu.pdf
    }

    \includegraphics[width=0.45\textwidth]{
        ./figs-mc-correction/reweighting-form-factors/sig/DstTau.pdf
    }
    \includegraphics[width=0.45\textwidth]{
        ./figs-mc-correction/reweighting-form-factors/sig/Dst0Tau.pdf
    }
    \caption{
        Form factor reweight for $B \rightarrow \D^* \lepton\neub$ MC samples.
    }
    \label{fig:ff-d0}
\end{figure}


\subsection{Tracking efficiency}

The track reconstruction efficiencies,
defined as $\frac{N_\text{reconstructed}}{N_\text{reconstructable}}$ for
\emph{long tracks}\footnote{
    Long tracks means that the tracks have hits recorded at both VELO (upstream)
    and T stations (downstream).
    They have the best tracking quality and most analyses use them exclusively.
},
are known to be different between data and MC.
The efficiencies are measured with a \emph{tag-and-probe} method on
\jpsi\mup\mun samples
which involves the following steps:
First, a long \muon-like track (tag) is and a partially reconstructed \muon-like
track (probe, non-long track) is required to form a \mun\mup vertex.
Then, the probe track is classified into two categories: The ones matched to a
long track, defined by sharing 70\% of hits with a long track in the T stations;
the ones \emph{not} matched.
Finally, a fit on the invariant mass of the \mun\mup vertex is performed to
extract number of signal events $N_\text{sig}$,
separately for matched and all probe tracks,
and the tracking efficiency is computed as:

\begin{equation}
    \epsilon_\text{tracking} \equiv
        \frac{N_\text{reconstructed}}{N_\text{reconstructable}}
        = \frac{N_\text{sig,matched}}{N_\text{sig,all}}
\end{equation}
For more information, refer to \cite{LHCb-PUB-2011-025,LHCb-DP-2013-002}.

To correct for the tracking discrepancy between data and MC,
the efficiency ratios\footnote{
    The efficiency ratios are taken from
    \techurllink{https://twiki.cern.ch/twiki/bin/viewauth/LHCbInternal/LHCbTrackingEfficiencies}{
        twiki/internal/LHCbTrackingEfficiencies}.
}
($\frac{\epsilon_\text{data}}{\epsilon_\text{MC}}$),
binned in $\eta$ and \ptot of the track,
between data and MC are provided by the \trackcalib package.
The correction is applied as a weight on a per-event basis.
The efficiency ratio table is shown in \cref{fig:trackcalib-eff}.

% Taken from RD+'s ANA
\begin{figure}[htb]
    \centering
    \includegraphics[width=0.6\textwidth]{./figs-mc-correction/reweighting-tracking/tracking_eff_2016.pdf}
    \caption{
        Tracking efficiency ratios
        $\frac{\epsilon_\text{data}}{\epsilon_\text{MC}}$.
        For 2016, only the table for the \emph{long} method is available.
    }
    \label{fig:trackcalib-eff}
\end{figure}

There are a few caveats: The efficiencies are available only for Sim09b MC
version, but in this analysis Sim09k is used.
The difference in simulation versions is ignored, given that this analysis
measures the \emph{ratio} of branching fractions, it is assumed that the
discrepancy would cancel in first order.
Additionally, the binning of \trackcalib efficiency ratio table,
shown in \cref{fig:trackcalib-slow-pi}, are more restricted than the selection
cuts.
Most notably in slow \pion, about 50\% of the tracks falls outside of the
binning range of the table for \Dstarp\mun MC decay mode.
To circumvent the problem, one method is to apply \trackcalib binning range cuts
on both data and MC, which leads to a significant loss in selection efficiency.
It is decided that in case a track falls outside of the binning range, the
efficiency ratio from its closest bin is applied.
It is hoped that additional discrepancy will be corrected in the
\emph{final reweighting} procedure.

% Generated in lhcb-ntuples-gen/studies/plot-RDX_spi_tracking_eff:
%   ./plot_spi_tracking.py
\begin{figure}[htb]
    \centering
    \includegraphics[width=0.7\textwidth]{./figs-mc-correction/reweighting-tracking/Dst_spi_p_eta.pdf}
    \caption{
        For slow \pion, about 50\% of tracks lies outside the binning region
        of the official \trackcalib binning range which is displayed by the
        dashed black box.
        The colored rectangles show number of events fall in each
        official \trackcalib bin.
        To avoid loss in selection efficiencies,
        correction ratios from nearby bin is applied.
        The study is done on $\Bzb \rightarrow \Dstarp \mun \neumb$ MC sample.
    }
    \label{fig:trackcalib-slow-pi}
\end{figure}


\subsection{$B$ kinematics and multiplicity}
\label{sec:data-mc:init-rwt:jpsi-k-correction}

Another known source of inconsistency between data and MC is the \B production
kinematics and multiplicity.
This is corrected by a separate set of
$\Bp \rightarrow \jpsi (\rightarrow \mun\mup) \Kp$
control samples, as described
in \cref{sec:reconstructed-samples:add-ctrl}, with both data and MC\footnote{
    In this case the MC is also \emph{stripped}, so the stripping line
    decisions exist.
}
sample filtered on the \smalltt{StrippingBetaSBu2JpsiKDetachedLine}.
Additional selections,
listed in \cref{tab:cut-jpsik},
are applied on the reconstructed sample, with PID
selections applied as weights (obtained from \pidcalib) for MC.
Tracking efficiency ratio corrections are also applied as weights
(obtained from \trackcalib) on MC samples.

An extended PID binning range is chosen to cover as much phase space of
\kaon and \muon as possible.
There may still be events outside the covering region, in which case the PID
weights from nearby bin is applied,
to avoid explicit cuts on the \jpsi\kaon data control sample.
The same strategy is applied on tracking weights.

\begin{table}[htb]
    \caption{Additional selections on $\jpsi K$ samples.}
    \label{tab:cut-jpsik}
    \centering
    \begin{tabular}{ c | rll}
        \toprule
        {\bf Particle}    & {\bf Variable}               & {\bf Cuts}               \\
        \midrule
        \kaon             & \pt                          & $> 500$ MeV              \\
                          & \PID{$K$}                    & $> 4$                    \\
        \midrule
        \mun, \mup        & \pt                          & $> 500$ MeV              \\
                          & \ipChiSq                     & $> 4$                    \\
                          & \PID{\muon}                  & $> 2$                    \\
        \midrule
        \mun\mup (\jpsi)  & $m_\text{mea}$\parnote{
            $m_\text{mea}$ refers to measured mass, which is the invariant
            mass given by the sum of daughters' four momenta,
            without any topological constraint.
            On the other hand, $m$ is given by a vertex fit,
            which is typically of better quality.
        }                                                & 3060--3140 MeV           \\
                          & \anyChiSq{FD}                & $> 25$                   \\
        \midrule
        \Bp               & $m$                          & 5150--5350 MeV           \\
                          & \anyChiSq{vertex}            & $< 18$                   \\
                          & \ipChiSq                     & $< 12$                   \\
                          & \DIRA                        & $> 0.9995$               \\
        \bottomrule
    \end{tabular}
    \begin{flushleft}
        \parnotes
    \end{flushleft}
\end{table}

A \sPlot\ is then performed on $J/\psi K$ data control sample to subtract
the combinatorial background, which is shown in \cref{fig:fit-JpsiK-data}.
The fit models consists of a double Gaussian plus a Crystal bell for signal,
and an exponential background.

% Generated in /lhcb-ntuples-gen/run2-JpsiK, with:
%   make fit-2016
% need to install zfit, which can be done via:
%   pip install -r ./requirements.txt
% in the same folder. And for now this probably doesn't work on macOS
\begin{figure}[htb]
    \centering
    \includegraphics[width=0.45\textwidth]{./figs-mc-correction/reweighting-JpsiK/fit-JpsiK/fit_final.pdf}
    \hspace{1em}
    \includegraphics[width=0.45\textwidth]{./figs-mc-correction/reweighting-JpsiK/fit-JpsiK/fit_final_log_scale.pdf}

    \caption{
        Fit on $J/\psi K$ with a \sPlot procedure.
        The signal is modelled by a double Gaussian plus a Crystal bell;
        the background an exponential.
        Left and right show the same plot, with left having a linear $y$ axis,
        and right a log $y$ axis.
    }
    \label{fig:fit-JpsiK-data}
\end{figure}

A multi-stage\footnote{
    ``Multi-stage'' means that for reweighting stage $i$,
    all previous $i-1$ weights are applied.
} reweighting is performed on \sWeight-ed data and MC
(both are normalized, so the reweighting only corrects shape differences),
with stages defined in \cref{tab:rwt-JpsiK},
and the results shown in \cref{fig:rwt-JpsiK}.
The occupancy (nTracks) has to be corrected first,
because the PID efficiency weights depend this variable, and the weights alter
the shapes of the kinematic distributions of the \B decay daughters, which in
turn affect the \B kinematic distributions themselves.
The reweighting binning ranges are chosen such that:
\begin{itemize}
    \item They cover most of signal ranges of the data after applying \sWeight.
    \item The MC have enough population within these binning
        ranges so that there is no or few bins without a event.
\end{itemize}

When applying the \jpsi\kaon-derived weights to the nominal MC sample,
for events outside the binning regions,
the weights from nearby bins are applied.
This is to avoid introduce explicit kinematic cuts on the \B meson,
which is hard to apply consistently due to the fact that for
$B \rightarrow \jpsi\kaon$ decays,
there is no missing particle in the decay chain thus the \B momentum is
reconstructed in a more complete way,
whereas for $B \rightarrow D^{(*)} \lepton\neulb$, the missing neutrino(s)
makes the reconstructed \B momentum less precise.
It is assumed that the final reweighting is able to correct the remaining
differences in \B kinematics and occupancy.

\begin{table}[htb]
    \centering
    \caption{
        Reweighting stages and binning schemes for \B kinematic and
        occupancy reweighting.
    }
    \label{tab:rwt-JpsiK}
    \begin{tabular}{ c | c | c | c | c }
        \toprule
        {\bf Variable 1}    & {\bf Binning 1}      & {\bf Variable 2}   & {\bf Binning 2}    & {\bf Figure}   \\
        \midrule
        \B PV ndf           & 20, 1--250           & nTracks            & 20, 0--450         & \cref{fig:rwt-JpsiK:stage1} \\
        \B $\eta$           & 9, 2--6              & \B \pt             & 20, 0--30 GeV      & \cref{fig:rwt-JpsiK:stage2} \\
        \bottomrule
    \end{tabular}
\end{table}

% Generated in /lhcb-ntuples-gen/studies/plot-JpsiK_kinematic_reweighting
% by running:
%   plot_JpsiK_reweighting.py
% in the specified folder
\begin{figure}[htb]
    \begin{subfigure}{\textwidth}
        \centering
        \includegraphics[width=0.45\textwidth]{./figs-mc-correction/reweighting-JpsiK/reweight-JpsiK/b_ownpv_ndof.pdf}
        \hspace{1em}
        \includegraphics[width=0.45\textwidth]{./figs-mc-correction/reweighting-JpsiK/reweight-JpsiK/ntracks.pdf}
        \caption{Stage-1: $B$ PV ndf and nTracks.}
        \label{fig:rwt-JpsiK:stage1}
    \end{subfigure}

    \begin{subfigure}{\textwidth}
        \centering
        \includegraphics[width=0.45\textwidth]{./figs-mc-correction/reweighting-JpsiK/reweight-JpsiK/b_eta.pdf}
        \hspace{1em}
        \includegraphics[width=0.45\textwidth]{./figs-mc-correction/reweighting-JpsiK/reweight-JpsiK/b_pt.pdf}
        \caption{Stage-2: $B$ $\eta$ and $B$ \pt.}
        \label{fig:rwt-JpsiK:stage2}
    \end{subfigure}

    \caption{
        Effect of \B kinematic and occupancy reweighting,
        derived from \jpsi\kaon control samples.
        The ``MC (after)'' plots contain corrections from \emph{all stages}.
    }
    \label{fig:rwt-JpsiK}
\end{figure}


\section{Post-fit cocktail}
\label{sec:data-mc:postfit-cocktail}

Selected data samples are themselves cocktails because many
decay modes contribute to the $D^{(*)}\mu$ final state.
Therefore, meaningful comparisons between data and MC can only be performed
after a fit (termed \emph{initial fit}).

After the initial weights are applied to the fit templates,
a multi-step fit,
termed \emph{initial fit},
which is similar to the \emph{canonical} fit described in
\cref{sec:fit-to-data:fit-procedure},
with the sole exception that in the step-2 control fit,
the restriction (constraint on the log-likelihood)
on $D^{**}$ shape due to form factor uncertainties is enabled,
is performed with these templates to determine the yield and variations of
each template.
The fit results are displayed in
\cref{appx:suppl:init-fit-cocktail}.

Since the main objective of final reweighting is to correct the remaining
inconsistencies between data and MC due to detector responses,
it is important to reweight in a region where most discrepancies are induced by
such responses.
Therefore, a $\mmSq < 0.4$ GeV region,
dominated in normalization decay mode,
mostly free of effects due to fit modelling,
is used to derive the final weights.

A separate Python script\footnote{
    This can be fount at \techurllink{https://github.com/umd-lhcb/rdx-run2-analysis/blob/master/scripts/make_variated_histos.py}{
        github/umd-lhcb/rdx-run2-analysis
    }.
} is then used to build the post-fit cocktail with the following procedure:
\begin{enumerate}
    \item Apply the $\mmSq < 0.4$ GeV cut to obtain yields in the reweighting
        region based on fitted total yields.

    \item Some fit templates have variations which are specified by
        additionally provide variation templates at $\pm 1 \alpha$ and a
        interpolation/extrapolation method.
        This is described in \cref{sec:fit-to-data:fit-tmpl-vars}.

        The variation procedure is implemented in the Python script,
        and is checked against the variated templates extracted from the
        fitter output directly.
        For each template, the externally and internally variated ones differ
        bin-by-bin by less than $10^{-3}$.

    \item After effecting possible variations,
        each fit template is rescaled such that the scaled integral of the
        template matches the fitted yield.

    \item All MC templates are added together to form the
        \emph{post-fit cocktail}.
        The data-driven templates are subtracted from the \emph{data}
        template to form \emph{subtracted data} template
        which is comparable to the \emph{post-fit cocktail}.
\end{enumerate}


\section{Final reweighting}
\label{sec:data-mc:final-rwt}

The final reweighting is a multi-stage reweighting process by comparing
selected
kinematic and geometric variables of the \B decay daughters,
listed in \cref{tab:rwt-final-vars},
between \emph{post-fit cocktail} and \emph{subtracted data}
in the reweighting region, selected by a $\mmSq < 0.4$ GeV cut.
The reweighting is performed separately for \Dz and \Dstar channel;
the latter contains additional stages due to corrections to slow \pion.

The effects of final reweighting can be seen at
\cref{fig:final-rwt-d0} for \Dz channel, and
\cref{fig:final-rwt-dst} for \Dstar channel.

\begin{figure}[htb]
    \begin{subfigure}{\textwidth}
        \centering
        \includegraphics[width=0.32\textwidth]{./figs-mc-correction/reweighting-final/plot_step1-D0_iso-b_log_fd_chi2.pdf}
        \includegraphics[width=0.32\textwidth]{./figs-mc-correction/reweighting-final/plot_step1-D0_iso-d0_log_ip_chi2.pdf}
        \includegraphics[width=0.32\textwidth]{./figs-mc-correction/reweighting-final/plot_step1-D0_iso-mu_log_ip_chi2.pdf}
        \caption{Stage 1.}
    \end{subfigure}

    \begin{subfigure}{\textwidth}
        \centering
        \includegraphics[width=0.32\textwidth]{./figs-mc-correction/reweighting-final/plot_step2-D0_iso-k_comp.pdf}
        \includegraphics[width=0.32\textwidth]{./figs-mc-correction/reweighting-final/plot_step2-D0_iso-k_log_ip_chi2.pdf}
        \includegraphics[width=0.32\textwidth]{./figs-mc-correction/reweighting-final/plot_step2-D0_iso-k_pt.pdf}
        \caption{Stage 2.}
    \end{subfigure}

    \begin{subfigure}{\textwidth}
        \centering
        \includegraphics[width=0.32\textwidth]{./figs-mc-correction/reweighting-final/plot_step3-D0_iso-pi_comp.pdf}
        \includegraphics[width=0.32\textwidth]{./figs-mc-correction/reweighting-final/plot_step3-D0_iso-pi_log_ip_chi2.pdf}
        \includegraphics[width=0.32\textwidth]{./figs-mc-correction/reweighting-final/plot_step3-D0_iso-pi_pt.pdf}
        \caption{Stage 3.}
    \end{subfigure}

    \begin{subfigure}{\textwidth}
        \centering
        \includegraphics[width=0.32\textwidth]{./figs-mc-correction/reweighting-final/plot_step4-D0_iso-mu_comp.pdf}
        \includegraphics[width=0.32\textwidth]{./figs-mc-correction/reweighting-final/plot_step4-D0_iso-mu_log_ip_chi2.pdf}
        \includegraphics[width=0.32\textwidth]{./figs-mc-correction/reweighting-final/plot_step4-D0_iso-mu_pt.pdf}
        \caption{Stage 4.}
    \end{subfigure}

    \caption{
        Final reweighting for \Dz channel, for a total of 10 stages.
        The black plots contain weights from \emph{all previous} stages,
        whereas the red ones contain weights from \emph{all} stages.
    }
    \label{fig:final-rwt-d0}
\end{figure}

\begin{figure}[htb]\ContinuedFloat
    \begin{subfigure}{\textwidth}
        \centering
        \includegraphics[width=0.32\textwidth]{./figs-mc-correction/reweighting-final/plot_step5-D0_iso-d0_comp.pdf}
        \includegraphics[width=0.32\textwidth]{./figs-mc-correction/reweighting-final/plot_step5-D0_iso-d0_log_ip_chi2.pdf}
        \includegraphics[width=0.32\textwidth]{./figs-mc-correction/reweighting-final/plot_step5-D0_iso-d0_pt.pdf}
        \caption{Stage 5.}
    \end{subfigure}

    \begin{subfigure}{\textwidth}
        \centering
        \includegraphics[width=0.32\textwidth]{./figs-mc-correction/reweighting-final/plot_step6-D0_iso-d0_eta.pdf}
        \includegraphics[width=0.32\textwidth]{./figs-mc-correction/reweighting-final/plot_step6-D0_iso-d0_pt.pdf}
        \caption{Stage 6.}
    \end{subfigure}

    \begin{subfigure}{\textwidth}
        \centering
        \includegraphics[width=0.32\textwidth]{./figs-mc-correction/reweighting-final/plot_step7-D0_iso-k_eta.pdf}
        \includegraphics[width=0.32\textwidth]{./figs-mc-correction/reweighting-final/plot_step7-D0_iso-k_pt.pdf}
        \caption{Stage 7.}
    \end{subfigure}

    \begin{subfigure}{\textwidth}
        \centering
        \includegraphics[width=0.32\textwidth]{./figs-mc-correction/reweighting-final/plot_step8-D0_iso-pi_eta.pdf}
        \includegraphics[width=0.32\textwidth]{./figs-mc-correction/reweighting-final/plot_step8-D0_iso-pi_pt.pdf}
        \caption{Stage 8.}
    \end{subfigure}

    \caption[]{Final reweighting for \Dz channel, stages 5--8 (cont'd).}
\end{figure}

\begin{figure}[htb]\ContinuedFloat
    \begin{subfigure}{\textwidth}
        \centering
        \includegraphics[width=0.32\textwidth]{./figs-mc-correction/reweighting-final/plot_step9-D0_iso-mu_eta.pdf}
        \includegraphics[width=0.32\textwidth]{./figs-mc-correction/reweighting-final/plot_step9-D0_iso-mu_pt.pdf}
        \caption{Stage 9.}
    \end{subfigure}

    \begin{subfigure}{\textwidth}
        \centering
        \includegraphics[width=0.32\textwidth]{./figs-mc-correction/reweighting-final/plot_step10-D0_iso-d0_comp2.pdf}
        \caption{Stage 10.}
    \end{subfigure}

    \caption[]{Final reweighting for \Dz channel, stages 9--10 (cont'd).}
\end{figure}

\begin{figure}[htb]
    \begin{subfigure}{\textwidth}
        \centering
        \includegraphics[width=0.32\textwidth]{./figs-mc-correction/reweighting-final/plot_step1-Dst_iso-b_log_fd_chi2.pdf}
        \includegraphics[width=0.32\textwidth]{./figs-mc-correction/reweighting-final/plot_step1-Dst_iso-d0_log_ip_chi2.pdf}
        \includegraphics[width=0.32\textwidth]{./figs-mc-correction/reweighting-final/plot_step1-Dst_iso-mu_log_ip_chi2.pdf}
        \caption{Stage 1.}
    \end{subfigure}

    \begin{subfigure}{\textwidth}
        \centering
        \includegraphics[width=0.32\textwidth]{./figs-mc-correction/reweighting-final/plot_step2-Dst_iso-k_comp.pdf}
        \includegraphics[width=0.32\textwidth]{./figs-mc-correction/reweighting-final/plot_step2-Dst_iso-k_log_ip_chi2.pdf}
        \includegraphics[width=0.32\textwidth]{./figs-mc-correction/reweighting-final/plot_step2-Dst_iso-k_pt.pdf}
        \caption{Stage 2.}
    \end{subfigure}

    \begin{subfigure}{\textwidth}
        \centering
        \includegraphics[width=0.32\textwidth]{./figs-mc-correction/reweighting-final/plot_step3-Dst_iso-pi_comp.pdf}
        \includegraphics[width=0.32\textwidth]{./figs-mc-correction/reweighting-final/plot_step3-Dst_iso-pi_log_ip_chi2.pdf}
        \includegraphics[width=0.32\textwidth]{./figs-mc-correction/reweighting-final/plot_step3-Dst_iso-pi_pt.pdf}
        \caption{Stage 3.}
    \end{subfigure}

    \begin{subfigure}{\textwidth}
        \centering
        \includegraphics[width=0.32\textwidth]{./figs-mc-correction/reweighting-final/plot_step4-Dst_iso-mu_comp.pdf}
        \includegraphics[width=0.32\textwidth]{./figs-mc-correction/reweighting-final/plot_step4-Dst_iso-mu_log_ip_chi2.pdf}
        \includegraphics[width=0.32\textwidth]{./figs-mc-correction/reweighting-final/plot_step4-Dst_iso-mu_pt.pdf}
        \caption{Stage 4.}
    \end{subfigure}

    \caption{
        Final reweighting for \Dstar channel, for a total of 12 stages.
        The black plots contain weights from \emph{all previous} stages,
        whereas the red ones contain weights from \emph{all} stages.
    }
    \label{fig:final-rwt-dst}
\end{figure}

\begin{figure}[htb]\ContinuedFloat
    \begin{subfigure}{\textwidth}
        \centering
        \includegraphics[width=0.32\textwidth]{./figs-mc-correction/reweighting-final/plot_step5-Dst_iso-d0_comp.pdf}
        \includegraphics[width=0.32\textwidth]{./figs-mc-correction/reweighting-final/plot_step5-Dst_iso-d0_log_ip_chi2.pdf}
        \includegraphics[width=0.32\textwidth]{./figs-mc-correction/reweighting-final/plot_step5-Dst_iso-d0_pt.pdf}
        \caption{Stage 5.}
    \end{subfigure}

    \begin{subfigure}{\textwidth}
        \centering
        \includegraphics[width=0.32\textwidth]{./figs-mc-correction/reweighting-final/plot_step6-Dst_iso-d0_eta.pdf}
        \includegraphics[width=0.32\textwidth]{./figs-mc-correction/reweighting-final/plot_step6-Dst_iso-d0_pt.pdf}
        \caption{Stage 6.}
    \end{subfigure}

    \begin{subfigure}{\textwidth}
        \centering
        \includegraphics[width=0.32\textwidth]{./figs-mc-correction/reweighting-final/plot_step7-Dst_iso-k_eta.pdf}
        \includegraphics[width=0.32\textwidth]{./figs-mc-correction/reweighting-final/plot_step7-Dst_iso-k_pt.pdf}
        \caption{Stage 7.}
    \end{subfigure}

    \begin{subfigure}{\textwidth}
        \centering
        \includegraphics[width=0.32\textwidth]{./figs-mc-correction/reweighting-final/plot_step8-Dst_iso-pi_eta.pdf}
        \includegraphics[width=0.32\textwidth]{./figs-mc-correction/reweighting-final/plot_step8-Dst_iso-pi_pt.pdf}
        \caption{Stage 8.}
    \end{subfigure}

    \caption[]{Final reweighting for \Dstar channel, stages 5--8 (cont'd).}
\end{figure}

\begin{figure}[htb]\ContinuedFloat
    \begin{subfigure}{\textwidth}
        \centering
        \includegraphics[width=0.32\textwidth]{./figs-mc-correction/reweighting-final/plot_step9-D0_iso-mu_eta.pdf}
        \includegraphics[width=0.32\textwidth]{./figs-mc-correction/reweighting-final/plot_step9-D0_iso-mu_pt.pdf}
        \caption{Stage 9.}
    \end{subfigure}

    \begin{subfigure}{\textwidth}
        \centering
        \includegraphics[width=0.32\textwidth]{./figs-mc-correction/reweighting-final/plot_step10-D0_iso-d0_comp2.pdf}
        \caption{Stage 10.}
    \end{subfigure}

    \begin{subfigure}{\textwidth}
        \centering
        \includegraphics[width=0.32\textwidth]{./figs-mc-correction/reweighting-final/plot_step11-Dst_iso-spi_eta.pdf}
        \includegraphics[width=0.32\textwidth]{./figs-mc-correction/reweighting-final/plot_step11-Dst_iso-spi_pt.pdf}
        \caption{Stage 11.}
    \end{subfigure}

    \begin{subfigure}{\textwidth}
        \centering
        \includegraphics[width=0.32\textwidth]{./figs-mc-correction/reweighting-final/plot_step12-Dst_iso-spi_comp.pdf}
        \includegraphics[width=0.32\textwidth]{./figs-mc-correction/reweighting-final/plot_step12-Dst_iso-spi_log_ip_chi2.pdf}
        \includegraphics[width=0.32\textwidth]{./figs-mc-correction/reweighting-final/plot_step12-Dst_iso-spi_pt.pdf}
        \caption{Stage 12.}
    \end{subfigure}

    \caption[]{Final reweighting for \Dstar channel, stages 9--12 (cont'd).}
\end{figure}

\paragraph{Bins with 0 MC event} For bins without MC events,
the weights are replaced by 1, to avoid arbitrarily large weights.
This is justified by the fact that if the corresponding data bins also have
0 event, no reweighting is needed;
otherwise it is likely that large weights will be produced by the reweighting
procedure and it is the case we want to avoided.

\paragraph{Treatment of large weights}
For large weights, defined as $w \geq 50$, the number of events $n$
in the corresponding bin of the \emph{subtracted data} sample is checked:
If $n \leq 10$, the large weight is treated as a fluctuation and is
set back to 1;
otherwise, it is capped at 50.
No other weight cap is imposed on the \jpsi\kaon-derived weight at this
stage\footnote{
    There are, however, weight caps on the products of weights,
    which will be described in
    \cref{sec:fit-to-data:fit-tmpl}.
}.

\paragraph{Under and overflow bins}
The under and overflow bins are taken into account during
the reweighting process.



\section{Validation of final reweighting}

This is a work in progress.


% Generated in umd-lhcb/rdx-run2-analysis/fit:
%   make tab-reweight
\begin{landscape}
\begin{table}[p]
    \centering
    \caption{
        Reweighting stages and binning schemes for final reweighting.
    }
    \label{tab:rwt-final-vars}
    \begin{tabular}{c|l|c|l|c|l}
        \toprule
         {\bf Variable 1}             & {\bf Binning 1}   & {\bf Variable 2}               & {\bf Binning 2}   & {\bf Variable 3}                     & {\bf Binning 3}   \\
        \midrule
         $D^0\mu$ $\log(FD\, \chi^2)$ & 10, 4 -- 12.5     & $D^0$ $\log(IP\, \chi^2)$      & 10, 2 -- 9        & $\mu$ $\log(IP\, \chi^2)$            & 10, 3.6 -- 11     \\
         $K$ $p_T$ [GeV]              & 10, 0 -- 11       & $K$ $\log(IP\, \chi^2)$        & 10, 3.6 -- 10.2   & $K$ $\sqrt{IP\, \chi^2} / IP$        & 10, 5 -- 100      \\
         $\pi$ $p_T$ [GeV]            & 10, 0 -- 12.5     & $\pi$ $\log(IP\, \chi^2)$      & 10, 3.6 -- 10.2   & $\pi$ $\sqrt{IP\, \chi^2} / IP$      & 10, 5 -- 100      \\
         $\mu$ $p_T$ [GeV]            & 10, 0 -- 12       & $\mu$ $\log(IP\, \chi^2)$      & 10, 3.6 -- 10.8   & $\mu$ $\sqrt{IP\, \chi^2} / IP$      & 10, 0 -- 100      \\
         $D^0$ $p_T$ [GeV]            & 10, 2 -- 18.5     & $D^0$ $\log(IP\, \chi^2)$      & 10, 2 -- 9        & $D^0$ $\sqrt{IP\, \chi^2} / IP$      & 10, 18 -- 102     \\
         $D^0$ $p_T$ [GeV]            & 20, 2 -- 18.5     & $D^0$ $\eta$                   & 10, 1.8 -- 5      & --                                   & --                \\
         $K$ $p_T$ [GeV]              & 20, 0 -- 11       & $K$ $\eta$                     & 10, 1.8 -- 5      & --                                   & --                \\
         $\pi$ $p_T$ [GeV]            & 20, 0 -- 12.5     & $\pi$ $\eta$                   & 10, 1.8 -- 5      & --                                   & --                \\
         $\mu$ $p_T$ [GeV]            & 20, 0 -- 12       & $\mu$ $\eta$                   & 10, 1.8 -- 5      & --                                   & --                \\
         $D^0$ $\log(1 - DIRA)$       & 20, -14.2 -- -8.4 & --                             & --                & --                                   & --                \\
         slow $\pi$ [GeV]\parnote{
             \label{parnote:final-rwt-dst}
             This is for \Dstar channel only.
         }                            & 6, 0 -- 1.6       & slow $\pi$ $\eta$              & 10, 1.8 -- 4.8    & --                                   & --                \\
         slow $\pi$ $p_T$ [GeV]\parnoteref{parnote:final-rwt-dst}
                                      & 6, 0 -- 1.6       & slow $\pi$ $\log(IP\, \chi^2)$ & 10, -4 -- 7       & slow $\pi$ $\sqrt{IP\, \chi^2} / IP$ & 10, 0 -- 50       \\
        \bottomrule
    \end{tabular}
    \parnotes
\end{table}
\end{landscape}
