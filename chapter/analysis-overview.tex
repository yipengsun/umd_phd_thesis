\chapter{Analysis overview}
\label{ref:overview}

This analysis aims at an extraction of \RD and \RDst from
LHCb 2016 data\footnote{
    It is planned to use full LHCb run 2 data (2016--2018) at a later date.
}, reconstructed in $\Dstarp\mun$ and $\Dz\mun$ final states.
These states are chosen so that the final state particles
(e.g. $K^-, \pi^+, \mu^-$) are all charged, because the LHCb detector is not
well-suited for neutral particle reconstruction.
A complication of these states is that they are \emph{inhomogeneous},
that is, not all \Dz\mun final state events are actually in \Dz\mun final
states:
\Dstarp\mun states feed down into the \Dz\mun states,
because \Dstarp decays strongly to a \Dz and a slow \pip,
with the latter often missed in the reconstruction;
for example, a slow \pip with a softer momentum may fail to register a
well-defined track in the detector, thus no \Dstarp is constructed for this
event.
In addition, \Dstarz\mun states,
which has a branch fraction of about 2.5 times larger compared to \Dz\mun
states\footnote{
    More precisely, the relevant branching fractions are:
    $\Bm \rightarrow \Dstarz \ell^- \neulb$ and
    $\Bm \rightarrow \Dz \ell^- \neulb$.
    % 5.58 vs 2.30, from PDG live
},
are contributing exclusively to the \Dz\mun states, because the neutral slow
\piz, a product of \Dstarz decay, is entirely missed.
Therefore, a \emph{simultaneous} fit,
assuming isospin symmetry,
to both \Dstarp\mun and \Dz\mun states is needed to
account for the correlations\footnote{
    \RD and \RDst are, however, considered to be theoretically uncorrelated.
} between \RD and \RDst due to experimental limitations mentioned above.

Each decay mode contributing to $\D^{*+,0}\mun$ final states have differing
kinematics which are used in the fit to determine its yield.
The differentiating kinematic variables, defined in the \B rest frame, are:

\begin{itemize}
    \item \mmSq: Defined as $(p_B - p_{D^{*}} - p_\mu)^2$.
        Typically due to missing neutrino(s).
        For $B \rightarrow D \mun \neumb$ decays, only 1 neutrino is missing,
        so \mmSq is small ($\sim$ the mass of neutrino);
        for $B \rightarrow D \taum (\rightarrow \mun\neumb\neut) \neutb$,
        3 neutrinos are missing, so \mmSq can be large.
    \item \el: The energy of \mun. \mun coming from \taum decays are typically
        softer, i.e. with smaller energy, due to reduced available phase space.
    \item \qSq:
\end{itemize}
