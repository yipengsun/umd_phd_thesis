\chapter{Analysis overview}
\label{ref:overview}

This analysis aims at an extraction of \RD and \RDst from
LHCb 2016 data\footnote{
    It is planned to use full LHCb run 2 data (2016--2018) at a later date.
}, reconstructed in $\Dstarp\mun$ and $\Dz\mun$ final states.
These states are chosen so that the final state particles
(e.g. $K^-, \pi^+, \mu^-$) are all charged, because the LHCb detector is not
well-suited for neutral particle reconstruction.
A complication of these states is that they are \emph{inhomogeneous},
that is, not all \Dz\mun final state events are actually in \Dz\mun final
states:
\Dstarp\mun states feed down into the \Dz\mun states,
because \Dstarp decays strongly to a \Dz and a slow \pip,
with the latter often missed in the reconstruction;
for example, a slow \pip with a softer momentum may fail to register a
well-defined track in the detector, thus no \Dstarp is constructed for this
event.
In addition, \Dstarz\mun states,
which has a branching fraction of about 2.5 times larger compared to \Dz\mun
states\footnote{
    More precisely, the relevant branching fractions are:
    $\Bm \rightarrow \Dstarz \ell^- \neulb$ and
    $\Bm \rightarrow \Dz \ell^- \neulb$.
    % 5.58 vs 2.30, from PDG live
},
are contributing exclusively to \Dz\mun states, because the neutral slow
\piz, a product of \Dstarz decay, is entirely missed.
Therefore, a \emph{simultaneous} fit,
assuming isospin symmetry,
to both \Dstarp\mun and \Dz\mun states is needed to
account for the correlations\footnote{
    \RD and \RDst are, however, considered to be theoretically uncorrelated.
} between \RD and \RDst and improve precision\footnote{
    As opposed to extract \RDst from \Dstarp\mun sample alone.
} for \RDst.

Both \Dstarp\mun and \Dz\mun final states contain contributions from many decay
modes.
Each decay mode has differing kinematics
which are often modelled from Monte-Carlo simulation (MC),
sometimes from data control samples.
These differences are used in the fit to determine the yields of all
relevant decay modes.
The differentiating kinematic variables, defined in the \B rest frame,
referred as ``fit variables'',
are the following:
%%%%
\begin{itemize}
    \item \mmSq: Defined as $(p_B - p_{D^{(*)}} - p_\mu)^2$,
        the missing mass (squared) is typically due to missing (unreconstructed)
        neutrino(s).
        For $B \rightarrow D \mun \neumb$ decays, only 1 neutrino is missing,
        so \mmSq is small ($\sim$ the mass of neutrino);
        for $B \rightarrow D \taum (\rightarrow \mun\neumb\neut) \neutb$,
        3 neutrinos are missing, so \mmSq can be large.
    \item \el: The energy of \mun. \mun coming from \taum decays are typically
        softer, i.e. with smaller energy, due to reduced available phase space.
    \item \qSq: The momentum transfer, defined as the invariant mass squared
        of the virtual $W$ boson.
        The \qSq spectrum are notably different between
        $B \rightarrow \text{P} \ell\neu$ and $B \rightarrow \text{V} \ell\neu$
        decays, where ``P'' stands for a pseudoscalar meson (e.g. \Dz),
        and ``V'' a vector (e.g. \Dstar);
        the \qSq spectrum is softer
        (more likely to be at lower \qSq) for pseudoscalar mesons.
        This will be discussed in more detail in \cref{ref:theory}.
        In addition, the semitauonic decays are restricted to the phase space
        where $\qSq > m^2_\tau$,
        whereas the semimuonic decays extend down to $\qSq > m^2_\mu \approx 0$.
        Both modes share a common \qSq upper bound $(m_B - m_{D^{(*)}})^2$
        where the $D$ meson is produced at rest in the $B$ rest frame.
\end{itemize}
%%%%
These variables, however, cannot be deduced exactly at LHCb,
because \B mesons
are typically produced from hadronization of $pp$ collisions,
carrying an unknown portion of $pp$ momenta,
so the \B rest frame is not known.
A technology, termed ``rest frame approximation'' (RFA), is developed for LHCb
to estimate the \B rest frame, making computation of fit variables possible.
% As an demonstration that RFA can reproduce these variables reasonably well,
% MC true rest frame variables and estimated variables with RFA for
% $\Bm \rightarrow \Dz\mun\neumb$ decay are shown in
% \cref{fig:rfa-variables}.
More information about RFA can be found in \cref{appx:rfa}.

% \begin{figure}[htb]
%     \caption{
%         MC true rest frame variables vs. estimated variables with RFA in
%         $\Bm \rightarrow \Dz\mun\neumb$ decay.
%     }
%     \label{fig:rfa-variables}
% \end{figure}

The workflow of this analysis is the following:
First, \Dstarp\mun and \Dz\mun final states are selected, which is discussed
in \cref{ref:selection}.
Then, emulation of certain detector responses to MC are carried out offline:
This analysis requires a large number of MC events to control statistical
uncertainty, making \emph{simulating} every detector response impractical.
Both the reason (with more details) and the emulation procedure are discussed in
\cref{ref:mc-emulation}.
After, various discrepancies between data and MC simulation,
mainly in detector responses, are corrected and the procedures detailed
in \cref{ref:mc-correction}.
Finally, \RDX are extracted from a simultaneous fit, with fit procedure listed
in \cref{ref:fit} and selected systematic uncertainties discussed in
\cref{ref:sys-uncert}.
More detailed workflow is plotted in \cref{fig:workflow},
and is included here for reference.
As an update to LHCb \RDX run 1 analysis, this analysis shares a similar
workflow with that of run 1.
Many technologies developed in run 1 are carried out to this analysis.
The main additional difficulties this analysis has to face are the emulation
on MC (already listed above) and a wide variety of software usage, including
different versions of the same software.
The latter is summarized as a side in \cref{ref:reprod-analysis}.

\begin{figure}[htb]
    \caption{
        More detailed workflow for LHCb \RDX analysis based on 2016 data.
    }
    \label{fig:workflow}
\end{figure}

Additional information are organized as follows:
An overview of theory on semileptonic \B decay is provided in
\cref{ref:theory},
followed by a brief introduction to the LHCb detector in \cref{ref:detector}.
The conclusion of this analysis, still ongoing, is provided in
\cref{ref:conclusion}.
