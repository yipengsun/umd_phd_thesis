\chapter{Analysis overview}
\label{ref:overview}

This analysis aims at an extraction of \RD and \RDst,
with the definition reproduced from the introduction:
\begin{equation}
    \RDX \equiv \frac{\BFDTau}{\BFDMu}
\end{equation}
from LHCb 2016 data\footnote{
    It is planned to use full LHCb run 2 data (2016--2018) at a later date.
}, reconstructed in (with visible final state particles marked in red):
%%%%
\begin{itemize}
    \item $\Bzb \rightarrow \Dstarp (\rightarrow \Dz (\rightarrow \textcolor{red}{\Km \pip})\textcolor{red}{\pip}) \taum (\rightarrow \textcolor{red}{\mun} \neumb \neut) \neutb)$
        for \RDst signal decay\footnote{
            A parton level diagram for the signal decays is drawn in
            \cref{fig:decay-diagrams} as a reference.
        }
    \item $\Bzb \rightarrow \Dstarp (\rightarrow \Dz (\rightarrow \textcolor{red}{\Km \pip})\textcolor{red}{\pip}) \textcolor{red}{\mun} \neumb$
        for \RDst normalization decay
    \item $\Bm \rightarrow \Dz (\rightarrow \textcolor{red}{\Km \pip}) \taum (\rightarrow \textcolor{red}{\mun} \neumb \neut) \neutb)$
        for \RD signal decay
    \item $\Bm \rightarrow \Dz (\rightarrow \textcolor{red}{\Km \pip}) \textcolor{red}{\mun} \neumb$
        for \RD normalization decay
\end{itemize}
such that $\Dstarp\mun$ and $\Dz\mun$ are the final reconstructed states
which are often referred in this text as \Dstar channel and \Dz channel in later
chapters.
These states are chosen so that the final state particles
(e.g. $K^-, \pi^+, \mu^-$) are all charged,
because the LHCb detector is not well-suited for neutral particle
reconstruction.

\begin{figure}[!htb]
    \centering
    \resizebox{0.8\textwidth}{!}{
        \begin{tikzpicture} \begin{feynman}
    \vertex (a1) {\(b\)};
    \vertex[right=7em of a1] (a2);
    \vertex[right=7em of a2] (a4) {\(c\)};

    \vertex[below=3em of a1] (b1) {\(\overline q\)};
    \vertex[below=3em of a4] (b2) {\(\overline q\)};

    \vertex at ($(a2)!0.5!(a2)!0.5cm!90:(a2)$) (d);

    \vertex[right=7em of a4] (e1) {\(\nu_\tau\)};
    \vertex[above right=3em of e1] (e2);
    \vertex[below right=of e2] (e3) {\(\mu^-\)};
    \vertex[above right=of e2] (e4) {\(\overline\nu_\mu\)};

    \vertex[above=3em of a4] (c2);
    \vertex at ($(c2)!0.5!(e1)$) (c1);
    \vertex[above right=3em of c2] (c3) {\(\nu_\tau\)};

    \diagram* {
        (a1) -- [fermion] (a2) -- [fermion] (a4),
        (b2) -- [fermion] (b1),
        (c3) -- [fermion] (c2) -- [fermion, edge label=\(\tau^-\)] (c1),
        (a2) -- [blue, boson, edge label=\(W^{-}\)] (c2),
        (c1) -- [fermion] (e1),
        (c1) -- [boson, edge label=\(W^-\)] (e2),
        (e2) -- [fermion] (e3),
        (e2) -- [anti fermion] (e4),
    };

    \draw [decoration={brace}, decorate] (b1.south west) -- (a1.north west)
        node [pos=0.5, left] {\(B\)};
    \draw [decoration={brace}, decorate] (a4.north east) -- (b2.south east)
        node [pos=0.5, right] {\(D^{(\ast)}\)};
\end{feynman} \end{tikzpicture}

    }

    \caption{
        Parton level diagrams for $B \rightarrow D^{(*)} \taum\neutb$ signal
        decays,
        where $\overline q$ can be either a $\overline u$ or a $\overline d$.
    }
    \label{fig:decay-diagrams}
\end{figure}

A complication of these states is that they are \emph{inhomogeneous},
that is, not all \Dz\mun final state events are actual decays with just \Dz\mun
in their final states:
$\Bzb \rightarrow \Dstarp\ellm\neulb$ decays feed down into the \Dz\mun states,
because \Dstarp decays strongly to a \Dz and a slow \pip,
with the latter often missed in the reconstruction;
for example, a slow \pip with a softer momentum may fail to register a
well-defined track in the detector, thus no \Dstarp is constructed for this
event.
In addition, $\Bm \rightarrow \Dstarz\ellm\neulb$ decays,
which has a branching fraction of about 2.5 times larger compared to
$\Bm \rightarrow \Dz\ellm\neulb$ decays,
contribute exclusively to \Dz\mun states, because the neutral slow
\piz, a product of \Dstarz decay, is entirely missed.
Therefore, a \emph{simultaneous} fit,
assuming isospin symmetry,
to both \Dstarp\mun and \Dz\mun states is needed to
account for the correlations between \RD and \RDst and improve
precision\footnote{
    As opposed to extract \RDst from \Dstarp\mun sample alone.
} for \RDst.

Both \Dstarp\mun and \Dz\mun final states contain contributions from signal,
normalization, and many background decay modes.
Each decay mode has differing kinematics
which are often modelled from Monte-Carlo simulation (MC),
sometimes from data control samples.
These differences are used in the fit to determine the yields of all
relevant decay modes.
The differentiating kinematic variables, defined in the \B rest frame,
referred as ``fit variables'',
are the following:
%%%%
\begin{itemize}
    \item \mmSq: Defined as $(p_B - p_{D^{(*)}} - p_\mu)^2$,
        the missing mass (squared) is typically due to missing (unreconstructed)
        neutrino(s).
        For $B \rightarrow D \mun \neumb$ decays, only 1 neutrino is missing,
        so \mmSq is small ($\sim$ the mass of neutrino);
        for $B \rightarrow D \taum (\rightarrow \mun\neumb\neut) \neutb$,
        3 neutrinos are missing, so \mmSq can be large.
    \item \el: The energy of \mun in the $B$ rest frame.
        \mun coming from \taum decays are typically
        softer, i.e. with smaller energy, due to reduced available phase space.
    \item \qSq: The momentum transfer, defined as the invariant mass squared
        of the virtual $W$ boson (marked in blue in \cref{fig:decay-diagrams})
        and calculated as $(p_B - p_{D^{(*)}})^2$.
        The \qSq spectra of
        $B \rightarrow \text{P} \ell\neu$ and $B \rightarrow \text{V} \ell\neu$
        decays are notably different,
        where ``P'' stands for a pseudoscalar meson (e.g. \Dz),
        and ``V'' a vector (e.g. \Dstar);
        the \qSq spectrum is softer
        (more likely to be at lower \qSq) for pseudoscalar mesons.
        This will be discussed in more detail in \cref{ref:theory}.
        In addition, the semitauonic decays are restricted to the phase space
        where $\qSq > m^2_\tau$,
        whereas the semimuonic decays extend down to $\qSq > m^2_\mu \approx 0$.
        Both modes share a common \qSq upper bound $(m_B - m_{D^{(*)}})^2$
        where the $D$ meson is produced at rest in the $B$ rest frame.
\end{itemize}
%%%%
These variables, however, cannot be deduced exactly at LHCb,
because \B mesons
are typically produced from hadronization of $pp$ collisions,
carrying an unknown portion of $pp$ momenta,
so the \B rest frame is not known.
A procedure, termed ``rest frame approximation'' (RFA), is developed for LHCb
to estimate the \B rest frame, making computation of fit variables possible.
% As an demonstration that RFA can reproduce these variables reasonably well,
% MC true rest frame variables and estimated variables with RFA for
% $\Bm \rightarrow \Dz\mun\neumb$ decay are shown in
% \cref{fig:rfa-variables}.
More information about RFA can be found in \cref{appx:rfa}.

% \begin{figure}[!htb]
%     \caption{
%         MC true rest frame variables vs. estimated variables with RFA in
%         $\Bm \rightarrow \Dz\mun\neumb$ decay.
%     }
%     \label{fig:rfa-variables}
% \end{figure}

This analysis has the same main idea as in the LHCb \RDX run 1 analysis:
fit the parameters characterizing various background decays with separate
control samples, then import these parameters either as constraints or fully
fixed into the signal fit;
this is because while the signal(-enriched) samples contain many backgrounds,
they are only a small fraction relative to the signal and normalization
combined,
thus no good description of the backgrounds can be extracted from the signal fit
alone.

As an update to the previous analysis, this analysis shares a similar
workflow with that of run 1.
The workflow of this analysis is:
First, \Dstarp\mun and \Dz\mun final states are selected with the procedure
described in \cref{ref:sel}.
Then, emulation of certain detector responses to MC are carried out offline
with the procedure described in
\cref{ref:mc-emulation},
as this analysis requires a large number of MC events with only partial detector
simulation to be computationally viable.
After, various mis-modellings in MC simulation,
mainly in form factor modelling and detector responses, are updated and
corrected with the procedures detailed in \cref{ref:mc-cor}.
Finally, \RDX are extracted from a simultaneous fit, with fit procedure and
results listed in \cref{ref:fit}
and selected systematic uncertainties discussed in
\cref{ref:sys-uncert}.
%%%%
In passing the author points out the major differences between this analysis
and the pathfinder run 1 \RDX analysis:
\begin{itemize}
    \item Due to a larger integrated luminosity
        (5.4~fb$^{-1}$ in run 2 versus 3.0~fb$^{-1}$ in run 1),
        a larger $B$ production cross section due to higher collision energy
        (13~TeV center of mass energy in run 2 versus 7~TeV in run 1),
        and a dedicated trigger resulting in a higher selection efficiency,
        the run 2 analysis can work with a much larger dataset:
        For 2016 alone the selected \Dz\mun signal sample contains 2,178,793
        events, whereas for run 1 combined 1,734,133 events, a factor of
        $\sim\!1.26$ gain.

        As briefly mentioned in the previous paragraph,
        a large amount of MC is needed,
        which makes it impractical to simulation most of the detector.
        Instead, only the tracking system is simulated and additional detector
        responses such as hardware triggers are \emph{emulated} offline.

    \item In order to update (reweight) MC form factor models offline,
        the run 1 analysis implemented two ad-hoc approaches for form factor
        reweighting:
        one is based on \texttt{XslFF} reweighting class from BaBar,
        the other is completely in-house.
        A large amount of time is spent on validation of the form factor
        reweighting implementations.

        Recently, a new software package, named \Hammer, allows easier form
        factor reweighting and comes pre-validated by the \Hammer authors.
        This analysis leverages on \Hammer for an easier to implement and more
        reliable form factor reweighting procedure.
\end{itemize}

Additional information is organized as follows:
An overview of theory of semileptonic \B decay is provided in
\cref{ref:theory},
followed by a brief introduction to the LHCb detector in \cref{ref:detector}.
The ``conclusion'' of this still ongoing analysis is provided in
\cref{ref:conclusion}.
