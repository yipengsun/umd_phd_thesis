\chapter{Conclusion}
\label{ref:conclusion}

The overall fit quality is good with reasonable pulls.
Still, there is a large discrepancy in the high-\qSq bin for the \Dstar signal
(ISO) fit that needs to be investigated further.
The current total uncertainties are: ???.
Since only the 2016 data is included,
smaller uncertainties are to be expected once the 2017 and 2018 data is
included,
provided that the \qSq bin discrepancy is resolved..

As a proof-of-concept, this 2016 \RDX analysis demonstrates that both the
general fit strategy, that is, obtain the background parameters with separate
control samples then import them into the signal fit, as well as most of the
methods developed in the run 1 pathfinder analysis,
such as the isolation BDT, the \UBDT, and the unfolding algorithm,
can be carried over for run 2.
The additional challenge of emulating the triggers due to the use of
tracker-only MC (to have enough MC statistics) is considered to be fully solved
with the method described in this thesis, with the emulation procedure portable
to the year 2017 and 2018.
Also, the form factor reweighting procedure has been updated to use \Hammer,
the current state-of-art form factor reweighting program,
which is more robust and easier to maintain.
It is expected that the 2017 and 2018 data can be processed with the same
workflow developed in this analysis.
