\chapter{Emulation of detector responses on MC simulation}
\label{ref:mc-emulation}

To ensure the statistical uncertainties from MC samples are small compared to
that of data, it is typical to request MC samples with a size 4 times as large
as data.
Due to the sheer size of LHCb run 2 data (a factor of $\sim\!1.4$ for 2016
alone compared to full run 1), it is not computationally possible to generate
enough full simulation (FullSim) MC, where every detector response is simulated.

It is estimated in \cite{LHCb-INT-2019-025}
that about 85\% of the computation time is spent on simulating RICH and the
calorimeter system.
Tracker-only (TO) MC sets all but the tracking system of the detector to
passive material, making the simulation about 8 times faster than FullSim.
Hence, to reach target statistical uncertainties,
TO MC is (almost) exclusively used for this analysis.
Therefore, detector responses like trigger and particle identification (PID)
need to be emulated offline.
The emulation procedures for these responses are described in the rest of the
chapter.


\section{Trigger emulation}
\subsection{Emulation of L0Hadron TOS}

Recall that the L0 trigger requirements for this analysis are: Either \Dz
triggers L0 hadron (\Dz L0Hadron TOS\footnote{
    TOS: Triggered On Signal. Can be used as a noun or a verb.
    For more information about trigger TIS, TOS, TISTOS, and TOB, refer to
    \cref{appx:trigger-cat}.
}) or
the rest of the event aside from \B (and its decay chain) triggers at least
one L0 trigger
(\B L0Global TIS\footnote{
    TIS: Triggered Independent of Signal.
    In this case \B is the signal track, and the rest of event is inspected to
    determine if it triggers at least a L0 trigger.
}).

The L0 hadron is triggered when the sum of the transverse energy\footnote{
    Defined as: $E_T \equiv \sqrt{m^2 + p^2} \sin\theta$.
    In massless limit, $E_T = p_T$.
} $E_T$ deposited
in a single $2 \times 2$ HCAL cluster is above a set threshold\footnote{
    See Table 1 in \cite{LHCb-DP-2019-001} for year-by-year thresholds of each
    L0 trigger.
}, as described in \cite{LHCb-DP-2019-001}.
For L0Hadron TOS, it is required that for all HCAL clusters associated with the
signal track (\Dz in this case), with association predicted by the tracking
system,
the highest $E_T$ is above the triggering threshold.

Tracker-only MC records the tracker-predicted position where a track hits the
calorimeters.
Hence, though the triggering variable $E_T$ is not accessible,
variables related to $E_T$, or, equivalently, trigger decision, are.
It is conceivable that L0Hadron TOS can be emulated by providing
the relevant variables and the trigger decision to train some regressor,
which can in turn be used to predict trigger decisions based on the training
variables alone, without requiring emulating $E_T$ at all.
Below the emulation procedure, inspired by \cite{LHCb-INT-2019-025}, is
described:

\begin{enumerate}
    \item The \smalltt{TupleToolL0Calo} tool is configured to save HCAL
        information.
    \item The training variables, listed in \cref{tab:l0-hadron-emu-vars}, and
        the trigger decision from a FullSim MC of decay
        $\Bz \rightarrow \Dstarp \mu \nu$, a normalization mode,
        is used to train \xgboost, a BDT-based regressor.
    \item Once the model is trained, with just the training variables,
        \xgboost predicts the probabilities of events passing L0 hadron trigger.
    \item The probabilities are applied as weights to the same FullSim MC
        to check the quality of the emulation.
        It is shown in \cref{fig:l0hadron-tos-emu} that emulated L0Hadron TOS
        trigger has excellent agreement with \emph{simulated} trigger in
        FullSim.
\end{enumerate}

\begin{table}[htb]
    \caption{Variables used to train \xgboost for L0 hadron emulation.}
    \label{tab:l0-hadron-emu-vars}
    \centering
    \begin{tabular}{cc}
        \toprule
        Group       & Variables \\
        \midrule
        Event-level & nTracks \\
        \Dz         & $p$, $p_T$ \\
        \midrule
        \Km         & \makecell{
                      $p$, $p_T$,
                      real $E_T$\parnote{
                          This is the $E_T$ measured by the tracker.
                          In massless limit, $E_T = p_T$.
                      }, \\
                      HCAL $x$-projection, HCAL $y$-projection,
                      HCAL region\parnote{
                          This indicates whether the hits are in the inner or
                          outer region of HCAL.
                          The outer region has a coarser granularity than the
                          inner, thus it is more likely that the energies from
                          \Km and \pip are deposited in the same cluster,
                          enhancing L0Hadron TOS efficiency.
                      }} \\
        \midrule
        \pip        & same as \Km \\
        \bottomrule
    \end{tabular}
    \begin{flushleft}
        \parnotes
    \end{flushleft}
\end{table}

% Generated in //lhcb-ntuples-gen/studies/trigger_emulation-l0_hadron_tos_debug:
% By running:
%   git annex get .
%   debug_l0hadron.py
% in the folder specified above.
\begin{figure}
    \centering
    \includegraphics[width=0.7\textwidth]{./figs-mc-emulation/emulate-l0hadron-tos/b0_L0Hadron_TOS_xgb4_valid_d0_pt.pdf}
    \caption{
        The emulated trigger probabilities applied as a weight to FullSim has
        excellent agreement with real L0Hadron TOS trigger \emph{simulated} in
        FullSim.
    }
    \label{fig:l0hadron-tos-emu}
\end{figure}


\subsection{Emulation of L0Global TIS}

As discussed before, L0Global TIS is determined by \emph{the rest of event}
(everything but the reconstructed \B decay chain) which is \emph{not saved}
after event reconstruction.
Therefore, a completely different approach is needed for its emulation.

Instead of predicting event-by-event L0Global TIS probability based on the
properties of the rest of event, it is possible to measure L0Global TIS as
efficiencies binned in \emph{some reconstructed variables\footnote{
    The choice of variables will be discussed later.
}} externally,
then apply L0Global TIS efficiencies based on the bin the event falls in.

In addition, for hadron colliders,
the rest of event is \emph{busy} due to hadronization:
Indeed, for data reconstructed in
$\Bm \rightarrow \Dz (\rightarrow \Km \pip) \mun \neumb$ mode which is used in
this analysis,
the average number of tracks per event is 170,
whereas the selected \Dz\mun final state contains only 3 tracks (\Km, \pip, and
\mun, as \Dz and \Bm are reconstructed from these tracks)!
Therefore, it can be argued that the rest of event are essentially the same
between different reconstructions,
and L0Global TIS efficiencies are portable across different \B decay modes.

Assuming the L0Global TIS efficiencies can be measured in a data
sample reconstructed in a particular decay mode, the measured efficiencies
can be applied to TO MC as weights regardless of the decays.
This has the added benefit of L0Global TIS efficiencies agreeing with data by
definition, so it does not need additional data/MC correction.

To summarize, if the following assumptions hold, then the emulation outlined
above should work:

\begin{enumerate}
    \item The L0Global TIS efficiencies can be measured in data.
    \item The L0Global TIS efficiencies do not depend on \B decay mode.
\end{enumerate}

Before we proceed, let us define L0Global TIS efficiency
$\epsilon_\text{L0Global TIS}$ more precisely.
According to \cite{LHCb-PUB-2014-039}, in LHCb $\epsilon_\text{L0Global TIS}$
is conventionally defined as:

\begin{equation}
    \label{eqn:eff-l0global-tis}
    \epsilon_\text{L0Global TIS} \equiv
    \frac{N_\text{L0Global TIS \& selected}}{N_\text{selected}}
\end{equation}
which is not directly evaluable in real data, as $N_\text{selected}$ refers to
number of selected events \emph{regardless} of their trigger decisions, which is
not measurable (in data).

However, the conditional efficiency, $\epsilon_\text{L0Global TIS|L0Muon TOS}$,
is measurable:

\begin{equation}
    \epsilon_\text{L0Global TIS|L0Muon TOS} \equiv
    \frac{N_\text{L0Global TIS \& L0Muon TOS \& selected}}{
          N_\text{L0Muon TOS \& selected}}
\end{equation}
as long as L0Muon on signal and all L0 triggers (L0Global) on rest of event are
uncorrelated,
$\epsilon_\text{L0Global TIS|L0Muon TOS}$ is equivalent to
$\epsilon_\text{L0Global TIS}$.
Such a method is called \emph{TISTOS method}.
Unfortunately, there is a correlation between the two triggers due to
the fact that \bbbar are produced in pairs, making kinematics of the selected
signal \B meson and the background \Bbar meson, part of rest of event,
correlated.

Still, it is argued in \cite{LHCb-PUB-2014-039} that L0Global TIS and L0Muon TOS
are \emph{uncorrelated} in a small region of signal \B phase space.
Thus, it is possible to measure $\epsilon_\text{L0Global TIS}$ \emph{binned} in
\B kinematic variables with TISTOS method.

In \cite{LHCb-INT-2019-025} it is checked that binned
$\epsilon_\text{TISTOS} = \epsilon_\text{TIS}$ in
$\B \rightarrow \jpsi K$ FullSim MC\footnote{
    TISTOS does not always work. See \cref{appx:suppl:l0global-tis} for a
    counter example.
}, and evaluated
$\epsilon_\text{L0Global TIS}$ on data reconstructed in $\jpsi K$, binned
in $\log(p_z)$ and $\log(p_T)$ of the \B meson.
It is also checked that the efficiencies are portable among
$\B \rightarrow \jpsi K$, $\B \rightarrow \Dp \muon \neu$, and
$\B \rightarrow \Dp \tauon \neu$ decays with MC samples.
%%%%
In \cref{fig:l0global-tis-portable} it is checked
that the portability still holds for this analysis in
$\B \rightarrow \Dstarp \muon \neu$ and $\B \rightarrow \Dstarp \tauon \neu$.
The efficiencies obtained in \cite{LHCb-INT-2019-025} are applied to all TO MC
samples in this analysis, with binning variables changed to true momenta, as
the reconstruction in this analysis contains missing neutrino(s),
whereas $\jpsi K$ does not.

% Generated in /lhcb-ntuples-gen/studies/trigger_emulation-l0_global_tis_debug:
% By running:
%   debug_l0global.py
% in the folder specified above.
\begin{figure}[ht]
    \centering
    \includegraphics[width=0.32\textwidth]{
        ./figs-mc-emulation/emulate-l0global-tis/l0_global_tis_eff_bin_idx_dir.pdf
    }
    \includegraphics[width=0.32\textwidth]{
        ./figs-mc-emulation/emulate-l0global-tis/l0_global_tis_eff_log_pt_dir.pdf
    }
    \includegraphics[width=0.32\textwidth]{
        ./figs-mc-emulation/emulate-l0global-tis/l0_global_tis_eff_log_pz_dir.pdf
    }
    \caption[Check that L0Global TIS is portable among signal and normalization MC modes]{
        Check that L0Global TIS is portable among signal and normalization MC modes

        The $p_T$ and $p_z$ momenta are MC true momenta of the $B$ meson.
        Bin index represents the index from an unrolled 2D
        true-$p_T$-true-$p_z$ histogram.
        Efficiencies are evaluated according to \cref{eqn:eff-l0global-tis},
        as the MC samples used here are \emph{not} filtered on trigger decision,
        so $N_\text{selected}$ can be evaluated directly.

        Efficiencies are in good agreement across all bins.
    }
    \label{fig:l0global-tis-portable}
\end{figure}


\subsection{Correlation between L0Hadron TOS and L0Global TIS}
\label{sec:emulation-for-to-mc:correlation-tos-tis}

L0 triggers have a global event cut on the number of SPD hits.
For all but the DiMuon trigger\footnote{
    Ignore Muon high \pt line for now.
}, the cut is placed at 450; for DiMuon, 900,
as listed in \cite{LHCb-INT-2019-025}.
For events that are both \emph{TIS and TOS}, a non-trivial correlation is
introduced by this cut, because it affects the L0 TIS efficiencies
inconsistently:
On a small patch of the signal \B phase space,
for events that are TIS on \emph{non-DiMuon}, TIS and TOS efficiency is:

\begin{equation}
    \epsilon_\text{TIS and TOS} =
        \epsilon_\text{L0-non-DiMuon TIS} \cdot \epsilon_\text{L0Hadron TOS}
\end{equation}

For L0DiMuon TIS only, however, its efficiency is affected by the SPD cut:
\begin{align}
    \epsilon_\text{L0DiMuon TIS \& nSPDhits < 450} & \neq
        \epsilon_\text{L0DiMuon TIS} \\
    \Rightarrow \epsilon_\text{TIS and TOS} & =
            \epsilon_\text{L0DiMuon TIS \& nSPDhits < 450} \cdot
            \epsilon_\text{L0Hadron TOS} \\
        & \neq
        \epsilon_\text{L0DiMuon TIS} \cdot \epsilon_\text{L0Hadron TOS}
\end{align}

Therefore:
\begin{align}
    \epsilon_\text{TIS and TOS} & =
        (\epsilon_\text{L0-non-DiMuon TIS} +
         \epsilon_\text{DiMuon TIS \& nSPDhits < 450}) \cdot
         \epsilon_\text{L0Hadron TOS} \\
    & \neq
        \underbrace{\epsilon_\text{L0Global TIS}}_{
            = \epsilon_\text{L0-non-DiMuon TIS} +
              \epsilon_\text{L0DiMuon TIS}
         } \cdot\; \epsilon_\text{L0Hadron TOS}
\end{align}

To minimize the correlation, a $\text{nSPDhits} < 450$ is applied on all data
samples\footnote{
    It is applied on the $\jpsi K$ sample as well, the one used to obtain
    binned L0Global TIS efficiencies.
    For TO MC, there is no nSPDhits variable.
}, which removes about an additional 4.4\% of candidates for both \Dz and
\Dstar channel.
After the SPD cut is applied, it is checked that
$\epsilon_\text{TIS or TOS} \approx \epsilon_\text{TIS} + \epsilon_\text{TOS} -
\epsilon_\text{TIS} \cdot \epsilon_\text{TOS}$
in \cite{LHCb-INT-2019-025},
that is, TIS efficiencies can be considered as uncorrelated from
TOS efficiencies.


\subsection{Emulation of Hlt1TrackMVA and Hlt1TwoTrackMVA}

The event selection requires that either $K$ or $\pi$ is TOS on Hlt1TrackMVA,
or \Dz is TOS on Hlt1TwoTrackMVA.
The single track Hlt1TrackMVA requires a track that has score based on
$p_T$ and \ipChiSq to be above some threshold;
the two track Hlt1TwoTrackMVA feeds the $p_T$ and \ipChiSq of both tracks to
a MatrixNet MVA, requiring the two-track combination to pass MVA selection.

All required variables are present in TO MC, and are extracted by adding
a tool\footnote{
    Named \lstinline{RelInfoHLT1Emulation}, available at
    \url{https://github.com/umd-lhcb/TrackerOnlyEmu/tree/master/davinci/Phys},
    courtesy of the authors of \cite{LHCb-INT-2019-025}.
} to event reconstruction.

The HLT1 triggers also require events to pass Global Event Cuts (GEC).
There is also an online-offline difference:
The real HLT1 triggers uses VELO-TT tracks, whereas the emulation can use
VELO tracks only, therefore, additional cuts and a 4.2\% penalty factor are
imposed to account for the differences.
The HLT1 trigger cuts and GEC are listed in \cite{LHCb-INT-2019-025}.
The emulated HLT1 are in good agreement with the real response in FullSim, as
can be seen in \cref{fig:hlt1-trackmva-emu,fig:hlt1-twotrackmva-emu}.

% Generated in /lhcb-ntuples-gen/studies/plot-trigger_emulation, run the script:
%   trigger_emulation.sh
\begin{figure}[ht]
    \centering
    \includegraphics[width=0.32\textwidth]{
        ./figs-mc-emulation/emulate-hlt1/b_Hlt1TrackMVA_TOS_mmiss2.pdf
    }
    \includegraphics[width=0.32\textwidth]{
        ./figs-mc-emulation/emulate-hlt1/b_Hlt1TrackMVA_TOS_el.pdf
    }
    \includegraphics[width=0.32\textwidth]{
        ./figs-mc-emulation/emulate-hlt1/b_Hlt1TrackMVA_TOS_q2.pdf
    }

    \includegraphics[width=0.32\textwidth]{
        ./figs-mc-emulation/emulate-hlt1/b_Hlt1TwoTrackMVA_TOS_mmiss2.pdf
    }
    \includegraphics[width=0.32\textwidth]{
        ./figs-mc-emulation/emulate-hlt1/b_Hlt1TwoTrackMVA_TOS_el.pdf
    }
    \includegraphics[width=0.32\textwidth]{
        ./figs-mc-emulation/emulate-hlt1/b_Hlt1TwoTrackMVA_TOS_q2.pdf
    }

    \caption{
        Emualted HLT1 triggers vs. real response in FullSim. For \Bp.
        The Hlt1TrackMVA agree very well, whereas Hlt1TwoTrackMVA still has a
        constant, albeit small ($\sim\!2.25$\%), discrepancy.
    }
    \label{fig:hlt1-trackmva-emu}
\end{figure}

\begin{figure}[ht]
    \centering
    \includegraphics[width=0.32\textwidth]{
        ./figs-mc-emulation/emulate-hlt1/b0_Hlt1TrackMVA_TOS_mmiss2.pdf
    }
    \includegraphics[width=0.32\textwidth]{
        ./figs-mc-emulation/emulate-hlt1/b0_Hlt1TrackMVA_TOS_el.pdf
    }
    \includegraphics[width=0.32\textwidth]{
        ./figs-mc-emulation/emulate-hlt1/b0_Hlt1TrackMVA_TOS_q2.pdf
    }

    \includegraphics[width=0.32\textwidth]{
        ./figs-mc-emulation/emulate-hlt1/b0_Hlt1TwoTrackMVA_TOS_mmiss2.pdf
    }
    \includegraphics[width=0.32\textwidth]{
        ./figs-mc-emulation/emulate-hlt1/b0_Hlt1TwoTrackMVA_TOS_el.pdf
    }
    \includegraphics[width=0.32\textwidth]{
        ./figs-mc-emulation/emulate-hlt1/b0_Hlt1TwoTrackMVA_TOS_q2.pdf
    }

    \caption{
        Emualted HLT1 triggers in \Bz have similar level of agreement to real
        response compared to that of \Bp.
    }
    \label{fig:hlt1-twotrackmva-emu}
\end{figure}


\subsection{Emulation of Hlt2XcMuXForTauB2XcMu}

The reconstructed \B is required to TOS on Hlt2XcMuXForTauB2XcMu. All selection
variables are available directly, and the selections are applied directly in
event reconstruction.


\section{Particle identification emulation}
\label{sec:emulation-for-to-mc:pid}

Tracker-Only MC has RICH and calorimeters, sub-detectors responsible for
particle identification (PID), set to passive material, thus no PID cut can be
applied directly on MC.
Therefore, PID cuts for MC are emulated as efficiencies (binned in some
variables, typically $p$, $\eta$, and nTracks) and applied as weights.

Most of PID efficiencies are obtained with \pidcalib, except for ghost and
$e$ \UBDT efficiencies. The ghost PID efficiencies are obtained from the bachelor
$\mu$ tracks truth-matched to ghost in
2016 inclusive $J/\psi K$ MC, % FIXME: MC IDs
and the $e$ \UBDT efficiency, a \emph{conditional efficiency}, is obtained from
$\mu$ track truth-matched to $e$ in 2016 $D^*\mu$ MC.

It is worth noting that \pidcalib does not populate under and overflow bins,
therefore an custom binning scheme, extended from the official scheme, is used
for \pidcalib efficiencies to ensure a wider coverage.

\techlink{appx:tech:obtain-pid-eff}
