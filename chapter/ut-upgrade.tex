\chapter{Upgrade of the Upstream Tracker}
\label{ref:ut}

Located in the fringing fields in front of the LHCb dipole Magnet,
the Upstream Tracker (UT) plays an important role in improving the tracking
quality and speeding up trigger decisions:
Combining hits from VELO, SciFi, and UT provides the best \pt resolution
and reduce the rate of random combinations from VELO and SciFi to form
ghost tracks to about one fourth.
Furthermore, combining hits from VELO and UT removes low-\pt tracks and narrows
down the hit search window in SciFi, allowing for a faster track reconstruction
which leads to faster trigger decisions.

The University of Maryland group is responsible for the design, testing, and
commissioning of the readout electronics, and electronics to supply and regulate
power;
these are discussed in \cref{ref:ut:overview},
together with the sensors.
The author is responsible for the testing and quality assurance of the
electronic board for both readout and control of the UT detector;
the UT data acquisition described in \cref{ref:ut:daq};
the control and configuration is described in \cref{ref:ut:ctrl}.


\section{Overview of the UT detector}
\label{ref:ut:overview}

\subsection{Stave}


\subsection{PEPI}


\subsection{LVR}


\section{Data acquisition in UT}
\label{ref:ut:daq}

\subsection{More on DCB}


\subsection{TELL40 the event builder}


\subsection{Data transmission path}


\section{Control system of UT}
\label{ref:ut:ctrl}


\subsection{Timing and Fast Control (TFC)}


\subsection{Slow control for detector initialization and monitoring}
