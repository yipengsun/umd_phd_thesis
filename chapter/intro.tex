\chapter{Introduction}
\label{ref:intro}

The standard model of particle physics (SM)
is the quantum field theory describing interactions of elementary particles,
namely quarks, leptons, and gauge bosons.
It has been successful in explaining many experimental results, such as parity
violation in weak interaction,
as well as predicting the existence of new particles, such as the Higgs boson.
%%%%
However,
there are experimental observations that cannot be explained by the SM,
such as the existence of dark matter
which has no viable candidate in SM.
These observations demand new physics (NP).
One way to probe and constrain NP is through precision measurements
in which observables associated with certain processes are measured very
precisely and compared with predictions from SM.

One such process is the semileptonic decay of the \B mesons:
$B \rightarrow D^{(*)} \ellm \neulb$,
where \ellm stands for a charged lepton which can be an electron $e^-$,
a muon \mun, or a tau \taum.
SM predicts lepton flavor universality (LFU),
that is, leptons participate in SM processes with the same strength
except for Higgs mechanism through which the leptons acquire different masses.
Therefore,
the ratios\footnote{
    Throughout the thesis, charge conjugation is assumed.
    Also, the convention $\hbar = c = 1$ is used unless specified explicitly.
} of branching fractions \RD and \RDst,
collectively referred as \RDX, defined as:
\begin{equation}
    \RDX \equiv \frac{\BFDTau}{\BFDMu}
\end{equation}
will not be 1 only because \taum and \mun have different masses.
Thus, the ratios can serve as a probe to LFU:
a significant deviation (often chosen to be at $5\sigma$) from SM prediction is
an indication of LFU violation,
which may provide hints and constraints to NP.

On a technical note,
it is more advantageous to measure the ratios instead of absolute
branching fractions because the ratios allow cancellations of many
theoretical and experimental parameters,
for example $V_{cb}$ and muon reconstruction efficiency,
making the measurement more precise.

Since 2012, tensions have been reported between measured \RDX and SM
predictions from \babar, \belle, and LHCb.
Currently, the tension stands at about $3 \sigma$ after averaging all
measurements,
as shown in \cref{fig:hflav}.
In 2015, the LHCb collaboration reported its first result on \RDst
based on LHCb run 1 (2011--2012) data
\cite{PhysRevLett.115.111803},
overcoming the difficulties of approximating key fit variables
as they are not directly computable in a hadron collider experiment
and demonstrating the viability of charactering background decays through
several control samples.
Recently, a follow-up measurement (paper in preparation),
labelled as \emph{LHCb22} in \cref{fig:hflav},
reported a simultaneous extraction of \RD and \RDst,
as opposed to \RDst only,
with the same run 1 data,
based on the techniques developed in the 2015 result but with
a better correction procedure for Monte-Carlo simulation samples,
a novel approach for characterization of the particles misidentified as muon,
and an update of various form factor models used in the simulation.
Since late 2018,
the author has been working on updating the LHCb \RDX measurement with the run 2
(2016--2018) data by porting and refining procedures developed in the run 1
pathfinder analyses and solving additional challenges imposed by the much larger
size of the run 2 dataset.
The first part of the thesis is about a preliminary update of the LHCb \RDX
measurement with LHCb 2016 data,
with the understanding that the developed procedure should work for all 3 years
of run 2 data.

The LHCb detector is undergoing an upgrade currently\footnote{
    Defined as Nov 2022 throughout the text.
} which is expected to greatly increase
the readout rate of the detector and removes the hardware triggers along
with the limitations that come with them.
We the University of Maryland group are actively participating in the upgrade
of the Upstream Tracker (UT), a part of the tracking system.
The author is deeply involved in the design and development of the UT readout
electronics.
The second part of the thesis will provide an overview of the LHCb upgrade, with
a focus on the UT detector and its readout system.

\begin{figure}[!htb]
    \centering
    \includegraphics[width=0.85\textwidth]{./figs-intro/hflav_2022_preliminary.pdf}
    \caption{
        World average of \RD and \RDst, as of Nov 2022.
        Recent result from LHCb run 1 measurement is displayed as
        \emph{LHCb 22}.
        The plot is taken from Heavy Flavor Averaging Group
        \cite{Amhis:2022mac}.
    }
    \label{fig:hflav}
\end{figure}
