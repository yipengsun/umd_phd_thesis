\chapter{Introduction}

The standard model of particle physics (SM)
is the quantum field theory describing interactions of elementary particles,
namely quarks, leptons, and gauge bosons.
It has been successful in explaining many experimental results, such as parity
violation in weak interaction,
as well as predicting the existence of new particles, such as the Higgs boson.
%%%%
However,
there are experimental observations that cannot be explained by SM,
such as the existence of dark matter
which has no viable candidate in SM.
These observations demand new physics (NP).
One way to probe and constrain NP is through precision measurements
in which observables associated with certain processes are measured very
precisely and compared with predictions from SM.

One such process is semileptonic decay of the \B mesons:
$B \rightarrow D^{(*)} \ellm \neulb$,
where \ellm stands for a charged lepton which can be an electron $e^-$,
a muon \mun, or a tau \taum.
SM predicts lepton flavor universality (LFU),
that is, leptons participate in SM processes with the same strength
except for Higgs mechanism through which the leptons acquire different masses.
Therefore,
the ratios\footnote{
    Throughout the thesis, charge conjugation is assumed.
    Also, the convention $\hbar = c = 1$ is used unless specified explicitly.
} of branching fractions \RDX, defined as:
\begin{equation}
    \RDX \equiv \frac{\BFDTau}{\BFDMu}
\end{equation}
will not be 1 only because \taum and \mun have different masses,
which serves as a probe to LFU.
Note that it is more advantageous to measure the ratios instead of absolute
branching fractions because the latter depends on additional SM parameters,
for example $V_{cb}$,
which add another source of uncertainty;
whereas the former (\RDX) have many parameters, both theoretical and
experimental, cancelled for they are shared between the nominator and
denominator, making the measurement more precise.

Since 2012, tensions have been reported between measured \RDX and SM predictions,
as shown in \cref{fig:hflav}.
Some NP has been excluded to high CL in certain paramter space.
The LHCb collaboration reported its latest measurement on \RDX based on LHCb run
1 (2011--2012) data very recently (color in fig).
This thesis is about a preliminary update of the LHCb run 1 measurement with
2016 data.

\begin{figure}[htb]
    \centering
    \includegraphics[width=0.6\textwidth]{./figs-intro/hflav_2022_preliminary.png}
    \caption{
        World average of \RD and \RDst, as of Nov 2022.
        Preliminary result from LHCb run 1 measurement is displayed as
        \emph{LHCb 22}.
    }
    \label{fig:hflav}
\end{figure}
