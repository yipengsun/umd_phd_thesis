\chapter{Systematic uncertainties}
\label{ref:sys-uncert}

A breakdown of all evaluated uncertainties is listed in
\cref{tab:sys-uncert}.
The corresponding uncertainties from the run 1 analysis evaluated with
the \HistFactory fitter is also provided as reference.
Not all uncertainties from run 1 are evaluated due to lack of time;
these are omitted from the table.

\begin{table}[htb]
    \caption{
        Evaluated systematic uncertainties.
        Run 1 uncertainties are provided as a reference inside parentheses.
        All uncertainties are absolute, not relative.
        The additive uncertainties are sorted in descending order by their run 1
        significance.
        Currently the MC statistical uncertainties are not taken into account.
    }
    \label{tab:sys-uncert}
    \centering
    \small
    \begin{tabular}{r | c | c }
        \toprule
        {\bf Source} & {\bf $\RD \times 10^3$} &
                       {\bf $\RD \times 10^3$} \\
        \midrule
        %%%%
        $B \rightarrow D^{(*)}\ell\neulb$ form factors &
        (0.58) & (2.37) \\
        %%%%
        $B \rightarrow D^{**}\ell\neulb$ form factors &
        (0.78) & (1.01) \\
        %%%%
        Control sample shape params &
        (0.87) & (4.36) \\
        %%%%
        $DD$ model dependence &
        (0.63) & (0.74) \\
        %%%%
        $B \rightarrow D^{**}\tau\neutb$ bkg &
        (0.17) & (0.30) \\
        %%%%
        $B \rightarrow D^{(*)} D^{**}_s (\rightarrow \tau\neutb) X$ bkg &
        (0.25) & (1.21) \\
        %%%%
        \muon misID unfolding algorithm &
        (0.74) & (1.19) \\
        %%%%
        Coulomb correction to $\mathcal{R}(\Dstarp)$ vs. $\mathcal{R}(\Dstarz)$ &
        (0.17) & (0.3) \\
        %%%%
        \muon misID decay-in-flight correction &
        (0.06) & (0.16) \\
        %%%%
        Comb. bkg. shape &
        (0.03) & (0.18) \\
        %%%%
        Vertex resolution correction &
        (0.03) & (0.21) \\
        %%%%
        Data/MC corrections (add.) &
        (0.40) & (0.75) \\
        %%%%
        \midrule
        Data/MC corrections (mul.) &
        \makecell{$? \times \RDst$ \\ ($1.16 \times \RDst$)} &
        \makecell{$? \times \RDst$ \\ ($0.91 \times \RDst$)} \\ % FIXME: Really? Both R(D*)?
        %%%%
        $\mathcal{B}(\taum \rightarrow \mun\neumb\neut)$ &
        0.23 & 0.23 \\
        \midrule
        %%%%
        Sys. + stats. &
        ? & ? \\
        \bottomrule
    \end{tabular}
\end{table}

The remainder of the chapter is dedicated to explain the evaluation method
for each tabulated systematic uncertainty.
These uncertainties can be loosely classified into the following groups:
Form factors (\cref{sys-ff-d0-dst,sys-ff-dstst}),
under-constrained $\tau$ components (\cref{sys-tau-dstst,sys-tau-ddx}),
uncertainties on the modelling of the background components
(\cref{sys-model-ctrl,sys-model-ddx,sys-model-dif,sys-model-comb}),
uncertainties on employed algorithms (\cref{sys-algo-misid}),
theoretical corrections not included in the nominal fit
(\cref{sys-theory-coulomb}),
and uncertainties on data/MC corrections
(\cref{sys-cor-vtx,sys-cor-rwt}).


%%%%%%%%%%%%%%%%%%%%%%%%%%%%%%%%%%%%%%%%%%%%%%%%%%%%%%%%%%%%%%%%%%%%%%%%%%%%%%%%
\section{$B \rightarrow D^{(*)}\ell\neulb$ form factors}
\label{sys-ff-d0-dst}

The uncertainties due to $D^(*)$ form factors can be determined in the following
manner:

\begin{enumerate}
    \item Perform a fit with nominal setting. Say the fitted value for
        $\RD = V\;{+\sigma_+}\;{-\sigma_-}$.
    \item Fix the variations ($\alpha$ parameters) associated with \Dz and
        \Dstar form factor parameters at best fitted value.
        This removes the constrains due to these $\alpha$ parameters in the
        likelihood thus
        eliminates all uncertainties due to these parameters.
    \item Repeat the fit with the $\alpha$ parameters fixed at the best fitted
        value.
        Say now the fitted value for
        $\RD = V'\;{+\sigma'_+}\;{-\sigma'_-}$.
    \item The systematic uncertainty due to \Dz and \Dstar form factors are
        obtained from subtracting uncertainties of $V, V'$ in quadrature.
        More specifically, due to the asymmetric nature of the error,
        the systematic uncertainty is calculated with
        \cref{eqn:sys-uncert-sub-quad}:

        \begin{equation}
            \sigma_\text{sys} = \sqrt{\sigma_+ \sigma_- - \sigma'_+ \sigma'_-}
            \label{eqn:sys-uncert-sub-quad}
        \end{equation}
\end{enumerate}


\section{$B \rightarrow D^{**}\ell\neulb$ form factors}
\label{sys-ff-dstst}

The procedure is identical to that in \cref{sys-ff-d0-dst}; the only difference
being: Now the fixed parameters are $D^{**}$ form factors.


%%%%%%%%%%%%%%%%%%%%%%%%%%%%%%%%%%%%%%%%%%%%%%%%%%%%%%%%%%%%%%%%%%%%%%%%%%%%%%%%
\section{$B \rightarrow D^{**}\tau\neutb$ background}
\label{sys-tau-dstst}

As discussed in \cref{tmpl:dstst},
the tauonic yields are not well-constrained by external measurements, and
are floated with a Gaussian constraint for each fit channel.
The systematic uncertainty introduced by the Gaussian constrains
is evaluated based on differences between floating and fixing these parameters
at best fit values, as described in \cref{sys-ff-d0-dst}.


\section{$B \rightarrow D^{(*)} D^{**}_s (\rightarrow \tau\neutb) X$ background}
\label{sys-tau-ddx}

The situation is similar to \cref{sys-tau-dstst}.
The uncertainty on the \tauon vs. \muon relative fractions,
discussed in \cref{ref:fit:tmpl:ddx},
is evaluated with results from floating vs. fixing at best values.


%%%%%%%%%%%%%%%%%%%%%%%%%%%%%%%%%%%%%%%%%%%%%%%%%%%%%%%%%%%%%%%%%%%%%%%%%%%%%%%%
\section{Control sample shape parameters}
\label{sys-model-ctrl}


\section{$DD$ model dependence}
\label{sys-model-ddx}

In the nominal fit, two variations on $m_{DD^{*}}$,
a key proxy variable to 3-body decay phase space,
are introduced on $DDX$ templates to allow for a data-driven approach to
determine the shape of these templates, as described in
\cref{ref:fit:var:ddx}.

To estimate the dependence on the choice of proxy variable to the phase space,
alternative proxy variables, namely $m_{DK}$ and $p_D$ in the \B rest frame,
are deformed linearly and quadratically and the nominal $DDX$ templates are
replaced with these alternative variations and a refit performed.
A profile likelihood study then determine the uncertainty of \RD and \RDst
based on a likelihood profile study\footnote{
    For more details, see APPX % TODO: List details for profile likelihood study
} of both the nominal and the alternative variations.

The variations of $m_{DK}$ take the same form as
\cref{eqn:dal-var-lin,eqn:dal-var-quad}, with $m_{DD^{(*)}}$ replaced by
$m_{DK}$.
The mass limits are also replaced:
$m_\text{min} = m_{D^{(*)}} + m_K$, $m_\text{max} = m_B - m_D^{(*)}$.
Similarly, for variations of $p_D$, the min and max are computed based on the
phase space limit of a 3 body decay.


\section{\muon misID decay-in-flight correction}
\label{sys-model-dif}

As discussed in \cref{ref:fit:var:misid-dif},
the DiF effect is included as a shape variation to allow the fit to determine
the strength of the DiF effect.
The uncertainty due to this variation is evaluated with results
from floating vs. fixing at best value for the associated $\alpha$ parameter.


\section{Combinatoric background shapes}
\label{sys-model-comb}

The nominal \BComb templates include a linear correction depend on $m_B$,
which is discussed in \cref{ref:fit:tmpl:comb}.
However, the actual correction is of a form $e^{-\lambda m_B}$
(where $\lambda$ is fitted from data;
the exponential fits are displayed side-by-side with the linear fits in
\cref{fig:b-comb-d0,fig:b-comb-dst}.)
and the linear correction is only the leading order approximation of the
exponential.

The uncertainty on the linear correction factor is quantified by
repeating the fit with the nominal \BComb templates substituted with
the ones generated with the exponential correction.
The quadrature difference between \RDX is taken as the associated systematic
uncertainty.
The exponential correction is \emph{not taken as the nominal} because the
existing data cannot resolve $\lambda$ to high precision.


%%%%%%%%%%%%%%%%%%%%%%%%%%%%%%%%%%%%%%%%%%%%%%%%%%%%%%%%%%%%%%%%%%%%%%%%%%%%%%%%
\section{\muon misID unfolding algorithm}
\label{sys-algo-misid}

This is a work in progress.


%%%%%%%%%%%%%%%%%%%%%%%%%%%%%%%%%%%%%%%%%%%%%%%%%%%%%%%%%%%%%%%%%%%%%%%%%%%%%%%%
\section{Coulomb correction to $\mathcal{R}(\Dstarp)$ vs. $\mathcal{R}(\Dstarz)$}
\label{sys-theory-coulomb}


%%%%%%%%%%%%%%%%%%%%%%%%%%%%%%%%%%%%%%%%%%%%%%%%%%%%%%%%%%%%%%%%%%%%%%%%%%%%%%%%
\section{Vertex resolution correction}
\label{sys-cor-vtx}


\section{Data/MC corrections}
\label{sys-cor-rwt}
