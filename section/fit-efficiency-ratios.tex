\section{Efficiency ratios}
\label{ref:fit:eff}

Similar to the run 1 \RDX pathfinder analysis,
this analysis employs four efficiency ratios to relate yields between
various signal and normalization decay modes in both \Dz and \Dstar channels
for extracting \RDX from fitted values and cross-constraining the
$B \rightarrow D^{*0/*+}\ellm\neulb$ feed downs in the \Dz channel.
These ratios are:

\paragraph{For \RD}
There is only one ratio for $\RD \equiv \mathcal{R}(\Dz)$,
a free-floating parameter of interest,
as this parameter occurs only in the \Dz channel:
\begin{equation}
    \tilde{\eta}_{\Dz} = \frac{
        \mathcal{B}(\taum \rightarrow \mun\neumb\neut)
        \epsilon{(\Bm \rightarrow \Dz \taum (\rightarrow \mun\neumb\neut) \neutb)}
    }{
        \epsilon({\Bm \rightarrow \Dz \mun \neumb})
    }
\end{equation}
where $\epsilon(p)$ denotes the selection efficiency of the decay process $p$
and is obtained from the corresponding MC sample.
The efficiency ratio
$\tilde{\eta}_{\Dz}$ is used in conjunction with \RD to constrain the signal
yield in this channel.
More details regarding its use will be provided in \cref{tmpl:sig-norm}.

\paragraph{For \RDst}
The parameter of interest \RDst is defined as
There is one ratio for
