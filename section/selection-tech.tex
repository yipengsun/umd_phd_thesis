\section{Additional technologies used in selection}
\label{ref:sel:tech}

\subsection{Isolation BDT}
\label{ref:sel:tech:iso-bdt}

To predict the possibility of a charged track originating from the \B vertex,
a multivariate BDT, referred as ``isolation BDT'', is used.
The BDT is conceptually identical to the one used in
\cite{LHCb-ANA-2020-056} but with the BDT re-trained based on LHCb run 2
simulation.

The isolation BDT loops over all charged tracks in the event,
providing isolation scores for each of them where
higher score implies higher probability of coming from the vertex, named as
``anti-isolated''.
Three tracks with highest isolation scores\footnote{
    The ones that are most likely to be anti-isolated.
} are saved in the output file, in descending order\footnote{
    So the first one is the most anti-isolated one among all charged tracks
    in the event.
}.


\subsection{Veto of overlapping candidates}
\label{ref:sel:tech:veto}

% See this note: https://github.com/umd-lhcb/rdx-run2-analysis/blob/master/docs/cuts/Dst_veto_in_D0.md
% TODO: The number need to be updated
Since the selection of a \Dstar\muon pair always requires the existence
of a \Dz\muon pair,
there is about 35\% of events in the signal sample\footnote{
    Defined in \cref{ref:sel:data:iso}.
} of the \Dstar channel leaks into \Dz channel.
This is likely due to the fact that slow \pion typically has poor tracking
resolution (small \ipChiSq), making the \Dz\pion vertex of poor quality which
makes the reconstruction fail and \Dstar partially reconstructed as \Dz.
To make \Dstar and \Dz channel more orthgonoal

To veto the overlapping candidates between \Dz and \Dstar channel,
a tool\footnote{
    Named \texttt{TupleToolApplyIsolationVetoDst}, which can be found at
    \techurllink{https://github.com/umd-lhcb/TupleToolSemiLeptonic/blob/master/Phys/TupleToolSemiLeptonic/src/TupleToolApplyIsolationVetoDst.cpp}{github/umd-lhcb/TupleToolSemiLeptonic}
}.
is added to the \davinci reconstruction sequence, which processes all tracks
in the event with the following procedure:

\begin{enumerate}
    \item Denote the track as $t$
    \item Refit a vertex from \Dz and $t$.
    \item Compute $\Delta m_\text{veto} \equiv m_{\Dz t} - m_\Dz$
    \item Test if $\Delta m_\text{veto} \in [140~\text{GeV}, 160~\text{GeV}]$.
        If it is, record $\Delta m_\text{veto}$ and the isolation BDT
        score of this track.
\end{enumerate}

From all recorded BDT scores and $\Delta m_\text{veto}$, the
$\Delta m_\text{veto}$ of the tracks with top two BDT scores are saved.
It is then filtered offline (listed in \cref{tab:offline-cut-d0})
that these two tracks,
the best slow \pion candidates,
are \emph{incompatible} with
forming a \Dz\pion vertex with a mass close to \Dstar PDG mass.

Note that this tool does not rank tracks based \emph{solely} on the isolation,
because the poor tracking resolution of the slow \pion implies that the track is
\emph{not guaranteed} to have a large isolation score, as it maybe more
compatible to be originating from PV, instead of the \B vertex.
Ranking tracks that fall within the $[140~\text{GeV}, 160~\text{GeV}]$
mass window based on their BDT scores
makes the veto procedure more robust, as reported in \cite{LHCb-ANA-2020-056}.

In addition, an alternative mass hypothesis is tested
(offline, also listed in \cref{tab:offline-cut-d0}), where the reconstructed
\muon is treated as a \pion,
and $\Delta m_\text{alt hypo} = m_{\Dz\muon_\text{as \pion}} - m_\Dz$
is computed\footnote{
    By using the 3-momentum of the \muon, while using $m_\pion$ as the invariant
    mass of the track.
} and required to be inconsistent with \Dstar PDG mass.

The veto procedures are applied on all samples of the \Dz channel only.


\subsection{Muon PID}
\label{ref:sel:tech:ubdt}

To maintain minimum bias on \muon \pt, a multivariate BDT,
referred as \UBDT, using uBDT method in TMVA class,
is trained to reject misidentified \muon while keeping rejection
efficiency flat on \pt,
as described in \cite{LHCb-ANA-2020-056}.

The \UBDT has been re-trained on LHCb run 2 MC\footnote{
    The updated project can be found at
    \techurllink{https://github.com/umd-lhcb/MuonBDTPid}{github/umd-lhcb/MuonBDTPid}.
}.
There is ongoing effort to study its efficiency against standard run 2 PID
variables.
Early report suggests the flatness on \pt is maintained while rejection
efficiencies relative to existing PID cut are high for all but proton ($p$),
as shown in \cref{fig:ubdt-eff}.

% Generated with:
%   https://github.com/umd-lhcb/pidcalib2, branch efficiency_study
% first, run efficiency_gen/rdx-run2-ubdt-misidgen.sh
% then, go inside scripts folder, run ./plots.py
\begin{figure}[htb]
    \centering
    \includegraphics[width=0.45\textwidth]{./figs-selection/eff_Brunel_PT_up_pidcalib_ubdt_eff.pdf}
    \hspace{1em}
    \includegraphics[width=0.45\textwidth]{./figs-selection/rej_v_eff_unbiased_Brunel_PT.pdf}

    \caption{
        Preliminary \UBDT study.
        Left: \UBDT \muon selection efficiency is flat in \pt, with
        global cut \isMuon \& $\text{\PID{\muon}}\! > 2$,
        and $\UBDT > 0.25$.
        Right: with the same global cut, \UBDT is more effective in rejecting
        \pion than LHCb official \ProbNN{\muon}.
        The \kaon rejection efficiencies are similar between the two;
        the $p$ rejection efficiency is lower for \UBDT, but the absolute
        rejection efficiency is high enough.
    }
    \label{fig:ubdt-eff}
\end{figure}
