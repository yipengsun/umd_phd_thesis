\section{Auxiliary samples}
\label{ref:sel:aux}

A $\Bp \rightarrow \jpsi \Kp$ sample is used to derive $B$ kinematic and
multiplicity corrections\footnote{
    The correction procedure is described in \cref{ref:mc-cor:init:jpsi-k}.
} to MC samples.
The data sample is selected from \mup\mun\Kp final states, with the \mup\mun
pair at \jpsi resonance,
and the invariant mass of the 3-particle final state around the reference mass
of \Bp.
The offline selection cuts for $\jpsi\Kp$ are listed in
\cref{tab:cut-jpsik}.
The MC sample used to compare with $\jpsi \Kp$ data is selected with the same
procedure, with PID cuts applied as weights\footnote{
    The weights are obtained from \pidcalib, with a method similar to that in
    \cref{ref:mc-emulation:pid}).

    An extended PID binning range is chosen to cover as much phase space of
    \kaon and \muon as possible.
    There may still be events outside the covering region, in which case the PID
    weights from nearby bin is applied,
    to avoid explicit cuts on the \jpsi\kaon data control sample.
}.
The number of MC events after filtering\footnote{
    This is also referred as ``event on disk''.
} is listed in \cref{tab:add-mc-samples}.

\begin{table}[!htb]
    \caption{Offline cuts on $\jpsi K$ samples.}
    \label{tab:cut-jpsik}
    \centering
    \begin{tabular}{ c | rll}
        \toprule
        {\bf Particle}    & {\bf Variable}               & {\bf Cuts}               \\
        \midrule
        \kaon             & \pt                          & $> 500$ MeV              \\
                          & \PID{$K$}                    & $> 4$                    \\
        \midrule
        \mun, \mup        & \pt                          & $> 500$ MeV              \\
                          & \ipChiSq                     & $> 4$                    \\
                          & \PID{\muon}                  & $> 2$                    \\
        \midrule
        \mun\mup (\jpsi)  & $m_\text{mea}$\parnote{
            $m_\text{mea}$ refers to measured mass, which is the invariant
            mass given by the sum of daughters' four momenta,
            without any topological constraint.
            On the other hand, $m$ is given by a vertex fit,
            which is typically of better quality.
        }                                                & 3060--3140 MeV           \\
                          & \anyChiSq{FD}                & $> 25$                   \\
        \midrule
        \Bp               & $m$                          & 5150--5350 MeV           \\
                          & \anyChiSq{vertex}            & $< 18$                   \\
                          & \ipChiSq                     & $< 12$                   \\
                          & \DIRA                        & $> 0.9995$               \\
        \bottomrule
    \end{tabular}
    \begin{flushleft}
        \parnotes
    \end{flushleft}
\end{table}

A simulated $\Dz \rightarrow \Km \pip$ MC sample,
also listed in \cref{tab:add-mc-samples},
is requested to model the momentum smearing in the misID sample due to
$K, \pi$ decay-in-flight into an authentic \muon,
which is described in \cref{ref:fit:var:misid-dif}.

\begin{table}[!htb]
    \caption{
        Additional MC control samples.
        All samples are full simulations.
    }
    \label{tab:add-mc-samples}
    \centering
    \parnotereset
    \begin{tabular}{l|c|c|c}
        \toprule
        \makecell{\centering\bf MC ID} & {\bf Decay mode} & {\bf Sim08}\parnote{
            This is a mixture of Sim08e (\pythia{6}, \pythia{8})
            and Sim08i (\pythia{8}).
        } & {\bf Sim09k} \\
        \midrule
        22162000\parnote{
            Currently the 2012 MC are still used for this analysis.
            It is planned to update to a run 2 MC at a future time.
        } & $\Dz \rightarrow K \pi$ & 3M & -- \\
            12143001 & $\Bp \rightarrow \jpsi \Kp$ & -- & 4M \\
        \bottomrule
    \end{tabular}
    \begin{flushleft}
        \parnotes
    \end{flushleft}
\end{table}
