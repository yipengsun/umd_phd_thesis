\section{Signal (ISO) and control background (1OS, 2OS, DD) skims and the isolation BDT}
\label{ref:sel:skims}

The signal and control skims (sub-samples) are separated by inspecting if there
are additional charged tracks in the event that are likely originating from the
same decay vertex (typically a $B$ decay vertex, and will be referred as so in
later text) as the reconstructed \DXmu pair;
if there is no such track,
the \DXmu pair is considered \emph{isolated} and is placed in the signal skim;
otherwise it is categorized as one of the control skims.

\paragraph{The isolation BDT}
A multivariate BDT,
referred as the ``isolation BDT'',
conceptually identical to the one used in \cite{LHCb-ANA-2020-056},
is trained on LHCb run 2 simulation
with training variables listed in \cref{tab:iso-bdt-input}
to provide a score-based estimation on if a charged track is originating from
the \B (\DXmu) vertex or not.


\begin{table}[!htb]
    \centering
    \caption{Training variables for the isolation BDT.}
    \label{tab:iso-bdt-input}

    % FIXME: Confirm these w/ Phoebe
    \begin{tabularx}{0.8\linewidth}{r|X}
        \toprule
        \textbf{Variable} & \textbf{Comment} \\
        \midrule
        PV \ipChiSq &
        Increase in the $\chi^2$ of the primary $pp$ vertex (PV) fit if the
        track is included.
        Higher $\chi^2$ implies that the track is less likely to coming from the
        fitted vertex.
        \\
        %%%%
        SV \ipChiSq &
        Increase in the $\chi^2$ of the secondary $D^{(*)}$ vertex (SV) fit if
        the track is included. \\
        %%%%
        track \pt & - \\
        track opening &
        The cosine of the angle between the track and the flight direction
        of the \DXmu pair.
        \\ %%%%
        \midrule
        \anyChiSq{FD} &
        The significance of the flight distance fit with this track included. \\
        %%%%
        $\Delta\anyChiSq{FD}$ &
        The difference in \anyChiSq{FD} with and without the track included. \\
        \bottomrule
    \end{tabularx}
\end{table}


The isolation BDT is applied on all charged long tracks in the event,
providing isolation scores for each of them where higher score implies higher
probability of coming from the \B vertex,
referred to as ``anti-isolated''.
Three tracks with highest isolation scores
(the ones that are most likely to be anti-isolated)
are saved in descending order\footnote{
    So the first one is the most anti-isolated one among all charged tracks
    in the event.
}.


Below a brief description for the signal sample and the main physics background
control samples is provided.
The dominate decay mode for each skim is also listed; these skims still contain
a composition of many decay modes.
The actual isolation selections are listed in \cref{tab:skim-cut}.

\begin{itemize}
    \item ISO: $B \rightarrow D^{(*)} \lepton \neulb$, with $\lepton \in \{\muon,
        \tauon\}$.
        This is referred as signal, or ``isolated'', sample.
        It requires that no additional charged track is from the \B vertex
        (in a probabilistic sense, with probability related to the isolation
        score)
        and is compatible with a fully reconstructed \B decay
        (ignoring missing neutrino(s)).
        % TODO: Maybe list yield, and compare to run 1, and include figures?

    \item 1OS: $B \rightarrow D^{**} \lepton \neulb$.
        This control sample, enriched in excited charm states,
        requires one and only one additional anti-isolated charged long track
        that is compatible with a \pion PID hypothesis and has correct charge
        for a $D^{**} \rightarrow D^{(*+)}\pim$ decay.

    \item 2OS: $B \rightarrow D^{**}_H \lepton \neulb$,
        where $H$ stands for ``heavy''.
        This control sample is enriched in highly excited (heavy) charm states,
        which is selected by requiring two and only two anti-isolated \pion-like
        long tracks of opposite charge,
        capable of $D^{**}_H \rightarrow D^{(*)} \pip\pim$ decay.

        % Why?
        This sample also provides an independent selection of
        $B \rightarrow D^{(*)}D X$ backgrounds, where the \pip\pim fit into the
        $X$ category and \kaon escapes isolation detection.

    \item DD: $B \rightarrow D^{(*)}D X$,
        with dominate $D \rightarrow \Kp \mun \neumb X$ sub-decays.
        This is a double-charm ($DD$) control sample,
        which is selected by requiring at least one anti-isolated track,
        a \kaon-like long track\footnote{
            This means the track went through both upstream and downstream
            trackers, and is generally of good tracking quality.
        } and a hard track in the three most anti-isolated tracks\footnote{
            Note that these three requirements can be satisfied by a single
            track.
        }.
\end{itemize}


\begin{table}[!htb]
    \caption{Signal and control sample isolation requirements.}
    \label{tab:skim-cut}
    \centering
    \begin{tabular}{c|rll}
        \toprule
        {\bf Sample}  & {\bf Variable}              & {\bf Cuts}     \\
        \midrule
        ISO           & \isoBDT{1}                  & $< 0.15$       \\
        \midrule
        1OS           & \isoBDT{1}                  & $> 0.15$       \\
                      & \isoBDT{2}                  & $< 0.15$       \\
                      & \isoTrack{1}                & $= 3$\parnote{
                          This means a long track.
                      }                                              \\
                      & $p_1$                       & $> 5$ GeV      \\
                      & $p_{T,1}$                   & $> 0.15$ GeV   \\
                      & \ProbNN{$K_1$}              & $< 0.2$        \\
                      & $Q_1 \cdot \text{PID}_\Dz$\parnote{
                          Apply to \Dz channel,
                          which implies that the anti-isolated \pip can
                          form a \Dstarp with the \Dz.
                      }                             & $> 0$          \\
                      & $Q_1 \cdot \text{PID}_\Dstar$\parnote{
                          Apply to \Dstar channel.
                          Here it is required that the anti-isolated \pim can
                          form a $D^{**0}$ with the \Dstarp.
                      }                             & $< 0$          \\
        \midrule
        2OS           & \isoBDT{1}                  & $> 0.15$       \\
                      & \isoBDT{2}                  & $> 0.15$       \\
                      & \isoBDT{3}                  & $< 0.15$       \\
                      & \isoTrack{1}                & $= 3$          \\
                      & \isoTrack{2}                & $= 3$          \\
                      & {\footnotesize
                         Max$(p_1 \cdot (p_{T,1} > 0.15 \text{ GeV}),
                              p_2 \cdot (p_{T,2} > 0.15 \text{ GeV}))$
                        }
                                                    & $> 5$ GeV      \\
                      & \ProbNN{$K_1$}              & $< 0.2$        \\
                      & \ProbNN{$K_2$}              & $< 0.2$        \\
                      & $Q_1 \cdot Q_2$             & $< 0$          \\
        \midrule
        DD            & \isoBDT{1}                  & $> 0.15$       \\
                      & {\footnotesize$\begin{aligned}
                            \text{Max}(
                            &p_1 \cdot (p_{T,1} > 0.15\text{ GeV}),  \\
                            &p_2 \cdot (p_{T,2} > 0.15\text{ GeV})
                                 \cdot (\text{\isoBDT{2}} > -1.1),   \\
                            &p_3 \cdot (p_{T,3} > 0.15\text{ GeV})
                                 \cdot (\text{\isoBDT{3}} > -1.1)
                            )
                        \end{aligned}$}             & $> 5$ GeV      \\
                      & Max(\ProbNN{$K_{1,2,3}$})   & $> 0.2$        \\
                      & \isoTrack{\text{the one passing $K$ PID requirement}}
                                                    & $= 3$          \\
                      & \isoBDT{\text{the one passing $K$ PID requirement}}
                                                    & $> -1.1$       \\
        \bottomrule
    \end{tabular}
    \begin{flushleft}
        \parnotes
    \end{flushleft}
\end{table}
