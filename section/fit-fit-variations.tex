\section{Fit variations}
\label{ref:fit:var}

In \HistFactory, fit templates are allowed to have bin-by-bin variations,
defined as:

\begin{equation}
    n_\text{var} = n_\text{nom} + \sum_i f_i(n_+, n_-, n_\text{nom}, \alpha)
\end{equation}
where $n$ denote yields in a single bin, and
$n_+, n_-$ are the yields when the variation $\alpha$ is at $\pm 1$.
$f$ is the interpolation/extrapolation function to specify variation
at any value; only
piecewise-linear and quadratic-interpolation-linear-extrapolation
in the \HistFactory are used.

The template variations employed in this analysis are listed in the following
paragraphs.

\subsection{Form factor uncertainties}

To take form factor uncertainties into account, the correlation
matrices of the form factor parameters are transformed
according to \cref{appx:ff-var}, which produces $\pm 1\sigma$ variations
in an orthonormal error eigen basis.

Additional fit templates are generated by reweighting MC samples at the
variations and are included in the fit.
A sample $\pm 1\sigma$ variation of the \Dz\mun templates is listed in
\cref{fig:fit-variations:ff}.

\begin{figure}[htb]
    %\begin{subfigure}{\textwidth}
    %    \caption{
    %        $\Bm \rightarrow \Dz\taum\neutb$ vs.
    %        $\Bzb \rightarrow \Dstarp\taum\neutb$.
    %    }
    %    \label{fig:d0-sig-vs-dst-sig}
    %\end{subfigure}

    \caption{
        \Dz\mun nominal and $\pm 1\sigma$ variation templates for
        parameter $u_1$, which is mostly aligned with
        some param in the BGL parameterization.
    }
    \label{fig:fit-variations:ff}
\end{figure}


\subsection{Data-driven Dalitz-inspired deformations to $DDX$ model}
\label{sec:fit-to-data:fit-tmpl-vars:ddx}

The $DDX$ are poorly understood and the phase space model are known to be
unphysical but no better model is available.
Additional variations are introduced to give more freedom to the fit variables,
allowing for a data-driven approach for the determination of $DDX$ decays.

The variations are inspired by Dalitz plots, where a single scaler decays into
three scaler, and the phase space is uniquely determined by two variables.
Here the invariant mass of $DD$ is chosen as the proxy to deform the phase
space linearly and quadratically, as defined in EQN.


\subsection{\Kstar weight up/down in $DDX$ templates}

An $DDX$ event is considered to contain a \Kstar if
$(p_B - p_{D_\text{1st}} - p_{D_\text{2nd}})^2 > 680$ MeV.
The \Kstar events are weighted up (to 2) and down (to 0) to allow additional
variations for the $DDX$ templates, because the
branching fraction of \Kstar is not precisely known.


\subsection{Data-driven phenomenological deformation to $D_H^{**}$ model}

The $D_H^{**}$ MC samples are generated with ISGW2 form factor model, which
is known to be insufficient in describing data:
Similar to run 1 analysis, the \qSq in the 2OS sample fits poorly with the ISGW2
model.
A deformation based on the true \qSq
in $D_H^{**}$ is introduced to provide more freedom in fit variables,
allowing for a data-driven approach to determine the shapes of these modes.
A comparison between nominal and variated $D_H^{**}$ templates can be seen at.


\subsection{misID decay-in-flight}

The \muon misID templates are subject to \kaon, \pion decay-in-flight (DiF)
effect,
which is not present in the \muon misID control sample.
A DiF-smeared template is produced according to
\cref{sec:data-driven-templates:mu-misid:dif} and included in the fit as a
variation.



