\section{Fit templates}
\label{ref:fit:tmpl}

This selection describes generation of fit templates and comments on why
each of the templates is included.
There are a total of 37 templates in \Dz channel and 24 in \Dstar channel,
with some templates contributing to both, as listed in
\cref{tab:fit-templates-d0,tab:fit-templates-dst}.
The data-driven templates, namely combinatorial background and \muon misID
templates, are discussed first in \cref{ref:fit:tmpl:comb,ref:fit:tmpl:misid}.
The rest are all MC templates and are discussed in the remaining sections.

The MC fit templates are generated by applying the same cuts as described
in \cref{sec:selection}, minus the trigger and PID cuts which are emulated in
\cref{sec:emulation-for-to-mc};
these templates also include weights to change form factor parameterizations
whenever applicable, as described in \cref{sec:form-factors};
in addition, the initial and final reweighting weights are also applied.
All weights applied during MC template generation are listed in
\cref{eqn:mc-wts}, with capping strategy also specified:

\begin{equation}
    w_\text{tot} = \underbrace{\left(
            w_\text{trigger} \cdot w_\text{tracking} \cdot
            w_\text{PID} \cdot w_{\jpsi\kaon}
        \right)}_\text{capped at 10} \;\; \times
        \underbrace{w_\text{form factor}}_{\substack{
            \text{capped at} \\
            \text{50 for \Dz and \Dstar} \\
            \text{10 for $D^{**}$}
        }} \times \;\;
        \underbrace{
            {\textstyle\prod_i} w_\text{final weight,step $i$}
        }_\text{capped at 10}
        \label{eqn:mc-wts}
\end{equation}


% Generated in umd-lhcb/rdx-run2-analysis/fit:
%   with: make tab-fit-templates
%% D0
\begin{table}[htb]
    \caption{Fit templates included in \Dz channel fit.}
    \label{tab:fit-templates-d0}
    \footnotesize
    \centering
\begin{tabular}{lllrr}
\toprule
 {\bf Decay mode}  & {\bf Run 2 \texttt{process}} & {\bf Alias in fitter}  & {\bf Variations}  & {\bf Index} \\
\midrule
 $B^- \rightarrow D^0 \mu^- \overline{\nu}_\mu$                                       & D0Mu                     & Dmu               &            5 &       1 \\
 $\overline{B}^0 \rightarrow D^{*+} \mu^- \overline{\nu}_\mu$                         & DstMu                    & dDstmu            &           10 &       2 \\
 $B^- \rightarrow D^{*0} \mu^- \overline{\nu}_\mu$                                    & Dst0Mu                   & uDstmu            &           10 &       3 \\
 $B^- \rightarrow D^0 \tau^- \overline{\nu}_\tau$                                     & D0Tau                    & Dtau              &            5 &       4 \\
 $\overline{B}^0 \rightarrow D^{*+} \tau^- \overline{\nu}_\tau$                       & DstTau                   & dDsttau           &           10 &       5 \\
 $B^- \rightarrow D^{*0} \tau^- \overline{\nu}_\tau$                                  & Dst0Tau                  & uDsttau           &           10 &       6 \\
 $\overline{B}^0 \rightarrow D_1 \mu \overline{\nu}_\mu$                              & D1ststMu                 & dD1mu             &            3 &       7 \\
 $\overline{B}^0 \rightarrow D_1 (\rightarrow D^0 \pi\pi) \mu \overline{\nu}_\mu$     & D1ststMuD0PiPi           & dD1mu\_pipi       &            3 &       8 \\
 $\overline{B}^0 \rightarrow D^*_2 \mu \overline{\nu}_\mu$                            & D2ststMu                 & dD2mu             &            3 &       9 \\
 $\overline{B}^0 \rightarrow D'_1 \mu \overline{\nu}_\mu$                             & D1pststMu                & dD1pmu            &            2 &      10 \\
 $\overline{B}^0 \rightarrow D^*_0 \mu \overline{\nu}_\mu$                            & D0ststMu                 & dD0mu             &            2 &      11 \\
 $\overline{B} \rightarrow D^{**} (\rightarrow D^{*0} \pi\pi) \mu \overline{\nu}_\mu$ & DststHMuDst0             & Dstzpipimu        &            1 &      12 \\
 $\overline{B} \rightarrow D^{**} (\rightarrow D^* \pi\pi) \mu \overline{\nu}_\mu$    & DststHMuDst              & Dstppipimu        &            1 &      13 \\
 $\overline{B} \rightarrow D^{**} (\rightarrow D^0 \pi\pi) \mu \overline{\nu}_\mu$    & DststHMuD0               & Dpipimu           &            1 &      14 \\
 $B^- \rightarrow D_1^0 \mu \overline{\nu}_\mu$                                       & D1stst0Mu                & uD1mu             &            3 &      15 \\
 $B^- \rightarrow D_1^0 (\rightarrow D^0 \pi\pi) \mu \overline{\nu}_\mu$              & D1stst0MuD0PiPi          & uD1mu\_pipi       &            3 &      16 \\
 $B^- \rightarrow D_2^{*0} \mu \overline{\nu}_\mu$                                    & D2stst0Mu                & uD2mu             &            3 &      17 \\
 $B^- \rightarrow {D'_1}^0 \mu \overline{\nu}_\mu$                                    & D1pstst0Mu               & uD1pmu            &            2 &      18 \\
 $B^- \rightarrow {D^*_0}^0 \mu \overline{\nu}_\mu$                                   & D0stst0Mu                & uD0mu             &            2 &      19 \\
 $\overline{B}_s \rightarrow D_{s2}^* \mu \overline{\nu}_\mu$                         & Ds2Mu                    & sDs2mu            &            3 &      20 \\
 $\overline{B}_s \rightarrow D'_{s1} \mu \overline{\nu}_\mu$                          & Ds1pMu                   & sDs1pmu           &            3 &      21 \\
 $\overline{B}^0 \rightarrow D_1 \tau \overline{\nu}_\tau$                            & D1ststTau                & dD1tau            &            3 &      22 \\
 $\overline{B}^0 \rightarrow D_1 (\rightarrow D^0 \pi\pi) \tau \overline{\nu}_\tau$   & D1ststTauD0PiPi          & dD1tau\_pipi      &            3 &      23 \\
 $\overline{B}^0 \rightarrow D^*_2 \tau \overline{\nu}_\tau$                          & D2ststTau                & dD2tau            &            3 &      24 \\
 $\overline{B}^0 \rightarrow D'_1 \tau \overline{\nu}_\tau$                           & D1pststTau               & dD1ptau           &            2 &      25 \\
 $\overline{B}^0 \rightarrow D^*_0 \tau \overline{\nu}_\tau$                          & D0ststTau                & dD0tau            &            2 &      26 \\
 $B^- \rightarrow D_1^0 \tau \overline{\nu}_\tau$                                     & D1stst0Tau               & uD1tau            &            3 &      27 \\
 $B^- \rightarrow D_1^0 (\rightarrow D^0 \pi\pi) \mu \overline{\nu}_\tau$             & D1stst0TauD0PiPi         & uD1tau\_pipi      &            3 &      28 \\
 $B^- \rightarrow D_2^{*0} \tau \overline{\nu}_\tau$                                  & D2stst0Tau               & uD2tau            &            3 &      29 \\
 $B^- \rightarrow {D'_1}^0 \tau \overline{\nu}_\tau$                                  & D1pstst0Tau              & uD1ptau           &            2 &      30 \\
 $B^- \rightarrow {D^*_0}^0 \tau \overline{\nu}_\tau$                                 & D0stst0Tau               & uD0tau            &            2 &      31 \\
 $\overline{B}^0 \rightarrow D^0 D_q (\rightarrow \mu \overline{\nu}_\mu X') X$       & dDDMu                    & dDDmu             &            3 &      32 \\
 $B^- \rightarrow D^0 D_q (\rightarrow \mu \overline{\nu}_\mu X') X$                  & uDDMu                    & uDDmu             &            3 &      33 \\
 $\overline{B}^0 \rightarrow D^0 D_q (\rightarrow \tau \overline{\nu}_\tau X') X$     & dDDTau                   & dDDtau            &            0 &      34 \\
 $B^- \rightarrow D^0 D_q (\rightarrow \tau \overline{\nu}_\tau X') X$                & uDDTau                   & uDDtau            &            0 &      35 \\
 $B^-$ comb. bkg.                                                                     & BComb                    & comb              &            1 &      36 \\
 misID.                                                                               & misID                    & misID             &            1 &      37 \\
\bottomrule
\end{tabular}
\end{table}

%% Dst
\begin{table}[htb]
    \caption{Fit templates included in \Dstar channel fit.}
    \label{tab:fit-templates-dst}
    \footnotesize
    \centering
\begin{tabular}{lllrr}
\toprule
 {\bf Decay mode}  & {\bf Run 2 \texttt{process}} & {\bf Alias in fitter}  & {\bf Variations}  & {\bf Index} \\
\midrule
 $\overline{B}^0 \rightarrow D^{*+} \mu^- \overline{\nu}_\mu$                      & DstMu                    & sigmu             &           10 &       1 \\
 $\overline{B}^0 \rightarrow D^{*+} \tau^- \overline{\nu}_\tau$                    & DstTau                   & sigtau            &           10 &       2 \\
 $\overline{B}^0 \rightarrow D_1 \mu \overline{\nu}_\mu$                           & D1ststMu                 & D1                &            3 &       3 \\
 $\overline{B}^0 \rightarrow D^*_2 \mu \overline{\nu}_\mu$                         & D2ststMu                 & D2                &            3 &       4 \\
 $\overline{B}^0 \rightarrow D'_1 \mu \overline{\nu}_\mu$                          & D1pststMu                & D1p               &            2 &       5 \\
 $\overline{B} \rightarrow D^{**} (\rightarrow D^* \pi\pi) \mu \overline{\nu}_\mu$ & DststHMuDst              & D2Smu             &            1 &       6 \\
 $B^- \rightarrow D_1^0 \mu \overline{\nu}_\mu$                                    & D1stst0Mu                & uD1               &            3 &       7 \\
 $B^- \rightarrow D_2^{*0} \mu \overline{\nu}_\mu$                                 & D2stst0Mu                & uD2               &            3 &       8 \\
 $B^- \rightarrow {D'_1}^0 \mu \overline{\nu}_\mu$                                 & D1pstst0Mu               & uD1p              &            2 &       9 \\
 $\overline{B}_s \rightarrow D_{s2}^* \mu \overline{\nu}_\mu$                      & Ds2Mu                    & Ds2               &            3 &      10 \\
 $\overline{B}_s \rightarrow D'_{s1} \mu \overline{\nu}_\mu$                       & Ds1pMu                   & Ds1p              &            3 &      11 \\
 $\overline{B}^0 \rightarrow D_1 \tau \overline{\nu}_\tau$                         & D1ststTau                & D1tau             &            3 &      12 \\
 $\overline{B}^0 \rightarrow D^*_2 \tau \overline{\nu}_\tau$                       & D2ststTau                & D2tau             &            3 &      13 \\
 $\overline{B}^0 \rightarrow D'_1 \tau \overline{\nu}_\tau$                        & D1pststTau               & D1ptau            &            2 &      14 \\
 $B^- \rightarrow D_1^0 \tau \overline{\nu}_\tau$                                  & D1stst0Tau               & uD1tau            &            3 &      15 \\
 $B^- \rightarrow D_2^{*0} \tau \overline{\nu}_\tau$                               & D2stst0Tau               & uD2tau            &            3 &      16 \\
 $B^- \rightarrow {D'_1}^0 \tau \overline{\nu}_\tau$                               & D1pstst0Tau              & uD1ptau           &            2 &      17 \\
 $\overline{B}^0 \rightarrow D^* D_q (\rightarrow \mu \overline{\nu}_\mu X') X$    & dDDMu                    & dDDmu             &            3 &      18 \\
 $B^- \rightarrow D^* D_q (\rightarrow \mu \overline{\nu}_\mu X') X$               & uDDMu                    & uDDmu             &            3 &      19 \\
 $\overline{B}^0 \rightarrow D^* D_q (\rightarrow \tau \overline{\nu}_\tau X') X$  & dDDTau                   & dDDtau            &            0 &      20 \\
 $B^- \rightarrow D^* D_q (\rightarrow \tau \overline{\nu}_\tau X') X$             & uDDTau                   & uDDtau            &            0 &      21 \\
 $\overline{B}^0$ comb. bkg.                                                       & BComb                    & comb              &            1 &      22 \\
 misID.                                                                            & misID                    & misID             &            1 &      23 \\
 $D^*$ comb. bkg.                                                                  & DstComb                  & doug              &            0 &      24 \\
\bottomrule
\end{tabular}
\end{table}


\subsection{Combinatorial backgrounds}
\label{ref:fit:tmpl:comb}

\begin{figure}[htb]
    %\begin{subfigure}{\textwidth}
    %    \caption{
    %        $\Bm \rightarrow \Dz\taum\neutb$ vs.
    %        $\Bzb \rightarrow \Dstarp\taum\neutb$.
    %    }
    %    \label{fig:d0-sig-vs-dst-sig}
    %\end{subfigure}

    \caption{
        Comparison between \DstComb and \Dstarp\taum signal template.
    }
    \label{fig:dst-comb-vs-dst-sig}
\end{figure}

\begin{figure}[htb]
    %\begin{subfigure}{\textwidth}
    %    \caption{
    %        $\Bm \rightarrow \Dz\taum\neutb$ vs.
    %        $\Bzb \rightarrow \Dstarp\taum\neutb$.
    %    }
    %    \label{fig:d0-sig-vs-dst-sig}
    %\end{subfigure}

    \caption{
        Comparison between \BComb and the signal templates in their
        respective channel.
    }
    \label{fig:b-comb-vs-sig}
\end{figure}

\subsection{Muon misID backgrounds}
\label{ref:fit:tmpl:misid}

\begin{figure}[htb]
    %\begin{subfigure}{\textwidth}
    %    \caption{
    %        $\Bm \rightarrow \Dz\taum\neutb$ vs.
    %        $\Bzb \rightarrow \Dstarp\taum\neutb$.
    %    }
    %    \label{fig:d0-sig-vs-dst-sig}
    %\end{subfigure}

    \caption{
        Comparison between misID and the signal templates in their
        respective channel.
    }
    \label{fig:misid-vs-sig}
\end{figure}


\subsubsection{Modelling of $K, \pi$ decay-in-flight}
\label{ref:fit:tmpl:misid:dif}


\subsection{Signal and normalization}
\label{tmpl:sig-norm}

\paragraph{$B \rightarrow D^{0,*+}\mun\neumb$}
These are normalization templates.
The yields of these two modes, denoted as $N_{D\mu}$ and $N_{\Dstar\mu}$,
are set to be the free-floating overall normalization of the respective
fit channels.
As a side, the \Dz\muon template has notably softer \qSq spectrum compared to
that of \Dstar\muon, due to reduced helicity states available in the
$B \rightarrow D \ell \neulb$ transitions
(discussed in \cref{sec:theory:form-factors}),
as shown in \cref{fig:d0-norm-vs-dst-norm}.
The \Dstarp templates (both muonic and tauonic) contribute to both fit channels
due to feed down.

\begin{figure}[htb]
    \caption{
        Comparison between $\Bm \rightarrow \Dz\mun\neumb$ and
        $\Bzb \rightarrow \Dstarp\mun\neumb$ templates.
    }
    \label{fig:d0-norm-vs-dst-norm}
\end{figure}

\paragraph{$B \rightarrow D^{0,*+}\taum\neutb$}
The signal templates, with \RD and \RDstp controlling the fraction
of yields compared to respective normalization yields.
The signal distributions are characterized by high \mmSq and softer
\el (due to \muon coming from a \tauon decay),
as shown in \cref{fig:d0-sig-vs-d0-norm,fig:dst-sig-vs-dst-norm}.
Also, the \Dz\tauon has a softer \qSq distribution, due to the same reason
mentioned in the normalization comparison, which is displayed in
\cref{fig:d0-sig-vs-dst-sig,fig:d0-sig-vs-dst0-sig}.

\begin{figure}[htb]
    \begin{subfigure}{\textwidth}
        \caption{
            $\Bm \rightarrow \Dz\mun\neumb$ vs.
            $\Bm \rightarrow \Dz\taum\neutb$.
        }
        \label{fig:d0-sig-vs-d0-norm}
    \end{subfigure}

    \begin{subfigure}{\textwidth}
        \caption{
            $\Bzb \rightarrow \Dstarp\mun\neumb$ vs.
            $\Bzb \rightarrow \Dstarp\taum\neutb$.
        }
        \label{fig:dst-sig-vs-dst-norm}
    \end{subfigure}

    \begin{subfigure}{\textwidth}
        \caption{
            $\Bm \rightarrow \Dstarz\mun\neumb$ vs.
            $\Bm \rightarrow \Dstarz\taum\neutb$.
        }
        \label{fig:dst0-sig-vs-dst0-norm}
    \end{subfigure}
    \caption{Comparison between signal and normalization templates.}
\end{figure}

\begin{figure}[htb]
    \begin{subfigure}{\textwidth}
        \caption{
            $\Bm \rightarrow \Dz\taum\neutb$ vs.
            $\Bzb \rightarrow \Dstarp\taum\neutb$.
        }
        \label{fig:d0-sig-vs-dst-sig}
    \end{subfigure}

    \begin{subfigure}{\textwidth}
        \caption{
            $\Bm \rightarrow \Dz\taum\neutb$ vs.
            $\Bm \rightarrow \Dstarz\taum\neutb$.
        }
        \label{fig:d0-sig-vs-dst0-sig}
    \end{subfigure}

    \caption{Comparison between signal templates.}
\end{figure}

\paragraph{$\Bm \rightarrow D^{*0}\ellm\neulb$}
The \Dstarz ($\rightarrow \Dz \pi$ or $\rightarrow \Dz \gamma$)
in these modes are not reconstructed,
so these templates contribute to the \Dz channel only.
The muonic mode contains an additional branching fraction ratio on top of
$N_{D\mu}$ in its normalization, and is the largest class of contribution
in \Dz data.
For tauonic mode, the \RDstz is set to be equal to \RDstp in the nominal fit.


\subsection{Feed down through $B \rightarrow D^{**}\ellm\neulb$ modes}
\label{tmpl:dstst}

The feed downs from 4 1P $D^{**}$ states
($D_1(2420), D_2^*(2460), D_0^*(2430), D'_1(2430)$, including both isospin
pairs)
are included in the fit, with $10+10$ ($\mu+\tau$) templates in \Dz channel
and $6+6$ in \Dstar.
These $D^{**}$ decay into a \Dz either directly or in a
cascade via a \Dstar or a lighter $D^{**}$,
possibly producing two \emph{charged} \pion
(hence the \Dz\pion\pion templates).

\paragraph{Muonic}
The $D^{**}$ decays often contain unreconstructed \piz or $\gamma$ which leads
to a broad \mmSq distribution peaking at smaller values $\approx m^2_\pion$,
as shown in \cref{fig:dstst-mu-vs-d0-sig}.
The \el spectrum is comparable, but typically softer, than that in normalization,
due to $D^{**}$ states having higher masses compared to \Dz and \Dstar.
These decays are suppressed at zero recoil, as discussed in
\cref{theory:form-factors:dstst},
but translates weakly in the reconstructed \qSq spectrum, due to rest frame
approximation (\cref{theory:rfa}) and definition of \qSq being
$(p_B - p_D)^2$, instead of $(p_B - p_{D^{**}})^2$.

Currently, the constraints for these modes are
taken from the run 1 analysis (\cite{LHCb-ANA-2020-056}), which uses PDG data.
For each sample (i.e. ISO, 1OS, 2OS, DD),
%It is worth noting that a factor
%$n^{0,*}_{D^{**}} \equiv 10^{-3} / \mathcal{B}(B \rightarrow D^{0,*}\mun\neumb)$
%is include such that floating parameters may be read as a branching fractions
%times an efficiency directly.

\begin{figure}[htb]
    %\begin{subfigure}{\textwidth}
    %    \caption{
    %        $\Bm \rightarrow \Dz\taum\neutb$ vs.
    %        $\Bzb \rightarrow \Dstarp\taum\neutb$.
    %    }
    %    \label{fig:d0-sig-vs-dst-sig}
    %\end{subfigure}

    \caption{Comparison between $D^{**}\mu$ and \Dz\taum signal templates.}
    \label{fig:dstst-mu-vs-d0-sig}
\end{figure}

\paragraph{Tauonic}
As in run 1, the tauonic $D^{**}$ decays are treated as fixed fractions
to their muonic counterparts.
The tauonic modes currently do not have enough MC statistics, and it is planned
to request more for these modes in the near future.


\subsection{Feed down through $B \rightarrow D_H^{**}(\rightarrow D^{(*)}\pi\pi)\mun\neumb$ modes}

The heavy $D_H^{**}$ decay into either a \Dz or a \Dstar, with two additional
\pion and one of them possibly uncharged.
Their yields are floated without constraint.
The fit variables are distributed similar to that in the lighter
$D^{**}$ states.
A comparison between the $D_H^{**}$ templates and \Dz\taum signal can be
seen at \ref{fig:dstst-heavy-vs-d0-sig}

\begin{figure}[htb]
    %\begin{subfigure}{\textwidth}
    %    \caption{
    %        $\Bm \rightarrow \Dz\taum\neutb$ vs.
    %        $\Bzb \rightarrow \Dstarp\taum\neutb$.
    %    }
    %    \label{fig:d0-sig-vs-dst-sig}
    %\end{subfigure}

    \caption{Comparison between $D_H^{**}\mu$ and \Dz\taum signal templates.}
    \label{fig:dstst-heavy-vs-d0-sig}
\end{figure}


\subsection{Feed down through $\Bsb \rightarrow D_s^{**}\mun\neumb$ modes}

The $D_s^{**}$\footnote{
    More specifically, the $D_{s1}^{'+}$ and $D_{s2}^{*+}$ are included in
    this analysis.
}, daughter of \Bsb, decays into either a $\Kp\Dz$ or $\Kz\Dstarp$.
The former feeds into \Dz channel, whereas the latter feeds into both.
These modes are constrained based on available measurements.
A comparison between the $D_s^{**}$ templates and \Dz\taum signal can be
seen at \cref{fig:d_s-vs-d0-sig}.

\begin{figure}[htb]
    %\begin{subfigure}{\textwidth}
    %    \caption{
    %        $\Bm \rightarrow \Dz\taum\neutb$ vs.
    %        $\Bzb \rightarrow \Dstarp\taum\neutb$.
    %    }
    %    \label{fig:d0-sig-vs-dst-sig}
    %\end{subfigure}

    \caption{Comparison between $D_s^{**}\mu$ and \Dz\taum signal templates.}
    \label{fig:d_s-vs-d0-sig}
\end{figure}


\subsection{Double-charm ($DDX$) backgrounds}

The double-charm backgrounds include contributions from
$B \rightarrow D^{(*)}(\rightarrow \mun\neumb)\bar{D}^{(*)} X$ decays, where $X$
is either a \kaon, a \Kstar, or a higher \Kstar resonance.
These modes are not understood well, and the MC samples are generated with
phase space models with a three-body decay, so additional deformations
to the phase space are provided in the fitter to allow for a more
data-driven approach in the determination of their yields.
More details will be discussed in
\cref{sec:fit-to-data:fit-tmpl-vars:ddx}.
The shape of the $DDX$ templates are somewhat similar to the signal templates,
as can be seen in \cref{fig:ddx-vs-d0-sig}.

\begin{figure}[htb]
    %\begin{subfigure}{\textwidth}
    %    \caption{
    %        $\Bm \rightarrow \Dz\taum\neutb$ vs.
    %        $\Bzb \rightarrow \Dstarp\taum\neutb$.
    %    }
    %    \label{fig:d0-sig-vs-dst-sig}
    %\end{subfigure}

    \caption{Comparison between $DDX$ and \Dz\taum signal templates.}
    \label{fig:ddx-vs-d0-sig}
\end{figure}
