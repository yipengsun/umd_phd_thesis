\section{Additional algorithms used in selection}
\label{ref:sel:algo}


\subsection{Veto of overlapping candidates}
\label{ref:sel:algo:veto}

% See this note: https://github.com/umd-lhcb/rdx-run2-analysis/blob/master/docs/cuts/Dst_veto_in_D0.md
Since the selection of a \Dstar\muon pair always requires the existence
of a \Dz\muon pair,
about 21\% of events in the signal (ISO) skim of the \Dstar channel
would contribute to the \Dz channel as well.
% 21 computed as: 630178 (overlap) / 3011082 (num of evt for D0)
%%%%
% This is likely due to the fact that slow \pion typically has poor tracking
% resolution (small \ipChiSq), making the slow \pion to be more likely
% \emph{unassociated} with the \Dz\mun vertex thus the reconstruction of \Dstar
% would fail.
%
To veto the overlapping candidates between the \Dz and \Dstar channels,
a tool, named \smalltt{TupleToolApplyIsolationVetoDst},
is added to the selection procedure, which loops over each track $t$
in the event whose $p > 2$~GeV in the following manner:
% track acceptance from:
%   https://indico.cern.ch/event/280053/contributions/1626872/attachments/513093/708085/Tracking_Seminar_131029.pdf
% p. 28

\begin{enumerate}
    \item Refit a vertex from \Dz and $t$.
    \item Compute $\Delta m_\text{veto} \equiv m_{\Dz t} - m_\Dz$.
    \item Test if $\Delta m_\text{veto} \in [140~\text{GeV}, 160~\text{GeV}]$.
        If it is, the track $t$ is able to form a \Dstar-like particle with \Dz,
        and the $\Delta m_\text{veto}$ and isolation BDT score
        of the track are recorded.
\end{enumerate}

From all recorded BDT scores and $\Delta m_\text{veto}$, the
$\Delta m_\text{veto}$ of the tracks with top two BDT scores are saved
for future use.
It is then required offline (listed in \cref{tab:offline-cut-d0})
that these two tracks,
the best slow \pion candidates,
are \emph{incompatible} with
forming a \Dz\pion vertex with a mass close to \Dstar reference mass.

Note that this tool does not rank tracks based \emph{solely} on the isolation:
The poor tracking resolution of the slow \pion makes the association of
the track more ambiguous; sometime the track may be associated with the PV,
instead of the \B vertex, in which case the track would receive a small BDT
score.
By ranking based on the BDT scores of the tracks that fall within the
$[140~\text{GeV}, 160~\text{GeV}]$
$\Delta m_\text{veto}$ window only,
it is more robust to retain the best slow \pion candidates,
as reported in \cite{LHCb-ANA-2020-056}.

In addition, an alternative mass hypothesis is tested
(offline, also listed in \cref{tab:offline-cut-d0}), where the reconstructed
\muon is treated as a \pion,
and $\Delta m_\text{alt hypo} = m_{\Dz\muon_\text{as \pion}} - m_\Dz$
is computed\footnote{
    By using the 3-momentum of the \muon, while using $m_\pion$ as the invariant
    mass of the track.
} and required to be inconsistent with \Dstar PDG mass.

The veto procedures are applied on all samples of the \Dz channel only.


\subsection{Muon PID}
\label{ref:sel:algo:ubdt}

To maintain minimum bias on \muon \pt, a multivariate BDT,
referred as \UBDT, using \texttt{uBDT} method in TMVA class,
is trained to reject misidentified \muon while keeping rejection
efficiency flat on \pt,
as described in \cite{LHCb-ANA-2020-056}.

The \UBDT has been re-trained on LHCb run 2 MC samples.
There is ongoing effort to study its efficiency against standard run 2 PID
variables.
Early report suggests the flatness on \pt is maintained while rejection
efficiencies relative to existing PID cut are high for all but proton ($p$),
as shown in \cref{fig:ubdt-eff}.

% Generated with:
%   https://github.com/umd-lhcb/pidcalib2, branch efficiency_study
% first, run efficiency_gen/rdx-run2-ubdt-misidgen.sh
% then, go inside scripts folder, run ./plots.py
\begin{figure}[!htb]
    \centering
    \includegraphics[width=0.45\textwidth]{./figs-selection-tech/eff_Brunel_PT_up_pidcalib_ubdt_eff.pdf}
    \hspace{1em}
    \includegraphics[width=0.45\textwidth]{./figs-selection-tech/rej_v_eff_unbiased_Brunel_PT.pdf}

    \caption{
        Preliminary \UBDT study.
        Left: \UBDT \muon selection efficiency is flat in \pt, with
        global cut \isMuon \& $\text{\PID{\muon}}\! > 2$,
        and $\UBDT > 0.25$.
        Right: with the same global cut, \UBDT is more effective in rejecting
        \pion than LHCb official \ProbNN{\muon}.
        The \kaon rejection efficiencies are similar between the two;
        the $p$ rejection efficiency is lower for \UBDT, but the absolute
        rejection efficiency is high enough.
    }
    \label{fig:ubdt-eff}
\end{figure}
