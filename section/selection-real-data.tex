\section{Real data}
\label{ref:sel:data}

This analysis currently uses 1665~pb$^{-1}$ of proton-proton collision data
recorded by LHCb in 2016, at $\sqrt{s} = 13$~TeV.
It is planned to extend the use of data to 2016--2018 LHCb data.
The rest of the section describes selection procedures for various data samples.


\subsection{Global online selection}
\label{ref:sel:data:global-online}

The \Dz\mun and \Dstarp\mun final states,
henceforth referred as \Dz channel and \Dstar channel,
share the sample online selection which involves
an online trigger path,
that is,
a composition of hardware (L0) and fast software triggers (HLT1, HLT2),
and additional pre-defined online cuts.
Overall, the online selection requires a \muon candidate to form
a well-defined vertex,
displaced from the $pp$ vertex (PV),
with a high $p_T$ $\Dz (\rightarrow \Km \pip)$ candidate, without any cut on the
invariant mass on the \Dz\mupm pair, nor on the \muon \pt.

The trigger path is listed in \cref{tab:triggers}.
%%%%
The L0 triggers are chosen to make sure events are not trigged by \muon
\emph{alone} which will introduce a \pt bias\footnote{
    According to table 1 in \cite{LHCb-DP-2019-001},
    the minimum \pt threshold across all L0 triggers is 1.35 GeV for 2018
    L0Muon trigger.
} on \muon,
which is very problematic for this analysis as \muon from signal
$\tauon \rightarrow \muon \neumb \neut$ decays are typically softer and
have smaller \pt,
making signal events less likely to be selected.
%%%%
The inclusive HLT1 triggers select events containing a particle whose decay
vertex is displaced from the PV,
in this case a $\Dz \rightarrow \Km \pip$,
as described in \cite{LHCb-INT-2019-025}, Section 6.7.1.
%%%%
The \smalltt{Hlt2XcMuXForTauB2XcMu} is specifically designed for this analysis
to select \Dz\mupm candidates\footnote{
    Yes, both signs of \muon are selected.
    The reason will become obvious in later sections.
} with a high \pt requirement on \Dz,
while maintain minimal \pt bias on the \mupm.
The cuts imposed by the HLT2 line are listed in \cref{tab:cut-hlt2}.
%%%%
Additional online cuts,
listed in \cref{tab:cut-stripping},
further tighten the kinematic, vertex quality, and particle identification (PID)
requirements.

\begin{table}[htb]
    \caption{
        Trigger path for this analysis.
        For terminologies such as TIS and TOS, refer to
        \cref{appx:trigger-cat}.
    }
    \label{tab:triggers}
    \centering
    \parnotereset
    \begin{tabular}{c|c}
        \toprule
        {\bf Trigger level} & {\bf Requirements} \\
        \midrule
        L0 & \Dz L0Hadron TOS || \B L0Global TIS \\
        HLT1 & \makecell{
            (\kaon Hlt1TrackMVA TOS || \pion Hlt1TrackMVA TOS)\parnote{
                This is almost equivalent to \Dz Hlt1TrackMVA TOS, with a
                $\sim\!0.0027\%$ difference in selected events in
                reconstructed data sample in \Dz channel.
                Henceforth these two trigger paths are considered equivalent.
            } || \\ \Dz Hlt1TwoTrackMVA TOS
        } \\
        HLT2 & \B Hlt2XcMuXForTauB2XcMu TOS \\
        \bottomrule
    \end{tabular}
    \begin{flushleft}
        \parnotes
    \end{flushleft}
\end{table}

\begin{table}[htb]
    \caption{Cuts defined in \smalltt{Hlt2XcMuXForTauB2XcMu}.}
    \label{tab:cut-hlt2}
    \centering
    \begin{tabular}{ c | rll}
        \toprule
        {\bf Particle} & {\bf Variable}               & {\bf Cuts}               \\
        \midrule
        \kaon, \pion   & \pt                          & $> 200$ MeV              \\
                       & Max \pt                      & $> 800$ MeV              \\
                       & \ptot                        & $> 5$ GeV                \\
                       & \ipChiSq                     & $> 9$                    \\
                       & \PID{$K$}                    & $> 2\;(K)$, $< 4\;(\pi)$ \\
        \midrule
        \Dz            & \pt                          & $> 2000$ MeV             \\
                       & $|\pi\;\pt|+|K\;\pt|$        & $> 2.5$ GeV              \\
                       & $m$                          & 1830--1910 MeV           \\
                       & \anyChiSq{vertex}/ndf        & $< 10$                   \\
                       & \anyChiSq{FD}                & $> 25$                   \\
                       & \DIRA                        & $> 0.999$                \\
                       & Child pair \DOCA             & $< 0.1$ mm               \\

        \midrule
        \muon          & \ipChiSq                     & $> 16$                   \\

        \midrule
        \Dz\muon       & \anyChiSq{vertex}/ndf        & $< 15$                   \\
                       & \anyChiSq{FD}                & $> 50$                   \\
                       & \DIRA                        & $> 0.999$                \\
                       & \DOCA                        & $< 0.5$ mm               \\
        \bottomrule
    \end{tabular}
\end{table}

\begin{table}[htb]
    \caption{Additional online cuts.}
    \label{tab:cut-stripping}
    \centering
    \begin{tabular}{c|rll}
        \toprule
        {\bf Event-Level }      & {\bf Variable}               & {\bf Cuts}               \\
        \midrule
        GEC                     & nSPDhits                     & $< 600$                  \\
        PV cut                  & nPV                          & $\geq1$                  \\
        \toprule
        {\bf Particle }         & {\bf Variable}               & {\bf Cuts}               \\
        \midrule
        $K, \pi$                & \pt                          & $> 300$ MeV              \\
                                & \ptot                        & $> 2$ GeV                \\
                                & \anyChiSq{IP}                & $> 9$                    \\
                                & \isMuon                      & \texttt{False}           \\
                                & \PID{$K$}                    & $> 4\;(K)$, $< 2\;(\pi)$ \\
                                & GhostProb                    & $< 0.5$                  \\
        \midrule
        \Dz                     & $|\pi\;\pt|+|K\;\pt|$        & $> 2.5$ GeV              \\
                                & $m - m_\text{PDG}$           & $< 80$ MeV               \\
                                & \anyChiSq{vertex}/nof        & $< 4$                    \\
                                & \anyChiSq{FD}                & $> 25$                   \\
                                & \DIRA                        & $> 0.999$                \\
        \midrule
        \muon                   & \ptot                        & $> 3$ GeV               \\
                                & \ipChiSq                     & $> 16$                  \\
                                & isMuon                       & \texttt{True}           \\
                                & \PID{$\mu$}                  & $> -200$                \\
                                & GhostProb                    & $< 0.5$                 \\
        \midrule
        \Dz\muon                & $m$                          & 0--10 GeV               \\
                                & \anyChiSq{vertex}/ndf        & $< 6$                   \\
                                & \DIRA                        & $>0.999$                \\
        \bottomrule
    \end{tabular}
\end{table}


\subsection{Global offline selection}

Additional offline selection with stricter vertex quality and PID requirements,
mostly aligned with the \emph{online} selection in LHCb run 1,
are applied to both \Dz and \Dstar channels.
%%%%
Furthermore, only the $D^{0,*+}\mun$ candidates,
referred as ``right-sign'' combinations,
are kept.
%%%%
The initial motivation was to bring 2016 data on par with run 1 data so
distributions of the same variables in 2016 and run 1 can be compared in a more
meaningful way,
providing check points for the development of this analysis.
%%%%
Later, it is checked with a survey of selection efficiencies of different decay
modes that the run 1 selection is still optimal.

The offline cuts that are shared among \Dz and \Dstar channels are listed
in \cref{tab:offline-cut-common}.
The additional \Dz only cuts are listed in \cref{tab:offline-cut-d0},
The \Dstar in \cref{tab:offline-cut-dst}.
Some of the cutting variables are not standard LHCb variables;
these are marked in the tables and will be discussed in \cref{ref:sel:tech}.

Due to the fact that both \Dz and \Dstar channel have the same online selection,
some candidates will appear in both,
making the two channels correlated.
%%%%
This correlation is removed by a \Dstar veto procedure in the \Dz channel;
such procedure makes the two channels mostly orthogonal, and is discussed in
\cref{ref:sel:tech:veto}.

\begin{table}[htb]
    \caption{Offline cuts shared among \Dz and \Dstar channels.}
    \label{tab:offline-cut-common}
    \centering
    \begin{tabular}{c|rll}
        \toprule
        {\bf Event-Level }  & {\bf Variable}               & {\bf Cuts}               \\
        \midrule
        GEC                 & nSPDhits                     & $< 450$\parnote{
            This is to reduce correlation for \emph{TIS and TOS} candidates to make
            trigger emulation more robust.
            See \cref{ref:emulation-for-to-mc:correlation-tos-tis} for more details.
        }                                                                             \\
        \toprule
        {\bf Particle}      & {\bf Variable}               & {\bf Cuts}               \\
        \midrule
        $K, \pi$            & \pt                          & $> 500$ MeV              \\
                            & HLT1 TOS track \pt           & $> 1.7$ GeV              \\
                            & \ptot                        & $< 200$ GeV\parnote{
                                This is to make \ptot consistent with extended
                                \pidcalib binning.
                            }                                                         \\
                            & \anyChiSq{IP}                & $> 45$                   \\
                            & \isMuon                      & \texttt{False}           \\
                            & \PID{$K$}                    & $> 4\;(K)$, $< 2\;(\pi)$ \\
                            & GhostProb                    & $< 0.5$                  \\
        \midrule
        \Dz                 & \pt                          & $> 2$ GeV                \\
                            & $|\pi\;\pt|+|K\;\pt|$        & $> 1.4$ GeV              \\
                            & $\log(\IP)$\parnote{
                                \IP in terms of the \PrmVtx.
                                That is, the reconstructed \Dz is inconsistent
                                of coming from \PrmVtx directly.
                            }                              & $> -3.5$                 \\
                            & \anyChiSq{IP}                & $> 9$                    \\
                            & \anyChiSq{vertex}/ndf        & $< 4$                    \\
                            & \anyChiSq{FD}                & $> 250$                  \\
                            & \DIRA                        & $> 0.9998$               \\
        \midrule
        \muon               & \ptot                        & 3-100 GeV                \\
                            & $\log_{10}(1 - \vec{p}_\mu \cdot \vec{p}_i / (|p_\mu||p_i|))$,
                              $i \in \{K, \pi\}$
                                                           & $> -6.5$                 \\
                            & $\eta$                       & 1.7--5                   \\
                            & \ipChiSq                     & $> 45$                   \\
                            & GhostProb                    & $< 0.5$                  \\
                            & \PID{$\mu$}                  & $> 2$                    \\
                            & \PID{$e$}                    & $< 1$                    \\
                            & \UBDT\parnote{
                                This is a BDT-based \muon PID algorithm,
                                which is discussed in \cref{ref:selection:mu-pid}.
                            }
                                                           & $> 0.25$                 \\
        \bottomrule
    \end{tabular}
    \begin{flushleft}
        \parnotes
    \end{flushleft}
\end{table}

\begin{table}[htb]
    \caption{Offline cuts exclusive to \Dz channel.}
    \label{tab:offline-cut-d0}
    \centering
    \begin{tabular}{c|rll}
        \toprule
        {\bf Particle}  & {\bf Variable}               & {\bf Cuts}      \\
        \midrule
        \Dz\muon        & $m$                          & $< 5200$ MeV    \\
                        & {\footnotesize$\begin{aligned}
                                \text{Max}(
                                    &|\Delta m_\text{veto,1} - 145.454|, \\
                                    &|\Delta m_\text{veto,2} - 145.454|
                                )
                           \end{aligned}$}\parnote{
                            \label{parnote:dst-veto}
                            To reduce overlapping events between
                            \Dz and \Dstar channels.
                            Discussed in \cref{ref:selection:veto}.
                        }                              & $> 4$ MeV       \\
                        & $\Delta m_\text{alt hypo}$\parnoteref{parnote:dst-veto}
                                                       & $> 165$ MeV     \\
                        & transverse flight distance   & $< 7$ mm        \\
                        & \anyChiSq{vertex}/ndf        & $< 6$           \\
                        & \DIRA                        & $> 0.9995$      \\
        \bottomrule
    \end{tabular}
    \begin{flushleft}
        \parnotes
    \end{flushleft}
\end{table}

\begin{table}[htb]
    \caption{Offline cuts exclusive to \Dstar channel.}
    \label{tab:offline-cut-dst}
    \centering
    \begin{tabular}{c|rll}
        \toprule
        {\bf Particle}  & {\bf Variable}                 & {\bf Cuts}    \\
        \midrule
        slow \pion     & GhostProb                       & $< 0.25$      \\
        \midrule
        \Dstarp        & $|m - m_\Dz - 145.454|$\parnote{
                           This is to require $m_\Dstarp$ to close to
                           its PDG mass,
                           up to a reconstruction effect on
                           $m_\Dz$.
                       }                                 & $< 2$ MeV     \\
                       & \anyChiSq{vertex}/ndf           & $< 10$        \\
        \midrule
        \Dz\muon\parnote{
            This pair is not available at \davinci output.
        }              & \anyChiSq{vertex}/ndf           & $< 6$         \\
                       & \DIRA\parnote{
                           This cut is applied at \davinci level,
                           \emph{not} offline.
                       }                                 & $> 0.999$     \\
        \midrule
        \Dstarp\muon   & $m$                             & $< 5280$ MeV  \\
                       & transverse flight distance      & $< 7$ mm      \\
                       & \anyChiSq{vertex}/ndf           & $< 6$         \\
                       & \anyChiSq{vertex}               & $< 24$        \\
                       & \DIRA                           & $> 0.9995$    \\
        \bottomrule
    \end{tabular}
    \begin{flushleft}
        \parnotes
    \end{flushleft}
\end{table}


\subsection{Signal (ISO) and control (1OS, 2OS, DD) samples}
\label{ref:sel:data:skims}

The signal and control samples are selected by inspecting if there exists
additional charged tracks that are compatible with coming from the
reconstructed \B vertex.
To predict the possibility of a charged track originating from the \B vertex,
a multivariate BDT, referred as ``isolation BDT'', is used.
The BDT is conceptually identical to the one used in
\cite{LHCb-ANA-2020-056} but with the BDT re-trained based on LHCb run 2
simulation.

The isolation BDT\footnote{
    The isolation BDT can be found at
    \techurllink{https://github.com/umd-lhcb/TupleToolSemiLeptonic/blob/master/Phys/TupleToolSemiLeptonic/src/TupleToolApplyIsolation.cpp}{github/umd-lhcb/TupleToolSemiLeptonic}.
} is added to the \davinci reconstruction sequence.
It loops over all charged tracks in the event\footnote{
    More specifically, these tracks are coming from
    \texttt{StdAllNoPIDsPions}, \texttt{StdNoPIDsUpPions},
    and \texttt{StdNoPIDsVeloPions}.
},
providing isolation scores for each of them where
higher score implies higher probability of coming from the vertex, named as
``anti-isolated''.
Three tracks with highest isolation scores\footnote{
    The ones that are most likely to be anti-isolated.
} are saved in the output file, in descending order\footnote{
    So the first one is the most anti-isolated one among all charged tracks
    in the event.
}.

Below a brief description for the signal sample and the main physics background
control samples is provided.
The actual isolation selections are listed in \cref{tab:skim-cut}.

\begin{itemize}
    \item ISO: $B \rightarrow D^{(*)} \lepton \neulb$, with $\lepton \in \{\muon,
        \tauon\}$.
        This is referred as signal, or ``isolated'', sample.
        It requires that no additional charged track is from the \B vertex
        (in a probabilistic sense, with probability related to the isolation
        score)
        and is compatible with a fully reconstructed \B decay
        (ignoring missing neutrino(s)).

    \item DD: $B \rightarrow D^{(*)}D X$,
        with dominate $D \rightarrow \Kp \mun \neumb X$ sub-decays.
        This is a double-charm ($DD$) control sample,
        which is selected by requiring at least one anti-isolated track,
        a \kaon-like long track\footnote{
            This means the track went through both upstream and downstream
            trackers, and is generally of good tracking quality.
        } and a hard track in the three most anti-isolated tracks\footnote{
            Note that these three requirements can be satisfied by a single
            track.
        }.

    \item 1OS: $B \rightarrow D^{**} \lepton \neulb$.
        This control sample, enriched in excited charm states,
        requires one and only one additional anti-isolated charged long track
        that is compatible with a \pion PID hypothesis and has correct charge
        for a $D^{**} \rightarrow D^{(*+)}\pim$ decay.

    \item 2OS: $B \rightarrow D^{**}_H \lepton \neulb$,
        where $H$ stands for ``heavy''.
        This control sample is enriched in highly excited (heavy) charm states,
        which is selected by requiring two and only two anti-isolated \pion-like
        long tracks of opposite charge,
        capable of $D^{**}_H \rightarrow D^{(*)} \pip\pim$ decay.

        % Why?
        This sample also provides an independent selection of
        $B \rightarrow D^{(*)}D X$ backgrounds, where the \pip\pim fit into the
        $X$ category and \kaon escapes isolation detection.
\end{itemize}

\begin{table}[htb]
    \caption{Signal and control sample isolation requirements.}
    \label{tab:skim-cut}
    \centering
    \begin{tabular}{c|rll}
        \toprule
        {\bf Sample}  & {\bf Variable}              & {\bf Cuts}     \\
        \midrule
        ISO           & \isoBDT{1}                  & $< 0.15$       \\
        \midrule
        1OS           & \isoBDT{1}                  & $> 0.15$       \\
                      & \isoBDT{2}                  & $< 0.15$       \\
                      & \isoTrack{1}                & $= 3$\parnote{
                          This means a long track.
                      }                                              \\
                      & $p_1$                       & $> 5$ GeV      \\
                      & $p_{T,1}$                   & $> 0.15$ GeV   \\
                      & \ProbNN{$K_1$}              & $< 0.2$        \\
                      & $Q_1 \cdot \text{PID}_\Dz$\parnote{
                          Apply to \Dz channel,
                          which implies that the anti-isolated \pip can
                          form a \Dstarp with the \Dz.
                      }                             & $> 0$          \\
                      & $Q_1 \cdot \text{PID}_\Dstar$\parnote{
                          Apply to \Dstar channel.
                          Here it is required that the anti-isolated \pim can
                          form a $D^{**0}$ with the \Dstarp.
                      }                             & $< 0$          \\
        \midrule
        2OS           & \isoBDT{1}                  & $> 0.15$       \\
                      & \isoBDT{2}                  & $> 0.15$       \\
                      & \isoBDT{3}                  & $< 0.15$       \\
                      & \isoTrack{1}                & $= 3$          \\
                      & \isoTrack{2}                & $= 3$          \\
                      & {\footnotesize
                         Max$(p_1 \cdot (p_{T,1} > 0.15 \text{ GeV}),
                              p_2 \cdot (p_{T,2} > 0.15 \text{ GeV}))$
                        }
                                                    & $> 5$ GeV      \\
                      & \ProbNN{$K_1$}              & $< 0.2$        \\
                      & \ProbNN{$K_2$}              & $< 0.2$        \\
                      & $Q_1 \cdot Q_2$             & $< 0$          \\
        \midrule
        DD            & \isoBDT{1}                  & $> 0.15$       \\
                      & {\footnotesize$\begin{aligned}
                            \text{Max}(
                            &p_1 \cdot (p_{T,1} > 0.15\text{ GeV}),  \\
                            &p_2 \cdot (p_{T,2} > 0.15\text{ GeV})
                                 \cdot (\text{\isoBDT{2}} > -1.1),   \\
                            &p_3 \cdot (p_{T,3} > 0.15\text{ GeV})
                                 \cdot (\text{\isoBDT{3}} > -1.1)
                            )
                        \end{aligned}$}             & $> 5$ GeV      \\
                      & Max(\ProbNN{$K_{1,2,3}$})   & $> 0.2$        \\
                      & \isoTrack{\text{the one passing $K$ PID requirement}}
                                                    & $= 3$          \\
                      & \isoBDT{\text{the one passing $K$ PID requirement}}
                                                    & $> -1.1$       \\
        \bottomrule
    \end{tabular}
    \begin{flushleft}
        \parnotes
    \end{flushleft}
\end{table}
