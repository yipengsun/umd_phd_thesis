\section{Real data}
\label{ref:sel:data}

This analysis currently uses 1665~pb$^{-1}$ of proton-proton collision data
recorded by LHCb in 2016, at $\sqrt{s} = 13$~TeV.
It is planned to extend the use of data to 2016--2018 LHCb data.
The rest of the section describes selection procedures for various data samples.


\subsection{Right sign samples with a ``real'' muon}
\label{ref:sel:data:rs}

It is important to first define the terms ``right sign'' and ```real' muon''.
``Right sign'' means the $D^{(*)} \mu$ pair can form a legitimate $B$ meson.
That is, only \Dz\mun and \Dstarp\mun (and their charge conjugates) pairs are
of the ``right sign''.
As of ``real'' muon, it requires the selected \muon to pass certain particle
identification requirements.

The online selection for \Dz\mun and \Dstarp\mun final states,
henceforth referred as \Dz channel and \Dstar channel,
are identical.
The selection involves an online trigger path,
that is,
a composition of hardware (L0) and fast software triggers (HLT1, HLT2),
and additional pre-defined online cuts.
Overall, the online selection requires a \muon candidate to form
a well-defined vertex,
displaced from the $pp$ vertex (PV),
with a high $p_T$ $\Dz (\rightarrow \Km \pip)$ candidate, without any cut on the
invariant mass on the \Dz\mupm pair, nor on the \muon \pt.
More details regarding the online selection is given below.

The trigger path is listed in \cref{tab:triggers}.
%%%%
The L0 triggers are chosen to make sure events are not trigged by
the \muon candidate \emph{alone} which will introduce a \pt bias\footnote{
    According to table 1 in \cite{LHCb-DP-2019-001},
    the minimum \pt threshold across all L0 triggers is 1.35 GeV for 2018
    L0Muon trigger.
} on \muon.
Such bias is very problematic for this analysis as \muon from signal
$\tauon \rightarrow \muon \neumb \neut$ decays are typically softer and
have smaller \pt,
making signal events less likely to be selected.
%%%%
The inclusive HLT1 triggers select events containing a particle whose decay
vertex is displaced from the PV,
in this case a $\Dz \rightarrow \Km \pip$,
as described in \cite{LHCb-INT-2019-025}, Section 6.7.1.
%%%%
The \smalltt{Hlt2XcMuXForTauB2XcMu} is specifically designed for this analysis
to select \Dz\mupm candidates\footnote{
    Yes, both signs of \muon are selected.
    The reason will become obvious in later sections.
} with a high \pt requirement on \Dz,
while maintain minimal \pt bias on the \mupm.
The cuts imposed by the HLT2 line are listed in \cref{tab:cut-hlt2}.
%%%%
Additional online cuts,
listed in \cref{tab:cut-stripping},
further tighten the kinematic, vertex quality, and particle identification (PID)
requirements.

\begin{table}[htb]
    \caption{
        Trigger path for this analysis.
        For terminologies such as TIS and TOS, refer to
        \cref{appx:trigger-cat}.
    }
    \label{tab:triggers}
    \centering
    \parnotereset
    \begin{tabular}{c|c}
        \toprule
        {\bf Trigger level} & {\bf Requirements} \\
        \midrule
        L0 & \Dz L0Hadron TOS || \B L0Global TIS \\
        HLT1 & \makecell{
            (\kaon Hlt1TrackMVA TOS || \pion Hlt1TrackMVA TOS)\parnote{
                This is almost equivalent to \Dz Hlt1TrackMVA TOS, with a
                $\sim\!0.0027\%$ difference in selected events in
                reconstructed data sample in \Dz channel.
                Henceforth these two trigger paths are considered equivalent.
            } || \\ \Dz Hlt1TwoTrackMVA TOS
        } \\
        HLT2 & \B Hlt2XcMuXForTauB2XcMu TOS \\
        \bottomrule
    \end{tabular}
    \begin{flushleft}
        \parnotes
    \end{flushleft}
\end{table}

\begin{table}[htb]
    \caption{Cuts defined in \smalltt{Hlt2XcMuXForTauB2XcMu}.}
    \label{tab:cut-hlt2}
    \centering
    \begin{tabular}{ c | rll}
        \toprule
        {\bf Particle} & {\bf Variable}               & {\bf Cuts}               \\
        \midrule
        \kaon, \pion   & \pt                          & $> 200$ MeV              \\
                       & Max \pt                      & $> 800$ MeV              \\
                       & \ptot                        & $> 5$ GeV                \\
                       & \ipChiSq                     & $> 9$                    \\
                       & \PID{$K$}                    & $> 2\;(K)$, $< 4\;(\pi)$ \\
        \midrule
        \Dz            & \pt                          & $> 2000$ MeV             \\
                       & $|\pi\;\pt|+|K\;\pt|$        & $> 2.5$ GeV              \\
                       & $m$                          & 1830--1910 MeV           \\
                       & \anyChiSq{vertex}/ndf        & $< 10$                   \\
                       & \anyChiSq{FD}                & $> 25$                   \\
                       & \DIRA                        & $> 0.999$                \\
                       & Child pair \DOCA             & $< 0.1$ mm               \\

        \midrule
        \muon          & \ipChiSq                     & $> 16$                   \\

        \midrule
        \Dz\muon       & \anyChiSq{vertex}/ndf        & $< 15$                   \\
                       & \anyChiSq{FD}                & $> 50$                   \\
                       & \DIRA                        & $> 0.999$                \\
                       & \DOCA                        & $< 0.5$ mm               \\
        \bottomrule
    \end{tabular}
\end{table}

\begin{table}[htb]
    \caption{Additional online cuts.}
    \label{tab:cut-stripping}
    \centering
    \begin{tabular}{c|rll}
        \toprule
        {\bf Event-Level }      & {\bf Variable}               & {\bf Cuts}               \\
        \midrule
        GEC                     & nSPDhits                     & $< 600$                  \\
        PV cut                  & nPV                          & $\geq1$                  \\
        \toprule
        {\bf Particle }         & {\bf Variable}               & {\bf Cuts}               \\
        \midrule
        $K, \pi$                & \pt                          & $> 300$ MeV              \\
                                & \ptot                        & $> 2$ GeV                \\
                                & \anyChiSq{IP}                & $> 9$                    \\
                                & \isMuon                      & \texttt{False}           \\
                                & \PID{$K$}                    & $> 4\;(K)$, $< 2\;(\pi)$ \\
                                & GhostProb                    & $< 0.5$                  \\
        \midrule
        \Dz                     & $|\pi\;\pt|+|K\;\pt|$        & $> 2.5$ GeV              \\
                                & $m - m_\text{PDG}$           & $< 80$ MeV               \\
                                & \anyChiSq{vertex}/nof        & $< 4$                    \\
                                & \anyChiSq{FD}                & $> 25$                   \\
                                & \DIRA                        & $> 0.999$                \\
        \midrule
        \muon                   & \ptot                        & $> 3$ GeV               \\
                                & \ipChiSq                     & $> 16$                  \\
                                & isMuon                       & \texttt{True}           \\
                                & \PID{$\mu$}                  & $> -200$                \\
                                & GhostProb                    & $< 0.5$                 \\
        \midrule
        \Dz\muon                & $m$                          & 0--10 GeV               \\
                                & \anyChiSq{vertex}/ndf        & $< 6$                   \\
                                & \DIRA                        & $>0.999$                \\
        \bottomrule
    \end{tabular}
\end{table}
%%%%


Additional offline selection with stricter vertex quality and PID requirements,
mostly aligned with the \emph{online} selection in LHCb run 1,
are applied to both \Dz and \Dstar channels.
%%%%
Furthermore, only the right-sign $D^{(*)}\mu$ candidates are kept.
%%%%
The initial motivation was to bring 2016 data on par with run 1 data so
distributions of the same variables in 2016 and run 1 can be compared in a more
meaningful way,
providing check points for the development of this analysis.
%%%%
Later, it is checked with a survey of selection efficiencies of different decay
modes on MC cocktail,
whose relative yields adjusted according fitted yields,
that the run 1 selection is still optimal.

The offline cuts that are shared among \Dz and \Dstar channels are listed
in \cref{tab:offline-cut-common}.
The additional \Dz only cuts are listed in \cref{tab:offline-cut-d0},
The \Dstar in \cref{tab:offline-cut-dst}.

Due to the fact that both \Dz and \Dstar channel have the same online selection,
some candidates will appear in both,
making the two channels correlated.
%%%%
This correlation is removed by a \Dstar veto procedure in the \Dz channel,
and is discussed in \cref{ref:sel:tech:veto}.

The standard LHCb \muon PID has low efficiency at low \pt.
However, previous analysis (\cite{LHCb-ANA-2020-056}) showed that a tighter
\muon PID reduces other charged species misidentified (misID) as \mu, reducing
the systematic uncertainty due to misID effects.
Therefore, an offline \muon PID method, termed UBDT, is developed to reduce
misID while maintain a flat \pt response.
A brief overview on this method is provided in \cref{ref:sel:tech:ubdt}.

\begin{table}[htb]
    \caption{Offline cuts shared among \Dz and \Dstar channels.}
    \label{tab:offline-cut-common}
    \centering
    \begin{tabular}{c|rll}
        \toprule
        {\bf Event-Level }  & {\bf Variable}               & {\bf Cuts}               \\
        \midrule
        GEC                 & nSPDhits                     & $< 450$\parnote{
            This is to reduce correlation for \emph{TIS and TOS} candidates to make
            trigger emulation more robust.
            See \cref{ref:emulation-for-to-mc:correlation-tos-tis} for more details.
        }                                                                             \\
        \toprule
        {\bf Particle}      & {\bf Variable}               & {\bf Cuts}               \\
        \midrule
        $K, \pi$            & \pt                          & $> 500$ MeV              \\
                            & HLT1 TOS track \pt           & $> 1.7$ GeV              \\
                            & \ptot                        & $< 200$ GeV\parnote{
                                This is to make \ptot consistent with extended
                                \pidcalib binning.
                            }                                                         \\
                            & \anyChiSq{IP}                & $> 45$                   \\
                            & \isMuon                      & \texttt{False}           \\
                            & \PID{$K$}                    & $> 4\;(K)$, $< 2\;(\pi)$ \\
                            & GhostProb                    & $< 0.5$                  \\
        \midrule
        \Dz                 & \pt                          & $> 2$ GeV                \\
                            & $|\pi\;\pt|+|K\;\pt|$        & $> 1.4$ GeV              \\
                            & $\log(\IP)$\parnote{
                                \IP in terms of the \PrmVtx.
                                That is, the reconstructed \Dz is inconsistent
                                of coming from \PrmVtx directly.
                            }                              & $> -3.5$                 \\
                            & \anyChiSq{IP}                & $> 9$                    \\
                            & \anyChiSq{vertex}/ndf        & $< 4$                    \\
                            & \anyChiSq{FD}                & $> 250$                  \\
                            & \DIRA                        & $> 0.9998$               \\
        \midrule
        \muon               & \ptot                        & 3-100 GeV                \\
                            & $\log_{10}(1 - \vec{p}_\mu \cdot \vec{p}_i / (|p_\mu||p_i|))$,
                              $i \in \{K, \pi\}$
                                                           & $> -6.5$                 \\
                            & $\eta$                       & 1.7--5                   \\
                            & \ipChiSq                     & $> 45$                   \\
                            & GhostProb                    & $< 0.5$                  \\
                            & \PID{$\mu$}                  & $> 2$                    \\
                            & \PID{$e$}                    & $< 1$                    \\
                            & \UBDT\parnote{
                                This is a BDT-based \muon PID algorithm,
                                which is discussed in \cref{ref:sel:tech:ubdt}.
                            }
                                                           & $> 0.25$                 \\
        \bottomrule
    \end{tabular}
    \begin{flushleft}
        \parnotes
    \end{flushleft}
\end{table}

\begin{table}[htb]
    \caption{Offline cuts exclusive to \Dz channel.}
    \label{tab:offline-cut-d0}
    \centering
    \begin{tabular}{c|rll}
        \toprule
        {\bf Particle}  & {\bf Variable}               & {\bf Cuts}      \\
        \midrule
        \Dz\muon        & $m$                          & $< 5200$ MeV    \\
                        & {\footnotesize$\begin{aligned}
                                \text{Max}(
                                    &|\Delta m_\text{veto,1} - 145.454|, \\
                                    &|\Delta m_\text{veto,2} - 145.454|
                                )
                           \end{aligned}$}\parnote{
                            \label{parnote:dst-veto}
                            To reduce overlapping events between
                            \Dz and \Dstar channels.
                            Discussed in \cref{ref:selection:tech:veto}.
                        }                              & $> 4$ MeV       \\
                        & $\Delta m_\text{alt hypo}$\parnoteref{parnote:dst-veto}
                                                       & $> 165$ MeV     \\
                        & transverse flight distance   & $< 7$ mm        \\
                        & \anyChiSq{vertex}/ndf        & $< 6$           \\
                        & \DIRA                        & $> 0.9995$      \\
        \bottomrule
    \end{tabular}
    \begin{flushleft}
        \parnotes
    \end{flushleft}
\end{table}

\begin{table}[htb]
    \caption{Offline cuts exclusive to \Dstar channel.}
    \label{tab:offline-cut-dst}
    \centering
    \begin{tabular}{c|rll}
        \toprule
        {\bf Particle}  & {\bf Variable}                 & {\bf Cuts}    \\
        \midrule
        slow \pion     & GhostProb                       & $< 0.25$      \\
        \midrule
        \Dstarp        & $|m - m_\Dz - 145.454|$\parnote{
                           This is to require $m_\Dstarp$ to close to
                           its PDG mass,
                           up to a reconstruction effect on
                           $m_\Dz$.
                       }                                 & $< 2$ MeV     \\
                       & \anyChiSq{vertex}/ndf           & $< 10$        \\
        \midrule
        \Dz\muon       & \anyChiSq{vertex}/ndf           & $< 6$         \\
                       & \DIRA\parnote                   & $> 0.999$     \\
        \midrule
        \Dstarp\muon   & $m$                             & $< 5280$ MeV  \\
                       & transverse flight distance      & $< 7$ mm      \\
                       & \anyChiSq{vertex}/ndf           & $< 6$         \\
                       & \anyChiSq{vertex}               & $< 24$        \\
                       & \DIRA                           & $> 0.9995$    \\
        \bottomrule
    \end{tabular}
    \begin{flushleft}
        \parnotes
    \end{flushleft}
\end{table}


\subsection{Wrong sign samples with a ``real'' muon}
\label{ref:sel:data:ws}

The wrong sign samples are used to model combinatorial backgrounds which are
most dominate in the $D^{(*)}\mu$ pairs,
as only one-sided mass cut is applied on them, as opposed to a two-sided
mass window cut
(e.g. $m < 5280$ instead of $5280 - \epsilon < m < 5280 + \epsilon$).
This is because the momenta and energies of the neutrinos are not taken into
account,
the invariant mass of $D^{(*)}\mu$ pairs might be far away from known $B$ meson
masses so \emph{no suitable} mass window cut can be applied.
For \Dz channel, the \emph{unphysical}\footnote{
    Recall that the right-sign sample are reconstructed as the following decay:
    $\Bm \rightarrow \Dz (\rightarrow \Km \pip) \mun$,
    so \Dz\mup pairs are unphysical.
} \Dz\mup pairs are purely of combinatorial
nature and is the only dominate combinatorial and is referred as
\emph{wrong-sign \mu}.
For \Dstar channel, a \Dstarp may randomly combine with a
\emph{wrong-sign \mup}; alternatively, a \Dz may combine with a
\emph{wrong-sign slow \pim} to form a \Dstarm.

The online selection for wrong sign samples are the same as the right sign,
as described in \cref{ref:sel:data:rs}.
The offline selection is also mostly identical, with differences listed below:

\begin{itemize}
    \item \Dz channel, wrong-sign \mu:
        The wrong-sign \Dz\mup candidates are used in place of the right-sign
        \Dz\mun.
    \item \Dstar channel, wrong-sign \mu:
        The wrong-sign \Dz\mup candidates are used.
    \item \Dstar channel, wrong-sign \pi:
        The right-sign \Dz\mun candidates are used.
        Furthermore, a wrong-sign slow \pim is selected so the \Dstar has the
        wrong-sign.
\end{itemize}


\subsection{Fake muon samples}

The fake \muon samples are used to model the misidentification of a charged
particle to a \muon.
Both right-sign and wrong-sign misID samples share the same selections
as the respective ``real'' \muon sample,
except for \muon PID, which is replaced by an explicit \muon veto
(not \isMuon).
