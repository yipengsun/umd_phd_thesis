\section{MC simulation}
\label{ref:sel:mc}

A large number of simulated events are used to model the signal, normalization,
and partially reconstructed backgrounds in data.
To reduce simulation time, tracker-only (TO) MC, which has all but the tracking
system of the detector turned off (set to passive material), are used.
Additional small full simulation (FullSim) samples are produced for the year
2016 to aid emulations of the missing detector responses (mainly for trigger).
The emulation procedures are discussed in \cref{ref:mc-emulation}.

MC samples used for both \Dz and \Dstar channels are listed in
\cref{tab:mc-d0-dst}, with TO and FullSim numbers listed separately.
Some samples contribute to both channels due to feed down\footnote{
    This means that the selection procedure was unable to find the
    additional slow \pion required to reconstruct a \Dstar, thus only a
    \Dz was constructed and the candidate is \emph{feed down} from \Dstar
    channel to \Dz channel.
}, and are indicated in the same table\footnote{
    As a remark, the TO and FullSim have different simulation conditions,
    which are denoted by \textbf{SimXXX}.
    The differences are technical and can be ignored for this thesis.
}.


% vim: set ft=none:


% This table is based on the output of:
%   ./size_mc_samples.py -m detail -i 12573012 11574021 12773410
\begin{landscape}
\begin{table}[p]
    \centering
    \caption{
        List of MC sample used in this analysis.
        Numbers are event on disk, not total simulated numbers.
    }
    \label{tab:mc-d0-dst}
    \scriptsize

    \begin{tabular}{c|c|c|c|c|r|r|r|r}
        \toprule
        {\bf Group} &
        {\bf MC ID} &
        {\bf Decay mode} &
        {\bf Channels}  &
        {\bf Cocktail}  &
        {\bf\centering \makecell{2016 Sim09j \\ FullSim}} &
        {\bf\centering \makecell{2016 Sim09k \\ tracker-only}} &
        {\bf\centering \makecell{2017 Sim09k \\ tracker-only}} &
        {\bf\centering \makecell{2018 Sim09k \\ tracker-only}} \\
        \midrule
        % \endhead
        norm.        & 12573012
                     & $\Bm \rightarrow \Dz \mun \neumb$
                     & \Dz & no
                     %%%%
                     & 3,564,053
                     & 45,564,529
                     & 47,965,869
                     & 64,386,408
                     \\
                     %%%%
                     & 11574021
                     & $\Bzb \rightarrow \Dstarp \mun \neumb$
                     & both & no
                     %%%%
                     & 3,012,029
                     & 85,470,057
                     & 81,075,745
                     & 103,168,826
                     \\
                     %%%%
                     & 12773410
                     & $\Bm \rightarrow \Dstarz \mun \neumb$
                     & \Dz & no
                     %%%%
                     & 4,003,029
                     & 129,200,391
                     & 134,602,735
                     & 169,664,396
                     \\
                     %%%%
        \midrule
        sig.         & 12573001
                     & $\Bm \rightarrow \Dz \taum \neutb$
                     & \Dz & no
                     %%%%
                     & 233,966
                     & 4,152,856
                     & 3,405,213
                     & 4,317,050
                     \\
                     %%%%
                     & 11574011
                     & $\Bzb \rightarrow \Dstarp \taum \neutb$
                     & both & no
                     %%%%
                     & 593,350
                     & 17,217,664
                     & 18,008,069
                     & 25,341,935
                     \\
                     %%%%
                     & 12773400
                     & $\Bm \rightarrow \Dstarz \mun \neumb$
                     & \Dz & no
                     %%%%
                     & 528,616
                     & 9,813,636
                     & 10,289,156
                     & 15,091,483
                     \\
                     %%%%
        \midrule
        $D^{**}$     & 11874430
                     & $\Bzb \rightarrow \D^{**+} \mun \neumb$
                     & both & $D_1^+,D_1^{'+}, D_2^{*+}, D_0^{*+}$
                     %%%%
                     & 2,865,898
                     & 46,653,556
                     & 45,469,066
                     & 58,082,278
                     \\
                     %%%%
                     & 11874440
                     & $\Bzb \rightarrow \D^{**+} \taum \neutb$
                     & both & $D_1^+,D_1^{'+}, D_2^{*+}, D_0^{*+}$
                     %%%%
                     & 35,322
                     & 375,581
                     & 519,245
                     & 561,800
                     \\
                     %%%%
                     & 12873450
                     & $\Bm \rightarrow \D^{**0} \mun \neumb$
                     & both & $D_1^0,D_1^{'0}, D_2^{*0}, D_0^{*0}$
                     %%%%
                     & 2,333,268
                     & 37,417,148
                     & 117,729,837
                     & 48,051,754
                     \\
                     %%%%
                     & 12873460
                     & $\Bm \rightarrow \D^{**0} \taum \neutb$
                     & both & $D_1^0,D_1^{'0}, D_2^{*0}, D_0^{*0}$
                     %%%%
                     & 112,138
                     & 618,529
                     & 598,526
                     & 744,330
                     \\
                     %%%%
        \midrule
        $D^{**}_H$   & 12675011
                     & $\Bm \rightarrow \D^{**0}_H (\rightarrow \Dz\pi\pi) \mun \neumb$
                     & \Dz & Ref.~\cite{LHCb-ANA-2020-056}
                     %%%%
                     & 419,829
                     & 6,204,217
                     & 7,215,251
                     & 8,573,968
                     \\
                     %%%%
                     & 11674401
                     & $\Bzb \rightarrow \D^{**+}_H (\rightarrow \Dz\pi\pi) \mun \neumb$
                     & \Dz & Ref.~\cite{LHCb-ANA-2020-056}
                     %%%%
                     & 353,015
                     & 6,997,221
                     & 6,746,518
                     & 12,956,078
                     \\
                     %%%%
                     & 12675402
                     & $\Bm \rightarrow \D^{**0}_H (\rightarrow \Dstarp\pi\pi) \mun \neumb$
                     & both & Ref.~\cite{LHCb-ANA-2020-056}
                     %%%%
                     & 294,872
                     & 5,560,586
                     & 4,739,776
                     & 6,162,695
                     \\
                     %%%%
                     & 11676012
                     & $\Bzb \rightarrow \D^{**+}_H (\rightarrow \Dstarp\pi\pi) \mun \neumb$
                     & both & Ref.~\cite{LHCb-ANA-2020-056}
                     %%%%
                     & 290,791
                     & 4,824,507
                     & 4,834,264
                     & 8,353,204
                     \\
                     %%%%
                     & 12875440
                     & $\Bm \rightarrow \D^{**0}_H (\rightarrow \Dstarz\pi\pi) \mun \neumb$
                     & \Dz & Ref.~\cite{LHCb-ANA-2020-056}
                     %%%%
                     & 450,907
                     & 7,840,307
                     & 7,983,973
                     & 10,039,996
                     \\
                     %%%%
        \midrule
        $D^{**}_s$   & 13874020
                     & $\Bsb \rightarrow \D^{**+}_s (\rightarrow \Dz\Kp) \mun \neumb$
                     & \Dz & no
                     %%%%
                     & 259,166
                     & 1,654,215
                     & 1,732,571
                     & 2,214,625
                     \\
                     %%%%
                     & 13674000
                     & $\Bsb \rightarrow \D^{**+} \mun \neumb$
                     & \Dstar & $D_{s1}^{'+}, D_{s2}^{*+}$
                     %%%%
                     & 142,549
                     & 1,498,067
                     & 1,531,966
                     & 2,070,074
                     \\
                     %%%%
        \midrule
        $DDX$        & 11894600
                     & $\Bzb \rightarrow \Dz X_c (\rightarrow X' \mump \neumb) X$
                     & \Dz & Ref.~\cite{LHCb-ANA-2020-056}
                     %%%%
                     & 1,062,375
                     & 36,888,835
                     & 37,826,396
                     & 49,768,186
                     \\
                     %%%%
                     & 12893600
                     & $\Bm \rightarrow \Dz X_c (\rightarrow X' \mump \neumb) X$
                     & \Dz & Ref.~\cite{LHCb-ANA-2020-056}
                     %%%%
                     & 1,196,059
                     & 21,854,863
                     & 22,577,692
                     & 28,900,392
                     \\
                     %%%%
                     & 11894200
                     & $\Bzb \rightarrow \Dz D_s^- (\rightarrow X' \taum \neutb) X$
                     & \Dz & Ref.~\cite{LHCb-ANA-2020-056}
                     %%%%
                     & 84,533
                     & 1,231,353
                     & 1,145,771
                     & 1,367,357
                     \\
                     %%%%
                     & 12893610
                     & $\Bm \rightarrow \Dz D_s^- (\rightarrow X' \taum \neutb) X$
                     & \Dz & Ref.~\cite{LHCb-ANA-2020-056}
                     %%%%
                     & 255,725
                     & 2,818,348
                     & 2,625,208
                     & 4,834,213
                     \\
                     %%%%
                     %%%%
                     & 11894610
                     & $\Bzb \rightarrow \Dstarpm X_c (\rightarrow X' \mump \neumb) X$
                     & \Dstar & Ref.~\cite{LHCb-ANA-2020-056}
                     %%%%
                     & 1,188,596
                     & 16,188,308
                     & 13,480,119
                     & 16,837,410
                     \\
                     %%%%
                     & 12895400
                     & $\Bm \rightarrow \Dstarpm X_c (\rightarrow X' \mump \neumb) X$
                     & \Dstar & Ref.~\cite{LHCb-ANA-2020-056}
                     %%%%
                     & 365,215
                     & 7,000,178
                     & 7,459,109
                     & 6,798,676
                     \\
                     %%%%
                     & 11894210
                     & $\Bzb \rightarrow \Dstarp D_s^- (\rightarrow X' \taum \neutb) X$
                     & \Dstar & Ref.~\cite{LHCb-ANA-2020-056}
                     %%%%
                     & 113,422
                     & 1,678,170
                     & 1,283,823
                     & 1,631,217
                     \\
                     %%%%
                     & 12895000
                     & $\Bm \rightarrow \Dstarp D_s^- (\rightarrow X' \taum \neutb) X$
                     & \Dstar & Ref.~\cite{LHCb-ANA-2020-056}
                     %%%%
                     & 117,696
                     & 899,320
                     & 1,555,844
                     & 1,282,503
                     \\
                     %%%%
        \bottomrule
    \end{tabular}
\end{table}
\end{landscape}



Both TO and FullSim MC samples are loosely filtered to keep events in LHCb
acceptance, with filtering cuts, mostly to save storage space,
listed in \cref{tab:gen-cut}.
%%%%
The online and offline selection are aligned with that of right sign real \muon
sample,
with the understanding that trigger and PID responses are emulated offline and
applied as weights (for L0 trigger and PID) and cuts
(for HLT1 and HLT2 trigger).
%%%%
Additional truth level requirements\footnote{
    For MC samples, each reconstructed particle is associated with a true
    particle ID. The truth level requirements (truth-matching) demands that
    the true ID is the same as specified in the simulation.
    For example, for a candidate from  MC $\Bm \rightarrow \Dz\mun$ decay,
    the reconstructed \muon is required to have a true ID of a MC true \muon.
} are placed to ensure the selected decay
chain is compatible with simulation specification.
This effectively removes ghost and misidentified tracks, providing a consistent
modeling and ensuring orthogonality between MC samples.

\begin{table}
    \caption{List of MC filtering cuts at generator level.}
    \label{tab:gen-cut}
    \centering
    \begin{tabular}{c|rll}
        \toprule
        {\bf Particle}  & {\bf Variable}               & {\bf Cuts}      \\
        \midrule
        \kaon, \pion    & $p_x / p_z$                  & -0.38--0.38     \\
                        & $p_y / p_z$                  & -0.28--0.28     \\
                        & $\theta$                     & $> 0.01$~rad    \\
                        & \pt                          & $> 250$~MeV     \\
        \midrule
        \muon           & $p_x / p_z$                  & -0.38--0.38     \\
                        & $p_y / p_z$                  & -0.28--0.28     \\
                        & $\theta$                     & $> 0.01$~rad    \\
                        & \ptot                        & $> 2950$~MeV    \\
        \midrule
        \Dz             & \pt                          & $> 15$~GeV      \\
                        & \ptot                        & $> 2450$~MeV    \\
        \bottomrule
    \end{tabular}
\end{table}

The heavy $D^{**}$ and $DDX$ MC are cocktails, with the same compositions as
listed in \cite{LHCb-ANA-2020-056}.
The heavy $D^{**}$ MC suffer from the problem that the $\piz \piz$ decays
are not simulated, inconsistent of the isospin limit.
The missing $\piz \piz$ contribution is modelled by randomly selecting $\frac{1}
{3}$ of $\pip \pim$ candidates and treating them as \emph{isolated} regardless
of their isolation BDT scores.
The isolation BDT, briefly discussed in \cref{ref:sel:tech:iso-bdt},
is used to estimate the probability of nearby charged tracks,
unused in selection procedure, to originate from the same $\D^{(*)}\mu$ vertex.
