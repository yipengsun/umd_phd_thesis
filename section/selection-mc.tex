\section{MC simulation}
\label{ref:sel:mc}


A large number of simulation (MC) samples is used to
model the signal, normalization, and partially reconstructed backgrounds in
data.
To save computation time, tracker-only (TO) MC, which has all but the tracking
system of the detector turned off (set to passive material), are used.
These MC samples are produced in Sim09k simulation condition, separate by years
(2015, 2016, 2017, 2018) and polarities (MagUp, MagDown), using
\pythia{8} generator.
Additional small full simulation (FullSim) samples are produced in Sim09j
condition for year 2016 to aid emulations of the missing detector responses
(mainly trigger).
The emulation procedures are discussed in \cref{sec:emulation-for-to-mc}.

Both TO MC and FullSim samples are loosely filtered to keep events in LHCb
acceptance, but without filtering on stripping line selections.
The filtering selections are listed in \cref{tab:gen-cut}, and are referred as
generator level cuts.

\begin{table}
    \caption{List of MC generator-level cuts.}
    \label{tab:gen-cut}
    \centering
    \begin{tabular}{c|rll}
        \toprule
        {\bf Particle}  & {\bf Variable}               & {\bf Cuts}      \\
        \midrule
        \kaon, \pion    & $p_x / p_z$                  & -0.38--0.38     \\
                        & $p_y / p_z$                  & -0.28--0.28     \\
                        & $\theta$                     & $> 0.01$~rad    \\
                        & \pt                          & $> 250$~MeV     \\
        \midrule
        \muon           & $p_x / p_z$                  & -0.38--0.38     \\
                        & $p_y / p_z$                  & -0.28--0.28     \\
                        & $\theta$                     & $> 0.01$~rad    \\
                        & \ptot                        & $> 2950$~MeV    \\
        \midrule
        \Dz             & \pt                          & $> 15$~GeV      \\
                        & \ptot                        & $> 2450$~MeV    \\
        \bottomrule
    \end{tabular}
\end{table}

MC samples used for both \Dz and \Dstar channels are listed in
\cref{tab:mc-d0-dst}, with TO and FullSim numbers listed separately.
Some samples contribute to both channels due to feed down\footnote{
    This means that the reconstruction algorithm was unable to find the
    additional slow \pion required to reconstruct a \Dstar, thus only a
    \Dz was constructed and the candidate is \emph{feed down} from \Dstar
    channel to \Dz channel.
}, and are indicated in the same table.

The heavy $D^{**}$ and $DDX$ MC are cocktails, with the same compositions as
listed in \cite{LHCb-ANA-2020-056}.
The heavy $D^{**}$ MC suffer from the problem that the $\piz \piz$ decays
are not simulated, inconsistent of the isospin limit.
The $\piz \piz$ contribution is modelled by randomly selecting $\frac{1}{3}$ of
$\pip \pim$ candidates and treating them as \emph{isolated} regardless of their
isolation BDT scores.
The isolation BDT is briefly discussed in \cref{sec:selection:skims}.

The reconstructed particles are truth-matched to ensure they have the correct
true particle IDs per MC mode.
This effectively removes ghost and misidentified tracks, providing a consistent
modelling and ensuring orthogonality between MC samples.
