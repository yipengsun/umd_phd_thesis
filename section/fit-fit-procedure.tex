\section{Fit procedure}
\label{ref:fit:procedure}

The fit procedure is broken into 3 steps:

\begin{enumerate}
    \item Pre-control:
        Performed on control samples (1OS, 2OS, DD).
        In this step, all but the \Kstar variations are turned off.
    \item Control:
        Performed on control samples (1OS, 2OS, DD).
        All but the signal and normalization variations are turned on.
        The fitted parameters from previous steps are loaded along with their
        errors as a starting point for the control fit.
    \item Signal:
        Performed on signal sample (ISO).
        The variations for $D^{**}, D_H^{**}$, and misID DiF are loaded as
        constraints to the signal fit;
        the $DDX$ variations are set to constant with fitted value from previous
        step,
\end{enumerate}

The fit models for each step are listed in TAB.
Four efficiencies computed from external values are needed in the fit.
(Copy Phoebe's "Efficiencies" section here when summarize fit model)
An index for these parameters is tabulated in \cref{tab:selected-fit-params}.

\begin{table}[!htb]
    \centering
    \caption{Index for selected fit parameters.}
    \label{tab:selected-fit-params}
    \begin{tabular}{r|l|l}
        \toprule
        {\bf Parameter(s)} & {\bf Utility} & {\bf Reference} \\
        \midrule
        $N_{D\mu}$, $N_{\Dstar\mu}$    & overall norm.  & \cref{tmpl:sig-norm} \\
        \RD, \RDstp, \RDstz & frac. of signal to related norm. &  \\
        \midrule
        $n^{0}_{D^{**}},f^{*}_{D^{**}}$ & mental aid & \cref{tmpl:dstst} \\
        \bottomrule
    \end{tabular}
\end{table}
